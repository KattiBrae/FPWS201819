Als Erstes werden die versuchsabhängigen Konstanten gemessen.
Dazu zählen die Breite und die Höhe des eckigen Stabes sowie der Durchmesser des Runden, zusätzlich das Gewicht des Stabes und das Gewicht der zu verwendenen Masse.
Desweiteren wird die Länge der Stäbe und der Abstand zwischen Gewicht und Auflagepunkt benötigt.
\subsection{Einseitige Einspannung}
Der eckige Stab wird auf einer Seite der Apparatur eingespannt.
Mit Hilfe von Messuhren wird zunächst an 10 Stellen eine Nullmessung durchgeführt.
Anschließend wird ein Gewicht angehangen und die Messung zur Bestimmung der Durchbiegung wiederholt.
Die selbe Messung wird nun für einen runden Stab durchgeführt.

\subsection{Zweiseitige Einspannung}
Der eckige Stab wird auf beiden Seiten aufgelegt.
Nach der Nullmessung wird wieder ein Gewicht angehangen, diesmal in die Mitte der Länge des Stabes.
