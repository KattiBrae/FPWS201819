Der Versuch zeigt, dass ein Lock-In-Verstärker Störfrequenzen aus einer Spannung filtert und die gewünschte Spannung deutlich verstärkt.
Die erste Messreihe zeigt die Spannungsverläufe ohne Störsignal.
Werden diese Bilder der Spannungsverläufe mit den Spannungsverläufen mit Störsignal verglichen, fällt das Rauschen in den Bildern auf, aber die Form der ungestörten Spannung ist eindeutig wiederzuerkennen.
Für beide Messreihen werden die Spannungsverläufe in Abhängigkeit der Phasenverschiebung $\varphi$ gemessen.
Daraus ergibt sich ein Wert für die Eingangsspannung $U_{0}$:
\begin{align*}
  \overline{U_{\text{0,ref}}}    &=& \SI{160.861}{V}\\
  \overline{U_{\text{0,rausch}}} &=& \SI{244.798}{V}.\\
\end{align*}
Die tatsächliche Eingangsspannung $U_{0}$ liegt bei
\begin{equation*}
  U_{0}=\SI{2.28}{V}.
\end{equation*}
Aus dem Verhältnis der gemessenen Eingangsspannungen und der tatsächlichen Eingangsspannung lässt sich die Verstärkung abschätzen.
\begin{align*}
  \frac{  \overline{U_{\text{0,ref}}     }}{U_{0}} &=& \SI{ 70.55}{}\\
  \frac{  \overline{U_{\text{0,rausch}}  }}{U_{0}} &=& \SI{107.37}{}.\\
\end{align*}
\\Die Messwerte zur Überprüfung der Rauschunterdrückung mit der Photodetektorschaltung sind teilweise stark verfälscht und werden daher nicht weiter beachtet.
Das Problem ist, dass die Messwerte zunächst fallen und dann wieder ansteigen, dies entspricht aber nicht den physikalischen Gesetzmäßigkeiten.
Die Vermutung liegt nahe, dass eine Lichtquelle, die nicht in das Experiment gehört, einen Einfluss auf den bewegten Photodetektor hat.
Ein ähnliches Problem führt ebenfalls zu stark schwankenden Werten der Intensität.
Das Ausschalten der Raumbeleuchtung zwischen der Aufnahme zweier Werte zeigt, dass die Raumbeleuchtung zu spontanen Intensitätschwankungen von mehr als $U=\SI{0.08}{V}$ führt.
Die Werte werden kontinuierlich mit dem Hintergrundrauschen der Raumbeleuchtung aufgenommen.
Als maximaler Abstand wird
\begin{equation*}
  x_{max}=\SI{132.5}{cm}
\end{equation*}
gemessen.
