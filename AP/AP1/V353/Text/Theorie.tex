\subsection{Allgemeine Relaxationsgleichung}

Eine Relaxationszeit ist die Zeit,
die zwischen dem Auslenken eines Systems aus einem Ausgangszustand und dem Zurückkehren in den Ausgangszustand vergeht.
Dies gilt jedoch nicht für schwingende Systeme.
In diesem Versuch wird die Entladung eines Kondensators über einen Widerstand als Beispiel für ein Relaxationsphänomen gewählt.
\\Nun wird zunächst allgemein die Änderungsgeschwindigkeit einer Größe A betrachtet.
Sie ist abhängig vom Grenzwert der Größe $A(\infty)$.
\\Aus
\begin{equation}
  \frac{dA}{dt} = c[A' - A(\infty)]
  \label{eqn:danachdt}
\end{equation}
\\folgt
\begin{equation}
  A(t) = A(\infty) + [A(0)-A(\infty)] \cdot e^{ct}
  \label{eqn:allrelax}
\end{equation}
\\Dabei muss c < 0 sein, da A(t) sonst nicht beschränkt ist.

\subsection{Entladekurve eines Kondensators}
Auf einem Kondensator mit der Kapazität C und der Ladung Q liegt eine Spannung $U_{C}$ an:
\begin{equation*}
  U_{C}= \frac{Q}{C}.
\end{equation*}
\\Die Spannung führt zu dem Strom I durch den Widerstand R (Ohm'sches Gesetz):
\begin{equation*}
  I = \frac{U_{C}}{R}.
\end{equation*}
\\Auf dem Kondensator verändert sich die Ladung Q pro Zeiteinheit um
\begin{equation*}
  dQ = -Idt.
\end{equation*}
\\Daraus ergibt sich die Gleichung
\begin{equation}
  \frac{dQ}{dt} = - \frac{Q(t)}{RC},
  \label{eqn:dgl1}
\end{equation}
\\die eine ähnliche Form wie Formel \eqref{eqn:danachdt} hat.
\\Daraus ergibt sich unter Beachtung des nicht erreichbaren Grenzwerts Q(\infty)
\begin{equation}
  Q(t)= Q(0) \cdot e^{-\frac{t}{RC}}
  \label{eqn:Entladekurve}
\end{equation}

\subsection{Relaxationsverhalten bei angelegten Wechselspannungen}
Allgemein kann eine Wechselspannung durch folgende Gleichung ausgedrückt werden:
\begin{equation}
  U(t)= U_{0} \cdot \cos(\omega t).
  \label{eqn:acdcein}
\end{equation}
\\Es bildet sich eine Phasenverschiebung \phi zwischen der eingehenden Wechselspannung und der vom Kodensator verzögerten ausgehenden Wechselspannung aus.
\\Damit kann man die ausgehende Wechselspannung als
\begin{equation}
  U_{C}(t)= A(\omega) \cdot \cos(\omega t + \phi (\omega))
  \label{eqn:acdcaus}
\end{equation}
\\beschrieben werden, mit der Kondensatorspannungsamplitude A.
\\Des weiteren gilt
\begin{equation}
  I(t) = \frac{dQ}{dt} = C \frac{d U_{C}}{dt}
  \label{eqn:dgl2}
\end{equation}
\\Aus den Kirchhoff'schen Gesetzen, der Formel \eqref{eqn:dgl1}, Formel \eqref{eqn:dgl2} und weiteren Überlegungen folgt dann
\begin{equation}
  A(\omega) = \frac{U_{0}}{\sqrt{1+ \omega^2 R^2 C^2}}.
  \label{eqn:ampkond}
\end{equation}

\subsection{Integrationsverhalten des RC-Kreises}
Ein RC-Kreis kann unter gewissen Vorraussetzungen als Integrator dienen.
Hierfür muss $\omega \gg \frac{1}{RC}$ sein.
Folgende Gleichung
\begin{equation*}
  U(t)= U_{R}(t) + U_{C}(t) = R \cdot I(t) + U_{C}(t)
\end{equation*}
\\führt zu
\begin{equation*}
  U(t)= RC \cdot \frac{d U_{C}(t)}{dt} + U_{C}(t).
\end{equation*}
Unter der Vorraussetzung $\omega \gg \frac{1}{RC}$ folgt dann:
\begin{equation}
  U_{C}(t) = \frac{1}{RC} \int_{0}^{t} \! U(t')dt'
  \label{eqn:integrator}
\end{equation}
