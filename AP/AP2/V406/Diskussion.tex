Im ersten Einzelspaltexperiment werden folgende Werte ermittelt:
\begin{align*}
  &\text{Theorie}&&b=\SI{0,022e-3}{m}\\
  &\text{Errechnet}&&b=\SI{0,045\pm0,001e-3}{m}\\
  &\text{Abweichung}&&\sigma = \pm\SI{104,5}{\percent}.
\end{align*}
Im zweiten Einzelspaltexperiment werden diese Werte ermittelt:
\begin{align*}
  &\text{Theorie}&&b=\SI{0,1e-3}{m}\\
  &\text{Errechnet}&&b=\SI{0.29\pm0.01e-3}{m}\\
  &\text{Abweichung}&&\sigma = \pm\SI{190}{\percent}.
\end{align*}
Beim Doppelspalt wird anhand von der Abbildung \ref{fig:2} deutlich,
dass die Formel des Doppelspaltes \ref{eqn:doppelspalt} deutlich besser zu den Messwerten passt als die
Formel zum Einzelspalt \ref{eqn:einzelspalt}.
Die Messungen des Experimentes haben große Fehlerquellen.
Es gibt Schwankungen in der Dunkelstrahlung.
Der Messaparat misst die Intensität über eine bestimmte Breite, dass heißt, er integriert die Intensität über einen Bereich.
Somit es eine genaue Bestimmung nicht möglich.
Desweiteren können paralaxe Fehler bei der Messung der Intensität auftreten.
