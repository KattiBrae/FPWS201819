Initial lässt sich sagen, dass der Versuch gut verlaufen ist.
Es werden Linsen mit den vom Hersteller angegebenen Brennweiten
\begin{align*}
  f_{\text{1, theo}} &=& \SI{0.1}{m}\\
  f_{\text{2, theo}} &=& \SI{0.05}{m}\\
  f_{\text{3, theo}} &=& \SI{-0.1}{m}\\
\end{align*}
verwendet.
Die Überprüfung der Linsengleichung liefert die gemessenen Brennweiten
\begin{align*}
  \overline{f_{1}} &=& \SI{0.09896 \pm 0.00076}{m}\\
  \overline{f_{2}} &=& \SI{0.05019 \pm 0.00069}{m}.\\
\end{align*}
Allgemein berechnet sich eine relative Abweichung $q$ über
\begin{align*}
  d= \frac{x_{\text{exp}}-x_{\text{theo}}}{x_{\text{theo}}}.
\end{align*}
Die Abweichungen der beiden gemessenen Brennweiten zu den vom Hersteller angegebenen Brennweiten liegen bei
\begin{align*}
  q_{1, Linse} &=& \SI{1.04}{\%}\\
  q_{2, Linse} &=& \SI{0.38}{\%}.\\
\end{align*}
In den beiden Abbildungen \ref{fig:f100} und \ref{fig:f50} ist die Messgenauigkeit dargestellt.
Darin treffen die Graphen alle ungefähr im Punkt $(f_{1}, f_{1})$ (Abb. \ref{fig:f100}) bzw. im Punkt $(f_{2}, f_{2})$ (Abb. \ref{fig:f50}) aufeinander.
\\Bei der Methode nach Bessel wird die erste Linse ($f_{\text{1, theo}} = \SI{0.1}{m}$) erneut verwendet.
Es wird die Brennweite
\begin{align*}
  \overline{f_{1}}= \SI{0.1013 \pm 0.0008}{m}.
\end{align*}
gemessen.
Diese Brennweite weicht um
\begin{align*}
  q_{1, Bessel} &=& \SI{1.30}{\%}\\
\end{align*}
von dem vom Hersteller angegebenen Wert der Brennweite ab.
\\Die Abweichung zum angegebenen Herstellerwert ist bei der Methode nach Bessel größer, als mit der reinen Überprüfung der Linsengleichung.
\\Die Untersuchung der chromatischen Abberation ergibt
\begin{align*}
  \overline{f_{\text{blau}}} &=& \SI{0.1012 \pm 0.0008}{m}\\
  \overline{f_{\text{rot}}}  &=& \SI{0.1022 \pm 0.0008}{m}.\\
\end{align*}
Die Brennweite des blauen Lichts ist $q_{\text{Abberation}}= \SI{0.99}{\%}$ kleiner als die Brennweite des roten Lichts.
Daraus lässt sich schließen, dass das blaue Licht unter einem kleineren Winkel gebrochen wird als das rote Licht.
\\Bei der Messung nach Abbe werden die Brennweite $f$ eines Linsensystems und die beiden Hauptebenen $h$ und $h'$ bestimmt.
Die gemittelte Brennweite wird als
\begin{align*}
  \overline{f_{\text{Abbe}}}= \SI{0.1569 \pm 0.0161}{m}
\end{align*}
bestimmt.
Die Hauptebenen liegen bei
\begin{align*}
  h &=& \SI{0.1355 \pm 0.0158}{m}\\
  h' &=& \SI{-0.0818 \pm 0.0051}{m}.\\
\end{align*}
\\Die Fehlerquellen für diese Messungen sind unter anderem die sphärische Abberation und die systematischen Fehler durch das subjektive Scharfstellen des Bildes.
Letzteres ist bereits während der Messungen auffällig, daher werden einzelne Messreihen nur von einem Versuchsdurchführenden durchgeführt.
\FloatBarrier
