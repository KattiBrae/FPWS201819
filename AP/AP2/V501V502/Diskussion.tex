\subsection{Diskussion zu V501}
In der ersten Messreihe gibt es einige Fehlerquellen.
Mögliche Fehlerquellen sind Spannungsschwankungen des Stromnetzes, parallaxe Fehler beim Ausrichten des leuchtenden Punktes auf den Linien des Gitternetzes,
störende elektrische und magnetische Felder und weiteres.
Die störenden Felder bewirken eine Ablenkung der Teilchen.
Insbesondere bei dem Versuch 502, der zuvor stattfand, werden diffuse Störungen des Erdmagnetfelds festgestellt.
Daraus wird gefolgert, dass die umgebenden Elektrogeräte und andere Dinge störende elektrische oder magnetische Felder emittieren.
Dennoch verläuft die Ablenkung des Elektronenstrahls wie erwartet.
Der Zusammenhang zwischen der Beschleunigungsspannung $U_{B}$, der Ablenkspannung $U_{d}$ und der Ablenkung $D$ ist linear.
\\Die Steigung des Graphen, in dem die Empfindlichkeit $D/U_{d}$ gegen $1/U_{B}$ aufgetragen ist, beträgt:
\begin{equation*}
  a=\SI{0,3730 \pm 0,0115}{m}
\end{equation*}
\\Die zweite Messreihe zur Erzeugung der stehenden Wellen mit dem selbst gebauten Oszilloskop verläuft weniger genau.
Probleme bei der Messung sind, dass die Frequenz der Wellen auf dem Schirm nach dem Einstellen der Frequenz der Sägezahnspannung $\nu_{Säg}$ schnell wieder zunimmt.
Dies führt dazu, dass die Wellen für einige Augenblicke tatsächlich stehen, sich dann aber wieder bewegen.
Außerdem ist der Regler für $\nu_{Säg}$ schwer einzustellen.
Der Spannungsgenerator des angelegten und untersuchten Wechselstroms gibt an, Spannungen mit Frequenzen von $\nu_{theo}= 80-90 \si{Hz}$ auszugeben.
Initial bei der Messung fällt auf, dass die Werte für $\nu_{Säg}$ ungefähr Vielfachen von $\SI{25}{Hz}$ entsprechen.
Dies passt jedoch nicht zu den vom Gerät angegebenen Frequenzen.
Als Frequenz des Wechselstroms wird
\begin{equation*}
  \nu_{exp} = \SI{75,146}{Hz}
\end{equation*}
\\ermittelt.
Die relative Messabweichung $f$ vom gemessenen Wert zum niedrigsten vom Gerät angegebenen Wert $\nu=\SI{80}{Hz}$ berechnet sich über
\begin{equation}
  f=\frac{x_{exp}-x_{theo}}{x_{theo}}
  \label{eq:abw}
\end{equation}
\\und beläuft sich auf $6,07\%$.

\FloatBarrier
\subsection{Diskussion zu V502}
Zunächst lässt sich sagen, dass die Messwerte nah an den Literaturwerten liegen.
Dennoch liegen mögliche Fehlerquellen vor.
Dazu gehören Störungen des induzierten Magnetfelds, Spannungsschwankungen des Stromnetzes, parallaxe Fehler und massive Störungen des Erdmagnetfeldes.
Die Störungen des Erdmagnetfelds führen zur Ablenkung der Elektronen und haben ihren Ursprung meist bei elektrischen Geräten, die elektrische und magnetische Felder emittieren.
Das Erdmagnetfeld kann außerdem durch die Beschaffenheit des Gebäudes abgeschirmt werden.
\\Der mithilfe von Literaturwerten \cite{taschenrechner} errechnete Wert der spezifischen Elektronenladung beläuft sich auf
\begin{equation*}
  \frac{e_{0, theo}}{m_{0, theo}}= \SI{1,759e11}{\frac{C}{kg}}.
\end{equation*}
\\Der Mittelwert der Messwerte zur spezifischen Elektronenladung ergibt sich zu:
\begin{equation*}
  \frac{e_{0}}{m_{0}}= \SI{2,821e11}{\frac{C}{kg}}.
\end{equation*}
\\Die Abweichungen des Mittelwerts der Messwerte zum Literaturwert errechnen sich mit Gleichung \eqref{eq:abw} zu $f=\SI{60,37}{\%}$.
\\Die zweite Messung beschäftigt sich mit dem lokalen Erdmagnetfeld.
Gerade die Messung der Magnetfeldrichtung ist abhängig vom Ort der Messung sehr unterschiedlich.
So ergibt sich bei der Messung eine Abweichung von bis zu $70°$ zu der Richtung, die als Norden bekannt ist und abgeschätzt wird.
Als Literaturwert wird
\begin{equation*}
    B_{tot, theo}=\SI{48}{\mu T}
\end{equation*}
verwendet \cite{4}.
\\Die gemessene totale Magnetfeldstärke beträgt:
\begin{equation*}
  B = \SI{0,053}{mT}=\SI{53}{\mu T}.
\end{equation*}
\\Die Abweichung dieser beiden Werte beläuft sich mit Gleichung \eqref{eq:abw} auf $f=\SI{9,4}{\%}$.
\FloatBarrier
