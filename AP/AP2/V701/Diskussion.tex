Initial lässt sich sagen, dass der Versuch wie erwartet verlaufen ist.
Die Abweichung berechnet sich allgemein durch
\begin{align*}
  f=\frac{x_{\text{1}} - x_{\text{2}}}{x_{\text{2}}}.
\end{align*}
Die Reichweiten werden als
\begin{align*}
x_{\text{15}} &=& \SI{13.881 \pm 0.242e-3}{m}\\
x_{\text{20}} &=& \SI{14.866 \pm 1.550e-3}{m}\\
\end{align*}
gemessen.
Die Abweichung der beiden Reichweiten beträgt
\begin{align*}
  f_{x} = \SI{6.63}{\%}.
\end{align*}
Als Literaturwert wird $x_{\text{Theo}}=\SI{29e-3}{m}$ \cite{2} verwendet.
Die gemessenen Werte weichen um
\begin{align*}
  f_{\text{x}_{\text{15}}} &=& \SI{52.13}{\%}\\
  f_{\text{x}_{\text{20}}} &=& \SI{48.74}{\%}\\
\end{align*}
vom Literaturwert ab.
Die Energieverluste belaufen sich auf
\begin{align*}
  \frac{dE_{\text{15}}}{dx}  &=& \SI{-119.78612 \pm 1.957012041}{\frac{MeV}{m}}\\
  \frac{dE_{\text{20}}}{dx}  &=& \SI{-103.33675 \pm 3.752994338}{\frac{MeV}{m}}.\\
\end{align*}
Die Abweichung der beiden gemessenen Energieverluste beträgt
\begin{align*}
  f_{\frac{dE}{dx}} = \SI{15.92}{\%}.
\end{align*}
Diese Abweichungen lassen sich einerseits durch den möglicherweise ungenauen Fit und andererseits durch die parallaxen Fehler und das nicht perfekte Vakuum erklären.
Im ersten Versuchsteil enthalten die Graphen ein angedeutetes Plateu und einen linearen Abfall.
Die linearen Regressionen werden nur über bestimmte Teilbereiche der Messdaten gemacht.
So wird bei dem Abstand $x=\SI{15e-3}{m}$ über vier Messwerte gefittet, wobei auch hier der mittlere Messwert der letzten fünf Werte ausgelassen wird, da er eindeutig nicht dem sonstigen Verlauf entspricht.
Bei dem Abstand $x=\SI{20e-3}{m}$ wird die lineare Regression über die letzten sieben Messwerte ausgeführt.
Dadurch, dass gerade bei dem geringeren Abstand sehr wenige Messwerte zum fitten genutzt werden, sind durch geringe Abweichungen der einzelnen Messdaten bereits große Auswirkungen in der Steigung möglich.
Die Skala zum Messen des Abstands ist außen auf dem Glaszylinder positioniert, während die Probe ca. $\SI{3}{cm}$ davon entfernt war.
Dies lässt massive parallaxe Fehler zu.
Außerdem ist der Unterdruck in dem Glaszylinder nicht perfekt.
\\Im zweiten Versuchsteil ist der Verlauf weniger eindeutig.
Der Verlauf der Messwerte passt auf den ersten Blick zu keiner der beiden Verteilungen.
Die Gaußverteilung ist deutlich steiler und schmaler als die gemessenen Werte und hat einen deutlich höheren Peak am Mittelwert $\mu$.
Die Poissonverteilung ist flacher und schmaler als die Messwerte, aber der Unterschied ist nicht so gravierend, wie zu der Gaußverteilung.
Die Messwerte entsprechen also eher einer Poissonverteilung als einer Gaußverteilung.
Mit mehr Messdaten ist ein eindeutigerer Kurvenverlauf passend zur Poissonverteilung zu erwarten.
Sie ist typisch für Messgrößen von zufälligen Ereignissen in bestimmten Zeitintervallen, wie in diesem Versuch die Zählrate.

\FloatBarrier
