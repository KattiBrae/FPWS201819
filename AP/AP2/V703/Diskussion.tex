Die Messwerte des Experimentes sind sehr Fehleranfällig. Das Geiger-Müller-Zählrohr hat zu Beispiel bei gleicher Spannung und gleicher $\beta$-Quelle in zwei Messungen unterschidliche Werte aufgenommen.
Ein perfektes Geiger-Müller-Zählrohr sollte keinen Platouanstieg haben. Das verwendete Geiger-Müller-Zählrohrhat einen Anstieg von $m=\SI{1,781 \pm 0,038}{\frac{1}{min V}}$.
Am Ende des Platous soll ein staker Anstieg beobachtet werden. In der Messung bliebt dieser jedoch aus. Das Ende des Platou ist somit nicht klat definiert.
Desweiteren ist das Ablesen des Ampermeters durch Schwankungen erschwert. Zusätzlich kommen beim Ablesen noch paralaxe Fehler hinzu.
Beim ablesen vom Oszilloskop sind die Werte sehr ungenau da diese nur geschätzt werden können.
Die ermittelte Totzeit mit Hilfe des Oszilloskop beträgt  $T_{U=540V} = (3,381 \pm 0,012)\cdot 10^{-6} {\text{min}} $.
dagegen ergiebt die Totzeitbestimmung durch die zwie Quellenmethode eine Totzeit von $T_{U=540V} = (6,8\cdot10^{-6}\pm 434\cdot10^{−4}){\text{min}}$.
Die beiden Werte sind sehr unterscheidlich es wird daher ein systematischer Fehler angenommen.
Die ausgerechneten Werte für Q können mit dem Bereich 4 der Abb. \ref{fig:charakteristik} verglichen werden. Die Größenortnung entspricht daher dem Theoriewert.
