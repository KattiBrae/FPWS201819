Das Experiment ist in sich gut verlaufen, die Messwerte entsprechen den erwarteten Größenordnungen.
Mögliche Fehlerquellen sind die statistischen Fehler, fehlerhafte Messgeräte und Näherungen in den Herleitungen, die nicht genau genug sind.
So ist zum Beispiel der Ablauf der Wechselwirkungen zu komplex um sie zu berechnen und einige der genutzten Formeln sind nur für bestimmte Bereiche empirisch bestimmt.
\\Die Absorptionskoeffizienten von Eisen und Blei sind beide größer, als der theoretische, Compton'sche Absorptionskoeffizient:
\begin{align*}
  \mu_{\text{Fe}} &= \SI{48.245 \pm 1.054}{\frac{1}{m}}\\
  \mu_{\text{com, Fe}} &= \SI{56.647}{\frac{1}{m}}\\
  \mu_{\text{Pb}} &= \SI{108.944 \pm 3.761}{\frac{1}{m}}\\
  \mu_{\text{com, Pb}}&= \SI{69.360}{\frac{1}{m}}.\\
\end{align*}
Damit lässt sich annehmen, dass in beiden Materialen neben dem Comptoneffekt auch der Photoeffekt auftritt.
Im Blei kommt aber der Photoeffekt deutlich häufiger vor als im Eisen.
Die relative Abweichung $f$ in Prozent lässt sich über
\begin{align*}
  f= \frac{x_{\text{exp}}-x_{\text{theo}}}{x_{\text{theo}}}\cdot 100
\end{align*}
berechnen.
Die Abweichungen berechnen sich zu
\begin{align*}
  f_{\text{Fe}}=\SI{14.83}{\%}\\
  f_{\text{PB}}=\SI{57.07}{\%}\\.
\end{align*}
Die große Abweichung bei den Blei-Absorbern lässt sich durch die deutlich höhere Dichte vom Blei erklären.
Die Strahlung tritt deutlich schneller in die Wechselwirkungen mit den vielen Kernen, Hüllen und Elektronen ein.
\\Bei der $\beta$-Strahlung berechnet sich die maximale Energie der Strahlung zu
\begin{align*}
  E_{\text{max}}= \SI{0.287 \pm 0.050}{MeV}.
\end{align*}
Der Literaturwert \cite{tc2} beträgt
\begin{align*}
  E_{\text{max, theo}}= \SI{0.293}{MeV}.
\end{align*}
Der relative Fehler $f_{\text{E}_{\text{max}}}$ der Messung beläuft sich zu
\begin{align*}
  f_{\text{E}_{\text{max}}}=\SI{2.05}{\%}.
\end{align*}
Damit ist die Messung der maximalen Emissionsenergie von $^{99}\text{Tc}$ über die $\beta$-Strahlung ziemlich genau.
