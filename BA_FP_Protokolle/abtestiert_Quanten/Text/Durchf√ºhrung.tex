%Aufbau
%- Schiene mit Lautsprecher und Mikrofon, zwischen denen die die Röhren eingeklemmt werden
%- Röhren verschiedener Längen (75mm, 50mm, 12.5mm)
%- Scheiben verschiedener Durchmesser (10mm, 13mm, 16mm)
%- Computer
%
\subsection{Aufbau}
Der Aufbau setzt sich zusammen aus einer Schiene mit einem Lautsprecher und einem Mikrofon, zwischen denen die Röhren eingeklemmt werden.
Lautsprecher und Mikrofon schließen dicht mit den Röhren ab.
Die Röhren bestehen aus Aluminium und sind in drei verschiedenen Ausführungen vorhanden.
Während der Durchmesser jeweils gleich ist, gibt es drei verschiedene Längen mit $L_{75} = \SI{75}{mm}$, $L_{50} = \SI{50}{mm}$ und für die letzten Versuchsteile $L_{12.5} = \SI{12.5}{mm}$.
Außerdem gibt es drei verschiedene Arten Blenden, Scheiben mit einem Loch in der Mitte, nämlich die Lochdurchmesser $d_{10} = \SI{10}{mm}$, $d_{13} = \SI{13}{mm}$ und $d_{16} = \SI{16}{mm}$.
Die initiale Datenaufnahme und -verarbeitung wird mit einem Rechner gemacht, an den sowohl Mikrofon als auch Lautsprecher angeschlossen sind.\\
%
%Durchführung
\subsection{Durchführung}
In der Durchführung des Versuchs werden immer Frequenzspektren aufgenommen.
Dies wird vom Rechner gesteuert.
Dabei können verschiedene Parameter vorgegeben werden, darunter die Größe des aufgenommenen Frequenzbereiches, die Messzeit pro Schritt und die Schrittgröße $s$.
Wenn nicht weiter angegeben, wird die Messung mit den Schritten $s = \SI{10}{Hz}$ durchgeführt.\\
%
%- 1: Schallgeschwindigkeit:
%    - QA1: Seite 6 unten (Experiment): 8x Frequenzspektrum für verschiedene Rohrlängen (75mm-Röhren), kleine Frequenzbereiche
%    - QA1: Seite 14 oben (Setup): 1x Frequenzspektrum mit kleinen Schritten (1x Steps 2Hz)
%           hier sollen die Graphen der Resonanz und der Quantenmechanik geplottet werden, guck mal bei Seite 12 oben
%           "The resonance peak A(ω) ... has the same shape as the ... quantum mechanical eigenstate with finite lifetime (eqn. 1.18). In the following figure the two line-shapes ... are plotted for comparison with the parameter ... ."
% ICH GLAUBE DAS ZWEITE MÜSSEN WIR DOCH NICHT MACHEN
%
\paragraph{Aufgabe 1}
In der ersten Messung wird als Einleitung die Schallgeschwindigkeit in Luft gemessen.
Dazu werden acht Frequenzspektren mit den $L_{75}-$Röhren mit der Schrittlänge $ s= \SI{10}{Hz}$ und dem Frequenzbereich $\SI{6000}{Hz} - \SI{9000}{Hz}$ aufgenommen.
Nach jeder Messung wird die Anzahl der Röhren um eine erhöht, bis acht Röhren erreicht sind.
Aus den verschiedenen Frequenzspektren werden die jeweiligen Resonanzfrequenzen notiert.
Außerdem wird ein größeres Frequenzspektrum ($\SI{5000}{Hz}-\SI{14000}{Hz}$) mit kleiner Schrittlänge ($s = \SI{2}{Hz}$) für eine einzelne $L_{50}-$Röhre aufgenommen.\\
%
%- 2: Eindimensionaler Festkörper / Dispersion Frequenz (k)
%    - QA4: Seite 2 unten (Experiment): 1x Frequenzspektrum 12x 50mm Röhren, großer Frequenzbereich
\paragraph{Aufgabe 2}
Der zweite Versuchsteil beschäftigt sich mit dem eindimensionalen Festkörper.
Zur Messung der Dispersionsrelation $\omega(k)$ in diesem wird ein Frequenzspektrum von 12 $L_{50}-$Röhren mit einem großen Frequenzbereich ($\SI{0}{Hz}-\SI{12000}{Hz}$) aufgenommen.\\
%
%- 3: Eindimensionaler Festkörper / Bandlücken
%    - QA4: Seite 3 mittig (Experiment), unten (Experiment): 3x Frequenzspektren mit 12x 50mm Röhren mit Blenden a 10, 13, 16mm Durchmesser
\paragraph{Aufgabe 3}
Danach werden in eine zusammengesetzte Röhre aus 12 $L_{50}-$Röhren je 11 Blenden eingefügt um die Bandlücken zu veranschaulichen.
Die Messung wird für die drei verschiedenen Durchmesser der Blenden in einem Frequenzbereich von $\SI{0}{Hz}-\SI{12000}{Hz}$ durchgeführt.\\
%
%- 4: Eindimensionaler Festkörper / Anzahl und Breite der Einheitszellen, Gitterkonstante
%    - QA4: Seite 4 mittig (Experiment): 1x Frequenzspektrum mit Blenden 16mm aber weniger Röhren (10x 50mm)
\paragraph{Aufgabe 4}
Zur Vermessung der Einheitszellen im eindimensionalen Festkörper werden 10 $L_{50}-$Röhren mit entsprechender Anzahl von $d_{16}-$Blenden mit dem Frequenzbereich $\SI{0}{Hz}-\SI{12000}{Hz}$ vermessen.\\
%
%- 5: Eindimensionaler Festkörper / das selbe wie in Aufgabe 4, reziproker Raum, Bandlücken
%    - QA4: Seite 4 mittig (Experiment): 2x Frequenzspektrum mit Blenden 16mm aber weniger Röhren, unterschiedliche Röhren (10x 50mm, 8 x 50mm)
\paragraph{Aufgabe 5}
Nun werden 8 $L_{50}-$Röhren mit $d_{16}-$Blenden und 8 $L_{75}-$Röhren mit $d_{16}-$Blenden mit dem Frequenzbereich $\SI{0}{Hz}-\SI{12000}{Hz}$ vermessen, um die Gitterkonstante des eindimensionalen Festkörpers zu bestimmen.\\
%
%- 6: Molekülkette Eigenzustände / ???
%    - QA4: Seite 9 oben (Experiment): 1x Einzelne Röhre (50mm)
\paragraph{Aufgabe 6}
Der dritte Versuchsteil behandelt eine Molekülkette, die aus einzelnen Atomen aufgebaut wird.
Hier wird zunächst das Frequenzspektrum einer einzelnen $L_{50}-$Röhre in einem großen Frequenzbereich von $\SI{0}{Hz}-\SI{22000}{Hz}$ gemessen.\\
%
%- 7: Molekülkette Eigenzustände / ???
%    - QA4: Seite 9 mittig (Experiment): 1x Einzelne Röhre (75mm)
\paragraph{Aufgabe 7}
Anschließend wird eine einzelne $L_{75}-$Röhre mit dem selben Frequenzbereich aufgenommen.\\
%
%- 8: Molekülkette Eigenzustände / Simulation einzelnes Molekül
%    - QA4: Seite 9 mittig (Experiment): 1x 50mm+10er Blende+50mm
%    - QA4: Seite 9 mittig (Experiment): 1x 50mm+13er Blende+50mm
%    - QA4: Seite 9 mittig (Experiment): 1x 50mm+16er Blende+50mm
\paragraph{Aufgabe 8}
Zur Simulation eines einzelnen Atoms werden nun zwei $L_{50}-$Röhren durch eine Blende getrennt.
Die Messung wird mit allen drei Blendenvarianten mit jeweils dem Frequenzbereich $\SI{0}{Hz}-\SI{12000}{Hz}$ durchgeführt.\\
%
%- 9: Molekülkette Eigenzustände / Simulation Molekülkette
%    - QA4: Seite 9 unten (Experiment): "Molekülanzahl" erhöht, 9 verschiedene Anzahlen von Einheitszellen
\paragraph{Aufgabe 9}
Danach wird die "Atomanzahl" erhöht, indem weitere Blenden und Röhren in den Aufbau eingefügt werden.
Dazu werden weiterhin abwechselnd $L_{50}-$Röhren und Blenden aufgebaut.
Die Messung wird mit den folgenden Konstellationen durchgeführt:\\
\begin{align*}
\text{Einheitszellen} && \text{Anzahl Röhren}  && \text{Blenden}   && \text{Anzahl Blenden}  && \text{Frequenzbereich}\\
  3                   &&  6             && d_{10}           &&  5             && \SI{0}{Hz}-\SI{12000}{Hz} \\
  3                   &&  6             && d_{13}           &&  5             && \SI{0}{Hz}-\SI{12000}{Hz} \\
  3                   &&  6             && d_{16}           &&  5             && \SI{0}{Hz}-\SI{12000}{Hz} \\
  4                   &&  8             && d_{10}           &&  7             && \SI{0}{Hz}-\SI{12000}{Hz} \\
  4                   &&  8             && d_{13}           &&  7             && \SI{0}{Hz}-\SI{12000}{Hz} \\
  4                   &&  8             && d_{16}           &&  7             && \SI{0}{Hz}-\SI{12000}{Hz} \\
  6                   && 12             && d_{10}           && 11             && \SI{0}{Hz}-\SI{12000}{Hz} \\
  6                   && 12             && d_{13}           && 11             && \SI{0}{Hz}-\SI{12000}{Hz} \\
  6                   && 12             && d_{16}           && 11             && \SI{0}{Hz}-\SI{12000}{Hz}.\\
\end{align*}
%
%- 10: Superstrukturen / Molekülkette mit Defekten
%    - QA4: Seite 10 mittig (Experiment): 12x 50mm Röhren mit 13mm und 16mm Blenden (abwechselnd)
\paragraph{Aufgabe 10}
In der darauffolgenden Messung werden Defekte in die Atomkette eingebracht, indem in einem Aufbau aus 12 $L_{50}-$Röhren abwechselnd $d_{13}-$ und $d_{16}-$Blenden eingefügt werden.\\
%
%- 11: Superstrukturen / Molekülkette mit Defekten
%    - QA4: Seite 10 mittig (Experiment): 5x 50mm und 5x 75mm (abwechselnd) mit 16mm Blenden
\paragraph{Aufgabe 11}
Danach werden statt unterschiedlichen Blenden unterschiedlich lange Röhren abwechselnd in den Aufbau gebracht.
Somit sind dann 5 $L_{50}-$ und 5 $L_{75}-$ Röhren durch $d_{16}-$Blenden getrennt.
%
%- 12: Defekte und Bandstruktur (Halbleiterdotierung)
%    - QA4: Seite 11 oben (Experiment): 4 Frequenzspektren, 12x 50mm mit 16mm Blenden, Defekte (= 1x 12.5mm bzw. 1x 75mm ) ersetzen einzelne Röhren (Stelle 3 bzw Stelle 8)
\paragraph{Aufgabe 12}
In der letzten Messreihe werden einzelne Röhren als Defekte in die 12 $L_{50}-$Röhren mit $d_{16}-$Blenden eingebracht:
\begin{align*}
\text{Defektstelle} && \text{Defektart}  \\
    3               &&   \SI{12.5}{mm}   \\
    3               &&   \SI{75}{mm}     \\
    8               &&   \SI{12.5}{mm}   \\
    8               &&   \SI{75}{mm}   . \\
\end{align*}
%
