Überhaupt
- große Schwankungen des Amperemeters
- hohe Präzision in der Justage erforderlich
Stabilitätsbedingung
- sphärisch-sphärisch
  + Lokale Abweichung der Messwerte von der Kurve ist gigantisch
  + Theoriekurve ist in den Messwerten kaum zu erkennen
  + Unerklärliche Fehler (Fehler in Python?)
- sphärisch-flach
  + Zwei Messungen, da der Laser so aus der Justage gebracht wurde, dass ein Neustart einfacher war
  + Erste Messung voll daneben (Fehler wtf, keine Steigung)
  + Zweite Messung ganz okay (Fehler okay :) )
  + Bei den späteren Messwerten ist das Justieren sehr schwierig/hohe Präzision gefordert
TEM-Grundmode
- Gauß trifft Messwerte genau (Abbildung)
- Fehler okay
- Strahltaille nur positiv
Erste Mode
- Fit schwierig
- trifft Messwerte nicht sehr genau (Fehler größer als Parameter)
- Intensitätsverlust auf einer Seite
  + Möglicherweise Draht nicht genau mittig
Polarisation
- Intensitätsverlust auf einer Seite
  + Liegt möglicherweise am Aufbau? Vielleicht zu doll auf den Tisch gelehnt
- Minima und Maxima genau zu erkennen
  + Fit-Extrema und Messwert-Extrema decken sich nicht völlig
