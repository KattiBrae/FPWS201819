Wie bereits beschrieben besteht ein Laser aus einer optischen Pumpe, einem Lasermedium und eimem Resonator.
Das Lasermedium ist im Versuch ein HeNe Gasgemisch welches ein verhältniss von 5 zu 1 besitzt.
Das Lasermaterial ist dabei das Neon.
Helium wird als Pumpgas verwendet da dieses aus metastabielen Zuständen seine Anregungsenergie durch Stöße zweiter Art an das Neon abgeben kann.
Das Gas befindet sich bei ca. einem Torr in einem Laserrohr welches an den Enden mit einem Brewsterfenster abgeschlossen ist um einen
definierte Polarisatitionsrichtung mit möglichst verlustfreiem Austritt des Strahls zu erhalten.
An das Laserrohr ist ebenfals an eine Hochspannung angeschlossen um die optische Pumpe zu realisieren welche mittels entladung eine Besetzungsinversion verursacht.
Das Laserrohr und die Resonatorspiegel befinden sich auf einer optischen Schiene.
Durch das Verändern der Spiegelabstände kann nun die länge des Resonators verändert werden.
Um den Resonator richtig einzustellen befindet sich ebenfals ein Justierlaser auf der optischen Schiene.
Für den Versuch stehen verschiedene Resonatorspiegel zur verfügung welche sich in der Krümung und der Oberfläche unterscheiden.
Nach dem einjusieren des Lasers mit hilfe des Justierlasers wird die Stabilitätsbedingung überprüft.
Dafür wird mit einer Photodieode die Strahlenintensität bei unterschiedlich großen Resonatorspiegelabständen gemessen.
Die Messung wird für unterschiedliche Resonatorspiegel  (sphärisch-flachen und flachen-flachen) durchgeführt.
In der zweiten Messreihe wird ein Draht in der Laserstrahl zwische Laserrohr und Resonator gebracht.
Mithilfe einer Photodiode wird die Intensitätsverteilung der Mode gemessen.

Mit einem Polarisationsfilter welcher in den Lasergang eingebaut wird, wird nun die Polarisation des Lasers gemessen.
Die Photodiode misst abermals die Intensität diesmal in Bezug auf den Winkel des Polarisationsfilters.

%Im letzten versuchstleil wird die Wellenlänge mithilfe eines Gitters bestimmt.
