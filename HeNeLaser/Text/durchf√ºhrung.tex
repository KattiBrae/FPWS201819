Wie bereits beschrieben besteht ein Laser aus einer optischen Pumpe, einem Lasermedium und einem Resonator.
Das Lasermedium ist im Versuch ein HeNe-Gasgemisch mit einem Verhältnis von 5:1.
Das Lasermaterial ist dabei das Neon.
Helium wird als Pumpgas verwendet, da dieses aus metastabilen Zuständen seine Anregungsenergie durch Stöße zweiter Art an das Neon abgeben kann.
Das Gas befindet sich bei ca. einem Torr in einem Laserrohr, welches an den Enden mit je einem Brewsterfenster abgeschlossen ist, um einen
definierte Polarisatitionsrichtung mit möglichst verlustfreiem Austritt des Strahls zu erhalten.
An das Laserrohr ist ebenfalls an eine Hochspannung angeschlossen, um die optische Pumpe zu realisieren, welche mittels Entladungen eine Besetzungsinversion verursacht.
Das Laserrohr und die Resonatorspiegel befinden sich auf einer optischen Bank.
Durch das Verändern der Spiegelabstände kann nun die Länge des Resonators verändert werden.
Um den Resonator richtig einzustellen befindet sich ebenfalls ein Justierlaser auf der optischen Bank.
Für den Versuch stehen verschiedene Resonatorspiegel zur Verfügung, welche sich in der Krümmung und der Oberfläche unterscheiden.
Nach dem Einjusieren des Lasers mithilfe des Justierlasers wird die Stabilitätsbedingung überprüft.
Dafür wird mit einer Photodiode die Strahlenintensität bei unterschiedlich großen Resonatorspiegelabständen gemessen.
Die Messung wird für unterschiedliche Resonatorspiegel (sphärisch-flachen und sphärisch-sphärisch) durchgeführt.
In der zweiten Messreihe wird ein Draht in den Laserstrahl zwische Laserrohr und Resonator gebracht.
Mithilfe einer Photodiode wird die Intensitätsverteilung der Mode gemessen.
%
Mit einem Polarisationsfilter, welcher in den Lasergang eingebaut wird, wird nun die Polarisation des Lasers gemessen.
Die Photodiode misst abermals die Intensität diesmal in Bezug auf den Winkel des Polarisationsfilters.
%
Die verwendeten Spiegel haben die folgenden Eigenschaften:
\begin{align*}
\text{flach }      &&       -       && \\
\text{sphärisch }  && r_{1}=\SI{1.4}{m} && \\
\text{sphärisch }  && r_{2}=\SI{1.4}{m} && \text{halbdurchlässig}\\.
\end{align*}
%
Im letzten Versuchstleil wird die Wellenlänge mithilfe eines Gitters bestimmt.
Dazu wird der Abstand a zwischen Gitter und Schirm, der Abstand d zwischen Hauptmaximum und erstem Nebenmaximum und der Linienabstand g des Gitters mit $g= \SI{0.01}{mm}=\SI{1e-05}{m}$ notiert.
