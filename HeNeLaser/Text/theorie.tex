Ein Laser besteht aus einem Lasermedium, einer Pumpe und einem Resonator.
Durch Hohspannung wird dem Lasermedium die Pumpenergie zugeführt.
Das Pumpen bewirkt, dass des Lasermedium in einen angeregten Zustand übergeht.
Durch das Einstrahlen von Licht in der passenden Wellenlänge kommt es zur induzierten Emission.
Das angeregte Atom fällt auf sein Grundzustand zurück.
Die freiwerdende Energie wird dabei in Form von Photonen ausgesendet, die der einfallenden Strahlung in Energie, Phase und Ausbreitungsrichtung gleicht.
Das einfallende Licht wird auf diese Weise verstärkt.
Der Resonator besteht aus zwei Spiegeln wobei einer dieser Spiegel halb durchlässig ist um ein Teil des Lichtest auszukoppeln.
Die Spiegel werfen das austretende Laserlicht zurück in das Lasermedium.
Auf diese Weise wird der Weg des Lichtes im Lasermedium verlängert und es entsteht ein selbsterregnder Oszilator.
