efficiency\section{Auswertung}
\subsection{Energiekalibration}
Die Energiekalibration wird anhand der Vermessung eines $^{152}\symup{Eu}$-Spektrums (Abb. \ref{fig:eu_spectrum}) durchgeführt.
Die Messdaten werden mit Python 3.7.3 und den Biblitheken \textit{numpy}, \textit{scipy} und \textit{uncertainties} ausgewertet.
Ausgleichsrechnungen erfolgen mit \\\textit{scipy.optimize.curve\_fit}.
Über eine Peak-Picking-Funktion werden die größten Peaks in den Daten ausfindig gemacht und sind in Tabelle \ref{tab:eu_peaks} notiert.
\begin{figure}[h!]
  \centering
  \includegraphics[width=0.9\textwidth]{content/images/spektrum_europium.pdf}
  \caption{Das aufgenommene Spektrum über $T=\SI{2134}{s}$ von $^{152}\symup{Eu}$ mit markierten Peaks. Dargestellt ist die Zählrate gegen den zugehörigen Channel des MCA.}
  \label{fig:eu_spectrum}
\end{figure}
Zum $\gamma$-Zerfall des $^{152}\symup{Eu}$ werden Literaturwerte bezüglich der Emissionsenergien und der Emissionswahrscheinlichkeiten recherchiert \cite{nucleide}.
Dabei werden zunächst die Emissionsenergien mit mindestens $\SI{1}{\%}$ Emissionswahrscheinlichkeit rausgesucht.
Diese sind in Tabelle \ref{tab:eu_peaks} aufgeführt.
\begin{table}[h!]
  \centering
  \caption{Parameter zu allen vermessenen Peaks des $^{152}Eu$-Spektrums}
  \label{tab:EU-Peaks}
  \begin{tabular}{ r  r  r  r  r  r  }
    \bottomrule
            Peak & Channel(Peak)  & Counts  &  $E_{\gamma}$ / keV \cite{nucleide} & rel. Channel                                       & rel. Energie  \\ % & $a$                              &      $\mu$                         & $\sigma$                         & $b$                              \\
                 &                &         &                                     & $\frac{\text{Channel}}{\text{Channel(Peak 9)}}$    & $\frac{E_{\gamma}}{E_{\gamma}\text{(Peak 9)}}$ \\ % & $a$                              &      $\mu$                         & $\sigma$                         & $b$                              \\
    \midrule
            0    &  594           &  1219   &  $\SI{121.7817}{\nothing}$          & 0,087                                              & 0,087 \\
            1    &  1187          &  196    &  $\SI{244.6974}{\nothing}$          & 0,175                                              & 0,174 \\
            2    &  1667          &  372    &  $\SI{344.2785}{\nothing}$          & 0,245                                              & 0,245 \\
            3    &  1988          &  42     &  $\SI{411.1165}{\nothing}$          & 0,292                                              & 0,292 \\
            4    &  2149          &  43     &  $\SI{ 443.965}{\nothing}$          & 0,316                                              & 0,315 \\
            5    &  3765          &  66     &  $\SI{778.9045}{\nothing}$          & 0,554                                              & 0,553 \\
            6    &  4655          &  41     &  $\SI{ 964.079}{\nothing}$          & 0,685                                              & 0,685 \\
            7    &  5245          &  32     &  $\SI{1085.837}{\nothing}$          & 0,771                                              & 0,771 \\
            8    &  5371          &  33     &  $\SI{1112.076}{\nothing}$          & 0,790                                              & 0,790 \\
            9    &  6801          &  37     &  $\SI{1408.013}{\nothing}$          & 1,0                                                & 1,0 \\
    \toprule
  \end{tabular}
\end{table}
%                                &  \scriptsize{$\sfrac{mN}{sm}{a}$}            &  \scriptsize{$\sfrac{mN}{m}{a}$}          &  \scriptsize{$\sfrac{mN}{m}{a}$} &  \scriptsize{$\sfrac{mN}{m}{a}$} & \scriptsize{$\sfrac{mN}{m}{a}$} \\ \hline \hline
%    \rowcolor{Lightgray} \multicolumn{7}{|c|}{Referenzen}\\ \hline
%        \multirow{2}{*}{\ce{C3F8}}  & 1  & $\SI{ 1.3396 \pm 0.0202 e-3}{\nothing}{a}$  &  $\SI{ -0.5415 \pm 0.0558 }{\nothing} $  &  $\SI{ 0.00 }{\nothing}{a}$      &  $\SI{ 5.72 }{\nothing}{a}$   & $\SI{ 5.72 }{\nothing}{a}$    \\

Zur Kalibration werden die jeweiligen Daten auf den zugehörigen Wert des letzten sichtbaren Peaks normiert.
Entsprechend werden folgende Rechnungen ausgeführt:
\begin{align*}
	\text{rel. Energie } && E_{\text{rel.}} 				&& = && \frac{ E_{\text{Peak}} }{ E (\text{Peak=9}) } \\
	\text{rel. Channel } && \text{Channel}_{\text{rel.}} 	&& = && \frac{ \text{Channel} }{ \text{Channel} (\text{Peak=9}) }. \\
\end{align*}
Die relativen Größen sind in Abbildung \ref{fig:eu_kalibration} gegen die Counts aufgetragen.
Die drei Emissionsenergien, die im gemessenen Spektrum nicht als Peak ersichtlich sind und auch die geringsten Emissionswahrscheinlichkeiten aufweisen, werden aus den Daten der Literaturwerte entfernt.
\begin{figure}[h!]
  \centering
  \includegraphics[width=0.9\textwidth]{content/images/spektrum_europium_kali.pdf}
  \caption{Die relativen Größen $E_{\text{rel.}}$ und $\text{Channel}_{\text{rel.}}$, normiert auf den letzten sichtbaren Peak des $^{152}\symup{Eu}$-Spektrums, sind gegen die zugehörigen Counts aufgetragen.
  Die Peaks lassen sich nun den Spektrallinien des $^{152}\symup{Eu}$ zuordnen.}
  \label{fig:eu_kalibration}
\end{figure}
Anschließend werden die zugeordneten Energien der Peaks gegen die Channel der Peaks geplottet (Abbildung \ref{fig:kalibration}) und es wird eine lineare Regression der Form
\begin{equation}
	E = m \cdot \text{Channel} + n
\end{equation}
durchgeführt.
Als Parameter der Regression ergeben sich über \textit{curve\_fit}:
\begin{align*}
	m = \SI{0.20726(4)}{\kilo \electronvolt \per Channel}, && n = \SI{-1.22(17)}{\kilo \electronvolt}.
\end{align*}
\begin{figure}[h!]
  \centering
  \includegraphics[width=0.9\textwidth]{content/images/spektrum_europium_kali.pdf}
  \caption{Ausgleichsrechnung über den Zusammenhang der Channel des MCA und der Energien der $\gamma$-Teilchen.}
  \label{fig:kalibration}
\end{figure}


%\begin{itemize}
%	\item Spektrum geplottet, Counts gegen Channel
%	\item Errorbars?
%	\item Peaks finden lassen, Peaks markiert
%	\item Literaturwerte Energien rausgesucht mit mind. 1\% Emissionswahrscheinlichkeit (Quelle %http://www.nucleide.org/DDEP_WG/Nuclides/Eu-152_tables.pdf 2019-12-11, 22:35)
%	\item Spektrallinien $E$ normiert mit dem größten Wert der Energie: $\frac{E}{max(E)}$
%	\item Channel normiert mit dem letzten Peak $\frac{channel}{max(channel)}$
%	\item Daten mit normierter x-Achse geplottet: norm(E)-0-Diagramm, norm(channel)-Count-Diagramm
%	\item Nicht vorhandene Spektrallinien aus E und doppelte aus Peaks entfernt
%	\item Peak-Channel gegen Energien geplottet, Fit:
%\end{itemize}
%\begin{equation*}
%	E = m \cdot \text{Channel} + n
%\end{equation*}
%\begin{align*}
%	m = \SI{0.20726(4)}{\kilo \electronvolt \per Channel} && n = \SI{-1.22(17)}{\kilo \electronvolt}
%\end{align*}




\FloatBarrier
\subsection{Vollenergienachweiswahrscheinlichkeit}
\label{kap:vw}
Zur Bestimmung der Vollenergienachweiswahrscheinlichkeit $Q$ (engl.: \textit{efficiency}) des Detektors wird zunächst die Aktivität der Probe ausgerechnet.
Zwischen dem angegebenen Herstellungsdatum (01.10.2000) \cite{anleitung} der $^{152}\symup{Eu}$-Probe und dem Versuchstag (09.12.2019) sind $t = \SI{605484000 (54000)}{s}$ vergangen.
Die Halbwertszeit des Isotops beträgt $T_{\sfrac{1}{2}} = \SI{426.7 (5) e+06}{s}$ \cite{nucleide}.
Mit der Anfangsaktivität $A_{0} = \SI{4130 (60)}{Bq}$ ergibt sich über
\begin{equation}
	A = A_{0} \exp{\left( - \frac{\ln{(2)}}{T_{\sfrac{1}{2}}} t \right)} = \SI{1545(29)}{\per \second}
	\label{result:aktivität}
\end{equation}
die aktuelle Aktivität der Probe.
Weiterhin wird der eingenommene Raumwinkel des Detektors benötigt.
Dabei wird der Raumwinkel über die Geometrie eines Kegels berechnet:
\begin{align*}
	\frac{r}{h} = \tan{( \varphi / 2 )} \Leftrightarrow \varphi = 2 \arctan{(\frac{r}{h})} \\
	\frac{\Omega}{4 \pi} = \sin^2{\frac{\varphi}{2 \cdot 4}} =  \sin^2{ \left( \frac{1}{4} \arctan{(r/h)} \right)} = \SI{0.0069}{sr}.
	\label{result:raumwinkel}
\end{align*}
Die eingesetzten Größen für den Radius der Detektoroberfläche und Höhe des Kegels sind $r = \SI{22.5e-03}{m}$ und $h = \SI{80e-03}{m}$.
Die gesamte Messzeit des $^{152}\symup{Eu}$-Spektrums beträgt $T=\SI{2134}{s}$.
Damit kann nun $Q$ nun über Gleichung \eqref{eqn:efficiency} berechnet werden.
%\begin{equation}
%	Q = \frac{4 \pi}{\Omega} \frac{N_{\text{Peakinhalt}}}{ A T P_{E_{\gamma}}}.
%    \label{eqn:efficiency}
%\end{equation}
%$P_{E_{\gamma}}$ ist hier die Emissionswahrscheinlichkeit einer $\gamma$-Energie \cite{nucleide}.
%Der Parameter $N_{\text{Peakinhalt}}$ beschreibt nun die gesamte Zahl der Counts, die sich in einem Peak befindet.
Zur Berechnung der Peakinhalte werden die Peaks einzeln betrachtet und die Messdaten passend zu der erwarteten Gaußverteilung eines Peaks abgeschnitten (vgl. Abb. \ref{fig:einzelnergaussfit_0}).
\begin{figure}[h!]
  \centering
  \includegraphics[width=0.9\textwidth]{content/images/einzelnergaussfit_0.pdf}
  \caption{Vergrößerung des ersten Peaks mit Ausgleichsfunktion einer Gaußkurve zur Veranschaulichung der Gaußpeaks.}
  \label{fig:einzelnergaussfit_0}
\end{figure}
Die Inhalte der Peaks werden durch Aufsummation der Counts im jeweiligen angepassten Datenbereich berechnet.
Die Ergebnisse zu den jeweiligen Peaks sind in Tabelle \ref{tab:vw} notiert.
\begin{table}[h!]
  \centering
  \caption{Parameter zur Berechnung der Vollenergienachweiswahrscheinlichkeit anhand eines $^{152}Eu$-Spektrums. Weitere verwendete Größen sind: \\ $A=\SI{1545(29)}{\per \second}$, $\frac{\Omega}{4 \pi} = \SI{0.0069}{sr}$, $T=\SI{2134}{s}$.}
  \label{tab:}
  \begin{tabular}{ r  r  r  r  r  r  }
    \bottomrule
              $E_{\gamma}$ / keV \cite{nucleide} & $P$ \cite{nucleide}   & $P_{\text{Peakinhalt}}$    & Q in $10^{-3}$            \\ % & $a$                              &      $\mu$                         & $\sigma$                         & $b$                              \\
    \midrule
                $\SI{121.7817}{\nothing}$        & $\SI{28.41}{\nothing}$& $\SI{8233 (91)}{\nothing}$ & $\SI{12.70 (24)}{\nothing}$            \\ % & $\SI{54643  (699)}{a}$  & $\SI{593.88 (0.02}}{a}$      & $\SI{2.72080278 (0.02096791)}{a}$  & $\SI{6.65464370 (0.155416486)}{a}$ \\
                $\SI{244.6974}{\nothing}$        & $\SI{ 7.55}{\nothing}$& $\SI{1515 (39)}{\nothing}$ & $\SI{ 8.79 (16)}{\nothing}$            \\ % & $\SI{10583 (1892)}{a}$  & $\SI{1186.58105 (0.33278)}{a}$       & $\SI{3.08985566 (0.33289872)}{a}$  & $\SI{7.46589080 (0.347744578)}{a}$ \\
                $\SI{344.2785}{\nothing}$        & $\SI{26.59}{\nothing}$& $\SI{3152 (56)}{\nothing}$ & $\SI{ 5.19 (10)}{\nothing}$            \\ % & $\SI{25365 (2038)}{a}$  & $\SI{1666.69026 (0.16131)}{a}$       & $\SI{3.33202744 (0.16137103)}{a}$  & $\SI{7.26196957 (0.334424003)}{a}$ \\
                $\SI{411.1165}{\nothing}$        & $\SI{2.238}{\nothing}$& $\SI{ 324 (18)}{\nothing}$ & $\SI{ 6.34 (12)}{\nothing}$            \\ % & $\SI{ 1320 (1390)}{a}$  & $\SI{1989.13968353 (1.58221965)}{a}$ & $\SI{2.49555989 (1.58264810)}{a}$  & $\SI{7.60693051 (0.351207638)}{a}$ \\
                $\SI{ 443.965}{\nothing}$        & $\SI{ 2.80}{\nothing}$& $\SI{ 367 (19)}{\nothing}$ & $\SI{ 5.74 (11)}{\nothing}$            \\ % & $\SI{ 1839 (1909)}{a}$  & $\SI{2147.41246865 (1.92603485)}{a}$ & $\SI{3.08211825 (1.92667938)}{a}$  & $\SI{7.60363599 (0.351271659)}{a}$ \\
                $\SI{778.9045}{\nothing}$        & $\SI{12.97}{\nothing}$& $\SI{ 741 (27)}{\nothing}$ & $\SI{ 2.50  (5)}{\nothing}$            \\ % & $\SI{ 6632 (3679)}{a}$  & $\SI{3763.60345 (1.60581703)}{a}$    & $\SI{4.79839449 (1.60665042)}{a}$  & $\SI{7.56537596 (0.351196062)}{a}$ \\
                $\SI{ 964.079}{\nothing}$        & $\SI{14.50}{\nothing}$& $\SI{ 596 (24)}{\nothing}$ & $\SI{1.801 (33)}{\nothing}$            \\ % & $\SI{ 5190 (4028)}{a}$  & $\SI{4656.71298 (2.39004332)}{a}$    & $\SI{5.10185800 (2.39136059)}{a}$  & $\SI{7.58314830 (0.351416184)}{a}$ \\
                $\SI{1085.837}{\nothing}$        & $\SI{10.13}{\nothing}$& $\SI{ 403 (20)}{\nothing}$ & $\SI{1.743 (32)}{\nothing}$            \\ % & $\SI{ 2217 (3342)}{a}$  & $\SI{5244.21808 (4.04291370)}{a}$    & $\SI{4.46961617 (4.04488016)}{a}$  & $\SI{7.60853282 (0.351488937)}{a}$ \\
                $\SI{1112.076}{\nothing}$        & $\SI{13.41}{\nothing}$& $\SI{ 502 (22)}{\nothing}$ & $\SI{1.640 (30)}{\nothing}$            \\ % & $\SI{ 3576 (3677)}{a}$  & $\SI{5370.46910 (2.93022325)}{a}$    & $\SI{4.74729689 (2.93174255)}{a}$  & $\SI{7.59600458 (0.351461227)}{a}$ \\
                $\SI{1408.013}{\nothing}$        & $\SI{20.85}{\nothing}$& $\SI{ 586 (24)}{\nothing}$ & $\SI{1.231 (23)}{\nothing}$            \\ % & $\SI{ 5304 (5433)}{a}$  & $\SI{6798.07269 (4)}{a}$    & $\SI{6.16476521 (3.79392683)}{a}$  & $\SI{7.59078829 (0.351623258)}{a}$ \\
    \toprule
  \end{tabular}
\end{table}
%                                &  \scriptsize{$\sfrac{mN}{sm}{a}$}            &  \scriptsize{$\sfrac{mN}{m}{a}$}          &  \scriptsize{$\sfrac{mN}{m}{a}$} &  \scriptsize{$\sfrac{mN}{m}{a}$} & \scriptsize{$\sfrac{mN}{m}{a}$} \\ \hline \hline
%    \rowcolor{Lightgray} \multicolumn{7}{|c|}{Referenzen}\\ \hline
%        \multirow{2}{*}{\ce{C3F8}}  & 1  & $\SI{ 1.3396 \pm 0.0202 e-3}{\nothing}{a}$  &  $\SI{ -0.5415 \pm 0.0558 }{\nothing} $  &  $\SI{ 0.00 }{\nothing}{a}$      &  $\SI{ 5.72 }{\nothing}{a}$   & $\SI{ 5.72 }{\nothing}{a}$    \\

Nun wird $Q$ gegen die Energie $E$ des jeweiligen Peaks aufgetragen.
Es wird eine Ausgleichsrechnung der Form $Q=a E^{b} + c$ durchgeführt.
\begin{figure}[h!]
  \centering
  \includegraphics[width=0.9\textwidth]{content/images/vollenergienachweiswahrscheinlichkeit.pdf}
  \caption{Ausgleichsrechnung zur Bestimmung der Vollenergienachweiswahrscheinlichkeit $Q$.
  Die Fehlerbereiche verschwinden hinter den Datenpunkten und sind zur Übersichtlichkeit nicht aufgeführt.}
  \label{fig:vw}
\end{figure}
Die Parameter der Ausgleichsrechnung betragen:
\begin{align*}
	a = \SI{0.113 (55)}{\frac{1}{\kilo \electronvolt}}, && b = \SI{-0.36 (17)}{\nothing}, && c = \SI{-0.0077 (59)}{\nothing}.
\end{align*}

%mit $t = \SI{605484000 (54000)}{s}$
%und $T_{\sfrac{1}{2}} = \SI{426.7 (5) e+06}{s}$ \cite{nucleide}
%\begin{equation}
%	A = A_{0} \exp{\left( - \frac{\ln{(2))}}{T_{\sfrac{1}{2}}} t \right)} = \SI{1545(29)}{\per \second}
%	\label{result:aktivität}
%\end{equation}
%mit $r=\SI{22.5e-03}{m}$
%und $h = \SI{80e-03}{m}$
%\begin{align*}
%	\frac{r}{h} = \tan{( \varphi / 2 )} \Leftrightarrow \varphi = 2 \arctan{(\frac{r}{h})} \\
%	\frac{\Omega}{4 \pi} = \sin^2{\frac{\varphi}{2 \cdot 4}} =  \sin^2{ \left( \frac{1}{4} \arctan{(r/h)} \right)} = \SI{0.0069}{sr}
%	\label{result:raumwinkel}
%\end{align*}
%und
%\begin{equation}
%	Q = \frac{4 \pi}{\Omega} \frac{N_{\text{Peakinhalt}}}{ A T P_{E_{\gamma}}}
%\end{equation}

%\begin{figure}[h!]
%  \centering
%  \includegraphics[width=0.9\textwidth]{content/images/einzelnergaussfit_0.pdf}
%  \caption{Vergrößerung des ersten Peaks mit Ausgleichsfunktion einer Gaußkurve zur Veranschaulichung der Gaußpeaks.}
%  \label{fig:vw}
%\end{figure}

%\begin{table}[h!]
  \centering
  \caption{Parameter zur Berechnung der Vollenergienachweiswahrscheinlichkeit anhand eines $^{152}Eu$-Spektrums. Weitere verwendete Größen sind: \\ $A=\SI{1545(29)}{\per \second}$, $\frac{\Omega}{4 \pi} = \SI{0.0069}{sr}$, $T=\SI{2134}{s}$.}
  \label{tab:}
  \begin{tabular}{ r  r  r  r  r  r  }
    \bottomrule
              $E_{\gamma}$ / keV \cite{nucleide} & $P$ \cite{nucleide}   & $P_{\text{Peakinhalt}}$    & Q in $10^{-3}$            \\ % & $a$                              &      $\mu$                         & $\sigma$                         & $b$                              \\
    \midrule
                $\SI{121.7817}{\nothing}$        & $\SI{28.41}{\nothing}$& $\SI{8233 (91)}{\nothing}$ & $\SI{12.70 (24)}{\nothing}$            \\ % & $\SI{54643  (699)}{a}$  & $\SI{593.88 (0.02}}{a}$      & $\SI{2.72080278 (0.02096791)}{a}$  & $\SI{6.65464370 (0.155416486)}{a}$ \\
                $\SI{244.6974}{\nothing}$        & $\SI{ 7.55}{\nothing}$& $\SI{1515 (39)}{\nothing}$ & $\SI{ 8.79 (16)}{\nothing}$            \\ % & $\SI{10583 (1892)}{a}$  & $\SI{1186.58105 (0.33278)}{a}$       & $\SI{3.08985566 (0.33289872)}{a}$  & $\SI{7.46589080 (0.347744578)}{a}$ \\
                $\SI{344.2785}{\nothing}$        & $\SI{26.59}{\nothing}$& $\SI{3152 (56)}{\nothing}$ & $\SI{ 5.19 (10)}{\nothing}$            \\ % & $\SI{25365 (2038)}{a}$  & $\SI{1666.69026 (0.16131)}{a}$       & $\SI{3.33202744 (0.16137103)}{a}$  & $\SI{7.26196957 (0.334424003)}{a}$ \\
                $\SI{411.1165}{\nothing}$        & $\SI{2.238}{\nothing}$& $\SI{ 324 (18)}{\nothing}$ & $\SI{ 6.34 (12)}{\nothing}$            \\ % & $\SI{ 1320 (1390)}{a}$  & $\SI{1989.13968353 (1.58221965)}{a}$ & $\SI{2.49555989 (1.58264810)}{a}$  & $\SI{7.60693051 (0.351207638)}{a}$ \\
                $\SI{ 443.965}{\nothing}$        & $\SI{ 2.80}{\nothing}$& $\SI{ 367 (19)}{\nothing}$ & $\SI{ 5.74 (11)}{\nothing}$            \\ % & $\SI{ 1839 (1909)}{a}$  & $\SI{2147.41246865 (1.92603485)}{a}$ & $\SI{3.08211825 (1.92667938)}{a}$  & $\SI{7.60363599 (0.351271659)}{a}$ \\
                $\SI{778.9045}{\nothing}$        & $\SI{12.97}{\nothing}$& $\SI{ 741 (27)}{\nothing}$ & $\SI{ 2.50  (5)}{\nothing}$            \\ % & $\SI{ 6632 (3679)}{a}$  & $\SI{3763.60345 (1.60581703)}{a}$    & $\SI{4.79839449 (1.60665042)}{a}$  & $\SI{7.56537596 (0.351196062)}{a}$ \\
                $\SI{ 964.079}{\nothing}$        & $\SI{14.50}{\nothing}$& $\SI{ 596 (24)}{\nothing}$ & $\SI{1.801 (33)}{\nothing}$            \\ % & $\SI{ 5190 (4028)}{a}$  & $\SI{4656.71298 (2.39004332)}{a}$    & $\SI{5.10185800 (2.39136059)}{a}$  & $\SI{7.58314830 (0.351416184)}{a}$ \\
                $\SI{1085.837}{\nothing}$        & $\SI{10.13}{\nothing}$& $\SI{ 403 (20)}{\nothing}$ & $\SI{1.743 (32)}{\nothing}$            \\ % & $\SI{ 2217 (3342)}{a}$  & $\SI{5244.21808 (4.04291370)}{a}$    & $\SI{4.46961617 (4.04488016)}{a}$  & $\SI{7.60853282 (0.351488937)}{a}$ \\
                $\SI{1112.076}{\nothing}$        & $\SI{13.41}{\nothing}$& $\SI{ 502 (22)}{\nothing}$ & $\SI{1.640 (30)}{\nothing}$            \\ % & $\SI{ 3576 (3677)}{a}$  & $\SI{5370.46910 (2.93022325)}{a}$    & $\SI{4.74729689 (2.93174255)}{a}$  & $\SI{7.59600458 (0.351461227)}{a}$ \\
                $\SI{1408.013}{\nothing}$        & $\SI{20.85}{\nothing}$& $\SI{ 586 (24)}{\nothing}$ & $\SI{1.231 (23)}{\nothing}$            \\ % & $\SI{ 5304 (5433)}{a}$  & $\SI{6798.07269 (4)}{a}$    & $\SI{6.16476521 (3.79392683)}{a}$  & $\SI{7.59078829 (0.351623258)}{a}$ \\
    \toprule
  \end{tabular}
\end{table}
%                                &  \scriptsize{$\sfrac{mN}{sm}{a}$}            &  \scriptsize{$\sfrac{mN}{m}{a}$}          &  \scriptsize{$\sfrac{mN}{m}{a}$} &  \scriptsize{$\sfrac{mN}{m}{a}$} & \scriptsize{$\sfrac{mN}{m}{a}$} \\ \hline \hline
%    \rowcolor{Lightgray} \multicolumn{7}{|c|}{Referenzen}\\ \hline
%        \multirow{2}{*}{\ce{C3F8}}  & 1  & $\SI{ 1.3396 \pm 0.0202 e-3}{\nothing}{a}$  &  $\SI{ -0.5415 \pm 0.0558 }{\nothing} $  &  $\SI{ 0.00 }{\nothing}{a}$      &  $\SI{ 5.72 }{\nothing}{a}$   & $\SI{ 5.72 }{\nothing}{a}$    \\


%\begin{itemize}
%	\item Unterteilung der Daten in die Abschnitte um die Peaks (channel und counts werden passend abgeschnitten)
%	\item Über die Peaks wird eine Gaußkurve gelegt $f (x) = \frac{a}{\sqrt{2 \pi \sigma^2}} \exp{\left( - \frac{(x-\mu)^2}{2 \sigma^2} \right)}+b$
%	\item Klappt bei den meisten Peaks ganz gut, aber manchmal ist der Fehler des Parameters $a$ größer als a :|
%	\item Alternative für Inhalt: Aufsummation der Counts in den jeweiligen Intervallen, Fehler über den Poisson-Fehler ($N \pm \sqrt{N}$)
%	\item $Q$ berechnet mit $A$ aus \eqref{result:aktivität}, $\sfrac{\Omega}{4 \pi}$ aus \eqref{result:raumwinkel},
%	gesamte Messzeit $T=\SI{2134}{s}$, Emissionswahrscheinlichkeit $P_{E_{\gamma}}$ aus \cite{nucleide}, Peakinhalt $N_{\text{Peakinhalt}}$
%	\item $Q$ gegen $E$ aufgetragen, $Q=a E^{b} + c$, Parameter:
%\end{itemize}
%\begin{align*}
%	a = \SI{0.113 (55)}{\frac{1}{J}}, && b = \SI{-0.36 (17)}{\nothing}, && c = \SI{-0.0077 (59)}{\nothing}
%\end{align*}

%\begin{figure}[h!]
%  \centering
%  \includegraphics[width=0.9\textwidth]{content/images/vollenergienachweiswahrscheinlichkeit.pdf}
%  \caption{Ausgleichsrechnung zur Bestimmung der Vollenergienachweiswahrscheinlichkeit $Q$.
%  Die Fehlerbereiche verschwinden hinter den Datenpunkten und sind zur Übersichtlichkeit nicht aufgeführt.}
%  \label{fig:vw}
%\end{figure}

\FloatBarrier
\subsection{Monochromatisches $^{137}\symup{Cs}$-Spektrum}
In Abbildung \ref{fig:cs_spectrum} ist das volle Spektrum des $^{137}\symup{Cs}$-Strahlers abgebildet.
\begin{figure}[h!]
  \centering
  \includegraphics[width=0.9\textwidth]{content/images/caesium_vollesspektrum.pdf}
  \caption{Volles aufgenommenes Spektrum des $^{137}\symup{Cs}$-Strahlers.}
  \label{fig:cs_spectrum}
\end{figure}
Der Photopeak wird über eine Peak-Picking-Funktion ermittelt.
Dieser ist vergrößert in Abbildung \ref{fig:cs_photopeak} abgebildet.
\begin{figure}[h!]
  \centering
  \includegraphics[width=0.9\textwidth]{content/images/caesium_peak_1.pdf}
  \caption{Vergrößerter Photopeak des $^{137}\symup{Cs}$-Strahlers.}
  \label{fig:cs_photopeak}
\end{figure}
An den Peak wird eine Gaußverteilung nach
\begin{equation*}
	f(E) = \frac{a}{\sqrt{2 \pi \sigma^2}} \, \exp{\left( - \frac{(E-\mu)^2}{2 \sigma^2} \right)} + b
\end{equation*}
gefittet.
Hierzu wird der augewertete Datenbereich angepasst.
Die Parameter der Ausgleichsrechnung ergeben sich zu
\begin{align*}
	\mu = & \SI{661.2327 (51)}{\kilo \electronvolt}, && \sigma = \SI{0.9023 (51)}{\kilo \electronvolt}, \\
	 a = & \SI{1868 (9)}{\kilo \electronvolt^2}, 	 && b = \SI{6.1 (1)}{\kilo \electronvolt}. \\
\end{align*}
Dabei entspricht der Mittelwert $\mu$ der Energie der Photolinie:
\begin{equation*}
	\Rightarrow \mu = E_{\text{Photo, Data}} = \SI{661.2327 (51)}{\kilo \electronvolt}.
\end{equation*}
Ein Literaturwert \cite{nucleide} zum Photopeak findet sich zu
\begin{equation*}
	E_{\text{Photo, Theo}} = \SI{661.657 (3)}{\kilo \electronvolt}.
\end{equation*}
Für den Inhalt des Photopeaks werden die Counts im geplotteten Bereich aufsummiert.
Der Inhalt beträgt:
\begin{equation*}
	N_{\text{Photo}} = \SI{9174 (96)}{\nothing}.
\end{equation*}
Die Halbwertsbreite (FWHM) und die Zehntelwertsbreite (FWTM) werden zu folgenden Daten ausgemessen, indem die Energie bei der Hälfte bzw einem Zehntel der Counts aus den Messdaten bestimmt wird:
\begin{align*}
	\text{FWHM}_{\text{Daten}} & = \SI{2.07}{\kilo \electronvolt}\\
	\text{FWTM}_{\text{Daten}} & = \SI{3.94}{\kilo \electronvolt}\\
	\frac{ \text{FWHM}_{\text{Daten}} }{ \text{FWTM}_{\text{Daten}} } & = \SI{0.53}{\nothing}. \\
\end{align*}
Aus der Standardabweichung $\sigma$ lässt sich ein Vergleichswert passend zur gefitteten Gaußverteilung finden:
\begin{align*}
	\text{FWHM}_{\text{Fit}} && =  2 \sigma \, \sqrt{2 \ln{(2)}}   && = \SI{2.13}{\kilo \electronvolt}\\
	\text{FWTM}_{\text{Fit}} && =  2 \sigma \, \sqrt{2 \ln{(10)}}  && = \SI{3.87}{\kilo \electronvolt}\\
	\frac{ \text{FWHM}_{\text{Fit}} }{ \text{FWTM}_{\text{Fit}} } && = \SI{0.55}{\nothing}. \\
\end{align*}
\FloatBarrier

In Abbildung \ref{fig:cs_kontinuum} ist das Compton-Kontinuum des Spektrums vergrößert dargestellt.
\begin{figure}[h!]
  \centering
  \includegraphics[width=0.9\textwidth]{content/images/caesium_kontinuum.pdf}
  \caption{Vergrößertes Compton-Kontinuum des $^{137}\symup{Cs}$-Strahlers mit linearen Ausgleichsrechnungen zur Identifitkation der Lage der Compton-Kante.}
  \label{fig:cs_kontinuum}
\end{figure}
Über den Schnittpunkt zweier Ausgleichsgeraden der Form $y = a \cdot E + b$ wird die Lage der Compton-Kante angenähert.
Die Parameter der Geraden ergeben sich zu
\begin{align*}
	\text{links: }  && a = \SI{0.0699 (58)}{\frac{1}{\kilo \electronvolt}}, && b = \SI{-11.3 (24)}{\nothing}, \\
	\text{rechts: } && a = \SI{-0.152 (17)}{\frac{1}{\kilo \electronvolt}}, && b = \SI{ 89 (8)}{\nothing}. \\
\end{align*}
Der Schnittpunkt, entsprechend die Compton-Kante, liegt über Gleichsetzen der Geradengleichungen bei
\begin{equation*}
	E_{\text{Compton, Data}} = \SI{450 (5)}{\kilo \electronvolt}.
\end{equation*}
Aus Gleichung \eqref{eq:compton_kante} folgt für die Compton-Kante folgender theoretischer Wert:
\begin{equation*}
	E_{\text{Compton, Theo}} = \SI{477.3340 (28)}{\kilo \electronvolt}.
\end{equation*}
Der Inhalt des Compton-Kontinuums als Summation der betreffenden Kanalinhalte bis zur Compton-Kante beträgt
\begin{equation*}
	N_{\text{Kontinuum}} = \SI{40797 (202)}{\nothing}. % ohne Wirkungsquerschnitt verrechnet
\end{equation*}
\FloatBarrier

Der Rückstreupeak wird erneut durch das Anpassen zweier Geraden an beide Flanken des Peaks ermittelt (Abb. \ref{fig:cs_rueck}).
\begin{figure}[h!]
  \centering
  \includegraphics[width=0.9\textwidth]{content/images/caesium_rueckstreupeak.pdf}
  \caption{Vergrößerter Bereich des Compton-Kontinuums um den Rückstreupeak des $^{137}\symup{Cs}$-Strahlers mit linearen Ausgleichsrechnungen zur Identifitkation der Lage des Rückstreupeaks.}
  \label{fig:cs_rueck}
\end{figure}
Die Parameter beider Geradengleichungen lauten:
\begin{align*}
	\text{links: }  && a = \SI{0.532 (51) }{\frac{1}{\kilo \electronvolt}},  && b = \SI{-69 (9) }{\nothing}, \\
	\text{rechts: } && a = \SI{-0.580 (52) }{\frac{1}{\kilo \electronvolt}}, && b = \SI{144 (11) }{\nothing}. \\
\end{align*}
Der Rückstreupeak entspricht dem Schnittpunkt beider Geraden und liegt bei
\begin{equation*}
	E_{\text{Rück, Data}} = \SI{191 (18)}{\kilo \electronvolt}.
\end{equation*}
Der entsprechende Vergleichswert errechnet sich aus Gleichung \eqref{eq:rueckstreu} mit $\vartheta = \SI{90}{°}$ zu
\begin{equation*}
	E_{\text{Rück, Theo}} = \SI{242.1(11)}{\kilo \electronvolt}.
\end{equation*}
%Rückstreuung bei $\vartheta = \SI{90}{°}$:
%\begin{equation}
%    {E'}_{\!\gamma} = \frac{E_\gamma}{1 + \frac{E_\gamma}{m_e c^2} \left(1 - \cos\vartheta \right)} = \SI{242.1(11)}{\kilo \electronvolt}
%    \label{eq:rückstreu}
%\end{equation}
Der Extinktionskoeffizient, oder auch Absorptionskoeffizient $\mu$, lässt sich aus Abbildung \ref{fig:absorptionskoeffizient} ablesen.
\begin{figure}[h!]
  \centering
  \includegraphics[width=0.6\textwidth]{content/images/V704.pdf}
  \caption{Verlauf von $\mu$ gegen die $\gamma$-Energie aufgetragen \cite{anleitungv704}.}
  \label{fig:absorptionskoeffizient}
\end{figure}
Für die jeweiligen Wechselwirkungen und die zugehörigen Energien werden folgende Absorptionskoeffizienten $\mu$ abgelesen:
\begin{align*}
	\text{Photo:}	&& E_{\text{Photo, Data}} = \SI{661.2327 (51)}{\kilo \electronvolt} && \mu_{\text{Photo}} = \SI{0.00435 (10)}{cm^{-1}} \\
	\text{Compton:}	&& E_{\text{Compton, Data}} = \SI{450 (5)}{\kilo \electronvolt} 	&& \mu_{\text{Compton}} = \SI{0.40 (1)}{cm^{-1}}. \\
\end{align*}
Über die Absorberdicke (maximale Detektordicke) $d=\SI{39}{mm}$ und Gleichung \eqref{eqn:WW_wahrscheinlichkeit} %  P\ua{i}=\left(1- \exp(-\mu_i d)\right) \cdot 100
ergeben sich die Wechselwirkungswahrscheinlichkeiten
\begin{align*}
	\text{Photo: }		& P= \SI{1.68 (4)}{\%} \\
	\text{Compton: }	& P= \SI{79.0 (8)}{\%}. \\
\end{align*}
Das Verhältnis der Wechselwirkungswahrscheinlichkeiten berechnet sich zu:
\begin{equation*}
	\frac{ P_{\text{Photo} } }{ P_{ \text{Compton} } } = \SI{47.0 (12)}{\nothing}.
\end{equation*}
Das Verhältnis der Inhalte des Photopeaks und des Compton-Kontinuums gibt ebenfalls Auskunft über das Verhältnis der Wechselwirkungswahrscheinlichkeiten:
\begin{equation*}
	\frac{ N_{\text{Photo}} }{ N_{\text{Kontinuum}} } = \SI{4.45 (5)}{\nothing}.
\end{equation*}

%Wirkungsquerschnitt ($\varepsilon = E_\gamma / m c^2 << 1$, $\sigma_{Th} = \frac{1}{6 \pi \epsilon_0^2} \frac{e^4}{m_{e}^2 c^4}$):
%\begin{equation}
%    \sigma_{Co} \sim \frac{3}{4}\sigma_{Th} \left(1 - 2\varepsilon + \frac{26}{5}\varepsilon^2 \right).
%\end{equation}




%
%Suche
%\begin{itemize}
%	\item Energie des Strahlers (Photo-Energie -> Gauss fitten? mu entspricht dann dem Channel und der Channel einer Energie)
%	\item Fit:
%	\begin{equation*}
%		f(x) = \frac{a}{\sqrt{2 \pi \sigma^2}} \, \exp{\left( - \frac{(x-\mu)^2}{2 \sigma^2} \right)} + b
%	\end{equation*}
%	\item Parameter
%	\begin{align*}
%		\mu = \SI{661.2327 (51)}{\kilo \electronvolt} && \sigma = \SI{0.9023 (51)}{\kilo \electronvolt} && a = \SI{1868 (9)}{\nothing} && b = \SI{6.1 (1)}{\nothing}
%	\end{align*}
%	\item Halbwertsbreite fitten
%	\item Zehntelwertsbreite fitten, $FWHM/FWTM(Theorie) = 0.5486620049392714$, in den Daten: $FWHM/FWTM = 0.5486620049392715$
%	\item
%	\begin{align*}
%		FWHM = 10 && FWTM = 19
%	\end{align*}
%	\item Comptonkante ausmessen, bei:
%	\begin{align*}
%		E = \SI{4.6 (4)e+02}{\kilo \electronvolt}
%	\end{align*}
%	\item
%	mit den Parametern
%	\begin{align*}
%		links   && rechts \\
%		a = \SI{0.0699 (58)}{\frac{1}{\kilo \electronvolt}} && a = \SI{ -0.337 (27)}{\frac{1}{\kilo \electronvolt}} \\
%		b = \SI{ -11.3 (24)}{\nothing} 						&& b = \SI{ 178 (13)}{\nothing} \\
%	\end{align*}
%	\item Theorie liegt bei


\subsection{Aktivität von Barium}
\begin{figure}[h!]
  \centering
  \includegraphics[width=0.9\textwidth]{content/images/barium_markiert.pdf}
  \caption{Spektrum des $^{133}\symup{Ba}$ mit zugeordneten Literaturwerten \cite{nucleide}.}
  \label{fig:ba_spectrum}
\end{figure}
In Abbildung \ref{fig:ba_spectrum} ist das Spektrum zum $^{133}\symup{Ba}$ dargestellt.
Die Peaks werden mithilfe einer Peak-Picking-Funktion ermittelt.
Die Literaturwerte zum Barium \cite{nucleide} sind in Tabelle \ref{tab:barium} aufgeführt.
Die Werte sind ebenfalls in Abbildung \ref{fig:ba_spectrum} abgebildet.

Nun wird die Aktivität der Probe zur Zeit der Messung bestimmt.
Hierzu werden die einzelnen Peaks genauer betrachtet, passende Bereiche der Gaußpeaks gewählt und die Inhalte der einzelnen Peaks werden berechnet.
Die Energien der einzelnen Peaks $E_{\gamma}$, die Emissionswahrscheinlichkeit $P$, die Peakinhalte $N_{\text{Peakinhalt}}$ sind in Tabelle \ref{tab:barium} notiert.
Über die Ausgleichsrechnung aus Kapitel \ref{kap:vw} werden aus den Energien die Vollenergienachweiswahrscheinlichkeiten $Q$ berechnet.
Die Gleichung \eqref{eqn:efficency} wird nach der Aktivität $A$ umgestellt und so für jeden Peak eine Aktivität bestimmt.
%\begin{equation*}
%	Q = \frac{4 \pi}{\Omega} \frac{N_{\text{Peakinhalt}}}{ A T P_{E_{\gamma}}}.
%	\Leftrightarrow A = \frac{4 \pi}{\Omega} \frac{N_{\text{Peakinhalt}}}{ Q T P_{E_{\gamma}}}.
%\end{equation*}
%Die umgerechneten Q aus den Energien (nach Kap. \ref{kap:vw}) ergeben sich zu den Werten in Tabelle \ref{tab:barium}.
\begin{table}[h!]
  \centering
  \caption{Parameter zur Berechnung der Aktivität anhand eines $^{133}\symup{Ba}$-Spektrums. Weitere verwendete Größen sind: \\ $\frac{\Omega}{4 \pi} = \SI{0.0069}{sr}$, $T=\SI{2347}{s}$.}
  \label{tab:barium}
  \begin{tabular}{ r  r  r  r  r  r  r }
    \bottomrule
              $E_{\gamma}$ / keV \cite{nucleide} & $P$ \cite{nucleide}   & $P_{\text{Peakinhalt}}$    & Q in $10^{-3}$              & A in $Bq$\\ % & $a$                              &      $\mu$                         & $\sigma$                         & $b$                              \\
    \midrule
                $\SI{81.0579}{\nothing}$         & $\SI{33.31}{\nothing}$& $\SI{7173 (85)}{\nothing}$ & $\SI{15.52 }{\nothing}$     & $\SI{854 (10)}{\nothing}$      \\ % & $\SI{54643  (699)}{a}$  & $\SI{593.88 (0.02}}{a}$      & $\SI{2.72080278 (0.02096791)}{a}$  & $\SI{6.65464370 (0.155416486)}{a}$ \\
                $\SI{276.2951}{\nothing}$        & $\SI{ 7.13}{\nothing}$& $\SI{1099 (33)}{\nothing}$ & $\SI{7.23 }{\nothing}$      & $\SI{1311 (40)}{\nothing}$     \\ % & $\SI{10583 (1892)}{a}$  & $\SI{1186.58105 (0.33278)}{a}$       & $\SI{3.08985566 (0.33289872)}{a}$  & $\SI{7.46589080 (0.347744578)}{a}$ \\
                $\SI{302.6169}{\nothing}$        & $\SI{18.31}{\nothing}$& $\SI{2290 (48)}{\nothing}$ & $\SI{6.75 }{\nothing}$      & $\SI{1140 (24)}{\nothing}$     \\ % & $\SI{25365 (2038)}{a}$  & $\SI{1666.69026 (0.16131)}{a}$       & $\SI{3.33202744 (0.16137103)}{a}$  & $\SI{7.26196957 (0.334424003)}{a}$ \\
                $\SI{356.0896}{\nothing}$        & $\SI{62.05}{\nothing}$& $\SI{6207 (79)}{\nothing}$ & $\SI{5.93 }{\nothing}$      & $\SI{1038 (13)}{\nothing}$     \\ % & $\SI{ 1320 (1390)}{a}$  & $\SI{1989.13968353 (1.58221965)}{a}$ & $\SI{2.49555989 (1.58264810)}{a}$  & $\SI{7.60693051 (0.351207638)}{a}$ \\
                $\SI{383.6549}{\nothing}$        & $\SI{ 8.94}{\nothing}$& $\SI{ 845 (29)}{\nothing}$ & $\SI{5.57 }{\nothing}$      & $\SI{1044 (36)}{\nothing}$     \\ % & $\SI{ 1839 (1909)}{a}$  & $\SI{2147.41246865 (1.92603485)}{a}$ & $\SI{3.08211825 (1.92667938)}{a}$  & $\SI{7.60363599 (0.351271659)}{a}$ \\
    \toprule
  \end{tabular}
\end{table}
%                                &  \scriptsize{$\sfrac{mN}{sm}{a}$}            &  \scriptsize{$\sfrac{mN}{m}{a}$}          &  \scriptsize{$\sfrac{mN}{m}{a}$} &  \scriptsize{$\sfrac{mN}{m}{a}$} & \scriptsize{$\sfrac{mN}{m}{a}$} \\ \hline \hline
%    \rowcolor{Lightgray} \multicolumn{7}{|c|}{Referenzen}\\ \hline
%        \multirow{2}{*}{\ce{C3F8}}  & 1  & $\SI{ 1.3396 \pm 0.0202 e-3}{\nothing}{a}$  &  $\SI{ -0.5415 \pm 0.0558 }{\nothing} $  &  $\SI{ 0.00 }{\nothing}{a}$      &  $\SI{ 5.72 }{\nothing}{a}$   & $\SI{ 5.72 }{\nothing}{a}$    \\

Anschließend wird eine finale Aktivität bestimmt, indem die einzelnen Aktivitäten gemittelt werden:
\begin{equation*}
    A = \SI{1077 (12)}{\frac{1}{s}}.
\end{equation*}

\subsection{Bestimmung der Zusammensetzung einer Probe aus Bananenchips}
