\section{Diskussion}
\paragraph{Kalibration und Vollenergienachweiswahrscheinlichkeit durch Messungen an Europium}
Die Energiekalibration des Germanium-Detektors mithilfe einer $^{152}\symup{Eu}$-Probe verläuft wie erwartet.
Die Unsicherheiten der Kalibrierung sind klein (max. $\SI{0.02}{\%}$).\\
Die Aktivität der Europiumprobe wird zu
\begin{equation*}
	A  = \SI{1545(29)}{\per \second}
\end{equation*}
bestimmt.
Die Wahrscheinlichkeit, dass ein Teilchen seine volle Energie im Detektor verliert, ist die Vollenergienachweiswahrscheinlichkeit $Q$ (engl.: \textit{efficiency}).
Die Ausgleichsrechnung dazu hat eine maximale Unsicherheit in den Parametern von $\SI{76.59}{\%}$.
Diese Abweichungen erklären sich möglicherweise durch einen ungenau apporoximierten Raumwinkel.\\

\paragraph{Untersuchung eines monochromatischen Cäsium-Spektrums}
Anschließend wird ein monochromatisches Cäsium-Spektrum ($^{137}\symup{Cs}$) untersucht.
In dem Spektrum wird der Peak, der durch den Photo-Effekt entsteht, genauer betrachtet.
Die Lage des Peaks wird aus einer Ausgleichsrechnung mit einer Gaußverteilung bestimmt und einmal wird ein Literaturwert nachgeschlagen \cite{nucleide}:
\begin{align*}
E_{\text{Photo, Data}} = \SI{661.2327 (51)}{\kilo \electronvolt} && E_{\text{Photo, Theo}} = \SI{661.657 (3)}{\kilo \electronvolt}.
\end{align*}
Dabei weichen die Daten zu $\SI{0.06}{\%}$ von der Literatur ab.
Die Ausgleichsrechnung der Gaußverteilung lässt sich mithilfe des Verhältnisses der Halbwertsbreite (FWHM) und der Zehntelwertsbreite (FWTM) beurteilen.
Das Verhältnis wird hierzu direkt aus den Daten, ohne Ausgleichsrechnung, und ein zweites Mal aus der Ausgleichsrechnung bestimmt:
\begin{align*}
	\frac{ \text{FWHM}_{\text{Daten}} }{ \text{FWTM}_{\text{Daten}} }  = \SI{0.53}{\nothing} && 	\frac{ \text{FWHM}_{\text{Fit}} }{ \text{FWTM}_{\text{Fit}} } = \SI{0.55}{\nothing}. \\
\end{align*}
Das Verhältnis der beiden Breiten aus den Daten weicht hier um $\SI{3.64}{\%}$ von der optimalen Gaußverteilung ab.
Dennoch lässt sich bestätigen, dass die 'Spektrallinien' in den Spektren schmalen Gaußverteilungen entsprechen.\\

Nach dem Photopeak wird nun das Compton-Kontinuum betrachtet.
Die Compton-Kante aus den Messdaten und der Vergleichswert ergeben sich zu
\begin{align*}
	E_{\text{Compton, Data}} = \SI{450 (5)}{\kilo \electronvolt} && E_{\text{Compton, Theo}} = \SI{477.3340 (28)}{\kilo \electronvolt}.
\end{align*}
Diese weichen um $\SI{5.73}{\%}$ voneinander ab.\\

Auch der Peak, der sich durch die Rückstreuung der Compton-Wechselwirkung ergibt, lässt sich sowohl aus den Daten bestimmen, als auch berechnen:
\begin{align*}
	E_{\text{Rück, Data}} = \SI{191 (18)}{\kilo \electronvolt} && E_{\text{Rück, Theo}} = \SI{242.1(11)}{\kilo \electronvolt}.
\end{align*}
Die relative Abweichung von Daten zur Theorie beträgt $\SI{21.11}{\%}$.\\

Weiterhin wird das Verhältnis der beiden Wechselwirkungswahrscheinlichkeiten (Photo/Compton) über zwei verschiedene Rechnungen gebildet.
Zuerst über die Absorptionskoeffizienten $\mu$ und die Wechselwirkungswahrscheinlichkeiten $P$, anschließend durch das Verhältnis der Inhalte von Photopeak und Compton-Kontinuum:
\begin{align*}
	\frac{ P_{\text{Photo} } }{ P_{ \text{Compton} } } = \SI{47.0 (12)}{\nothing} && 	\frac{ N_{\text{Photo}} }{ N_{\text{Kontinuum}} } = \SI{4.45 (5)}{\nothing}.
\end{align*}
Zwischen diesen Werten liegt der Faktor $\SI{10.56}{\nothing}$.
Dieser Unterschied ist unerwartet, es gibt aber auch Möglichkeiten diese Abweichung zu erklären.
Die Wechselwirkungswahrscheinlichkeiten $P$ beinhalten nur einfache Wechselwirkungen.
Dennoch sind Wechselwirkungen höherer Ordnungen denkbar und möglich.\\

\paragraph{Aktivität von Barium}

Die Aktivität von Barium ($^{133}\symup{Ba}$) berechnet sich zu
\begin{equation*}
    A = \SI{1077 (12)}{\frac{1}{s}}.
\end{equation*}

\paragraph{Bestimmung der Isotope in Bananenchips}
Im Spektrum der Probe aus Bananenchips können die Linien dem Strahler $^{40}\symup{K}$ zugeordnet werden.
Die Lokalisation der Linien im gemessenen Spektrum weicht maximal $\SI{0.05}{\%}$ von den Literaturwerten ab.
Weitere Linien können nicht genau zugeordnet werden.
Zwar können einzelne Energien mit Isotopen in Verbindung gebracht werden, aber diese Strahler würden weitere Linien im Spektrum produzieren, die in der Messung jedoch nicht vorhanden sind.
Andererseits gibt es Energien, die im Rahmen der Unsicherheit bei vielen verschiedenen Isotopen als Spektrallinie aufgeführt wird.
Dennoch ist kein weiteres Element außer $^{40}\symup{K}$ sinnvoll zuzuordnen.
Die Zerfallsprodukte von $^{40}\symup{K}$ sind keine $\gamma$-Strahler und sind nicht signifikant im Spektrum zu erkennen.\\

Insgesamt lässt sich sagen, dass der Versuch wenige Fehlerquellen bietet und gut verlaufen ist.
