\FloatBarrier
\newpage
\section{Diskussion}
In Abbilung \ref{fig:Kontrast} ist eine $|\sin{(2\phi)}|$ - Abhängigkeit deutlich zu erkennen.
Dabei liegen auch alle Messwerte nahe der Kurve.
Zusätzlich sind auch die erwarteten Maxima bei $\SI{45}{\degree}$ und $\SI{130}{\degree}$ deutlich zu erkennen. \\
Die Untersuchung des Brechungsindexes für Glas ergab einen Brechungsindex von
\begin{equation}
    n_\text{Glas} = \SI{1.549 \pm 0.011}{\nothing} .
\end{equation}
Dies entspricht einer Abweichung von $3,26 \%$ vom Literaturwert von $n_\text{Glas} = 1,5$ und liegt somit innerhalb der Messungenauigkeiten.  \\
Dagegen liegt der gemessene Brechungsindex von Luft bei
\begin{equation}
    \bar{n}_\text{Luft} = \SI{1.16215 \pm 0.00016}{\nothing} ,
 \end{equation}
was einer Abweichung von $16 \%$ enstpricht zum Literaturwert von $n_\text{Luft} = 1$. entspricht. Dies kann mehrere Ursachen haben. Zu einem ist diese Abweichung durch einen defekten oder unpassend eingestellten Zähler zu erklären, da bereits während der Messung in einigen Fällen die Anzahl der gemessenen Maxima bei konstanten Druck schlagartig stieg. Zum anderem sind eventuelle Verunreinigungen innerhalb der Röhre nicht auszuschließen.
