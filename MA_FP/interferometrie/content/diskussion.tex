\subsection{Diskussion}
Bei dem in Abbilung \ref{fig:Kontrast} ist deutlich eine $|\sin{(2\phi)}|$ - Abhängigkeit zu erkennen. Dabei liegen auch alle Messwerte nahe der Kurve. Zusätzlich sind auch die erwarteten Maxima bei $\SI{45}{\degree}$ und $\SI{13}{\degree}$ deutlich zu erkennen. \\
Die Untersuchung des Brechungsindexes für Glas ergab einen Brechungsindex von 
\begin{equation}
    n_\text{Glas} = 1,549 \pm 0,011 .
\end{equation}
Dies entspricht einer Abweichung von $3,26 \%$ vom Literaturwert von $n_\text{Glas} = 1,5$ und liegt somit innerhalb der Messungenauigkeiten.  \\
Dagen liegt der gemessene WEirkungsquerschnitt von Luft bei 
\begin{equation}
    \bar{n}_\text{Luft} = 1,16215 \pm 0,00016 ,
 \end{equation}
was einer Abweichung von $16 \%$ enstpricht zum Literaturwert von $n_\text{Luft} = 1$. entspricht. Dies kann mehrere ursachen haben. Zu einem ist diese Abweichung durch einen defekten Zähler zu erklären, da bereits während der Messung die Anzahl der gemessenen Maxima bei konstanten Druck schlagartig stieg. Zum anderem sind eventuelle Verunreinigungen innerhalb der Röhre nicht auszuschließen.    