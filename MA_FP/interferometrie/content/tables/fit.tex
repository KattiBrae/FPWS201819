\begin{table}
    \centering
    \caption{Ermittelte Regressionsparameter $A$ und $B$ der Messung zur Bestimmung der Abhängigkeit zwischen Brechungsindex $n$ und Druck p. Zudem ist jeweils der Wert $n_{\mathup{norm}}$ zum Vergleich mit der Literatur angegeben.}
    \label{tab:fit}
    \begin{tabular}{c c c}
    \toprule
    {Messung} & {$a \:/\: \si{ 10^{-4}\milli\bar}$} & {$b$} \\
    \midrule
    1 & 7,414899744 \pm 0,10993544 & 0.99999093 \pm 0,0000027 \\
    2 & 7,192185871 \pm 0,08469601 & 1.00000269 \pm 0,0000020 \\
    3 & 7,876603239 \pm 0,24499417 & 1.00000852 \pm 0,0000060 \\
    \bottomrule
    \end{tabular}
    \toprule
    {Messung} & {$a \:/\: \si{ 10^{-4}\milli\bar}$} & {$b$} \\
    \midrule
    1 & 7,414899744 \pm 0,10993544 & 0.99999093 \pm 0,0000027 \\
    2 & 7,192185871 \pm 0,08469601 & 1.00000269 \pm 0,0000020 \\
    3 & 7,876603239 \pm 0,24499417 & 1.00000852 \pm 0,0000060 \\
    \bottomrule
    \end{tabular}
    \toprule
    {Messung} & {$a \:/\: \si{ 10^{-4}\milli\bar}$} & {$b$} \\
    \midrule
    1 & 7,414899744 \pm 0,10993544 & 0.99999093 \pm 0,0000027 \\
    2 & 7,192185871 \pm 0,08469601 & 1.00000269 \pm 0,0000020 \\
    3 & 7,876603239 \pm 0,24499417 & 1.00000852 \pm 0,0000060 \\
    \bottomrule
    \end{tabular}
    \end{table}
