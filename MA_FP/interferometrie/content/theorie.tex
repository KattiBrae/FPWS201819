\section{Theoretische Grundlagen}
\subsection{Interferenz und Kohärenz}
Überlagern sich zwei Lichtwellen kann es zur \texttt{Interferenz} kommen.
Diese äußert sich in destruktiver Interferenz (Auslöschung der Welle, keine Intensität) und konstruktiver Interferenz (Addition der Wellenextrema, verstärkte Intensität).
Eine wichtige Bedingung hierfür ist, dass die Wellen die gleiche Wellenlänge haben.
Genauer gesagt, für eine länger andauernde, stabile Interferenz, müssen beide Wellen kohärent sein.
\texttt{Kohärenz} beschreibt den Aspekt, dass die Wellenlänge zweier überlagernder Wellen über die Dauer der Kohärenzzeit gleich ist.
Die Kohärenzzeit wiederum ist die Zeit, über die sich die Welle nicht ändert.
Die Kohärenzzeit endet, sobald die Lichtquelle einen Phasensprung aussendet oder, seltener, wenn sich die Eigenschaften der Lichtwelle (Wellenlänge, Phase, etc.) der Lichtquelle ändert.
Andersherum bedeutet \texttt{zeitliche Kohärenz}, dass bei eine ausgekoppelte Teilwelle nach einiger Zeit kohärent zur Ursprungswelle zurückgeführt werden kann.
\texttt{Räumliche Kohärenz} liegt vor, wenn zwei ausgekoppelte Teilwellen trotz räumlicher Verschiebung miteinander interferieren können.
Der \texttt{Kohärenzgrad} stellt die Interferenzfähigkeit zweier Wellen dar.

\subsection{Polarisation}
- was ist das?

- typen der Polarisation

- lineare Polarisation

- Polarisationsfilter / Reflexion / Dielektrika
