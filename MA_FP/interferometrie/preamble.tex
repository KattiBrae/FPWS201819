%%%%%%%%%%%%%%%%%%%%%%%%%%%%%%%%%%%%%%%%%%%%%%%%%%%%%%%%%%%%%%%%%%%%%%%%%%%%%%%%
%%%%%%%%%%%%%%%%%%   Vorlage für eine Abschlussarbeit   %%%%%%%%%%%%%%%%%%%%%%%%
%%%%%%%%%%%%%%%%%%%%%%%%%%%%%%%%%%%%%%%%%%%%%%%%%%%%%%%%%%%%%%%%%%%%%%%%%%%%%%%%

% Erstellt von Maximilian Nöthe, <maximilian.noethe@tu-dortmund.de>
% ausgelegt für lualatex und Biblatex mit biber

% Kompilieren mit
% latexmk --lualatex --output-directory=build thesis.tex
% oder einfach mit:
% make

%Das war meine BA
%\documentclass[
%%  tucolor,       % remove for less green,
%  BCOR=12mm,     % 12mm binding corrections, adjust to fit your binding
%  parskip=half,  % new paragraphs start with half line vertical space
%  open=any,      % chapters start on both odd and even pages
%  cleardoublepage=plain,  % no header/footer on blank pages
%]{tudothesis}

%Das ist ein einfacheres Dokument
\documentclass[
  bibliography=totoc,     % Literatur im Inhaltsverzeichnis
  captions=tableheading,  % Tabellenüberschriften
  titlepage=firstiscover, % Titelseite ist Deckblatt
]{scrartcl}


% Warning, if another latex run is needed
\usepackage[aux]{rerunfilecheck}

% just list chapters and sections in the toc, not subsections or smaller
\setcounter{tocdepth}{2}

%------------------------------------------------------------------------------
%------------------------------ Fonts, Unicode, Language ----------------------
%------------------------------------------------------------------------------
\usepackage{fontspec}
\defaultfontfeatures{Ligatures=TeX}  % -- becomes en-dash etc.

% german language
\usepackage{polyglossia}
\setdefaultlanguage{german}

% for english abstract and english titles in the toc
\setotherlanguages{english}

% intelligent quotation marks, language and nesting sensitive
\usepackage[autostyle]{csquotes}

% microtypographical features, makes the text look nicer on the small scale
\usepackage{microtype}

%------------------------------------------------------------------------------
%------------------------ Math Packages and settings --------------------------
%------------------------------------------------------------------------------

\usepackage{amsmath}
\usepackage{amssymb}
\usepackage{mathtools}

% Enable Unicode-Math and follow the ISO-Standards for typesetting math
\usepackage[
  math-style=ISO,
  bold-style=ISO,
  sans-style=italic,
  nabla=upright,
  partial=upright,
]{unicode-math}
\setmathfont{Latin Modern Math}

% nice, small fracs for the text with \sfrac{}{}
\usepackage{xfrac}


%------------------------------------------------------------------------------
%---------------------------- Numbers and Units -------------------------------
%------------------------------------------------------------------------------

\usepackage[
  locale=DE,
  separate-uncertainty=true,
  per-mode=symbol-or-fraction,
]{siunitx}
\sisetup{math-micro=\text{µ},text-micro=µ}
\DeclareSIUnit{\nothing}{\relax}

%------------------------------------------------------------------------------
%-------------------------------- tables  -------------------------------------
%------------------------------------------------------------------------------

\usepackage{booktabs}       % \toprule, \midrule, \bottomrule, etc
\usepackage{multirow}
\usepackage{tabularx}


%------------------------------------------------------------------------------
%-------------------------------- graphics -------------------------------------
%------------------------------------------------------------------------------

\usepackage{graphicx}
\usepackage{grffile}

% allow figures to be placed in the running text by default:
\usepackage{scrhack}
\usepackage{float}
\floatplacement{figure}{htbp}
\floatplacement{table}{htbp}

% keep figures and tables in the section
\usepackage[section, below]{placeins}


%------------------------------------------------------------------------------
%---------------------- customize list environments ---------------------------
%------------------------------------------------------------------------------

\usepackage{enumitem}

%------------------------------------------------------------------------------
%------------------------------ Bibliographie ---------------------------------
%------------------------------------------------------------------------------

\usepackage[
  backend=biber,   % use modern biber backend
  autolang=hyphen, % load hyphenation rules for if language of bibentry is not
                   % german, has to be loaded with \setotherlanguages
                   % in the references.bib use langid={en} for english sources
  sorting=none,
]{biblatex}
\addbibresource{lit.bib}  % the bib file to use
\DefineBibliographyStrings{german}{andothers = {{et\,al\adddot}}}  % replace u.a. with et al.

%------------------------------------------------------------------------------
%----------------------    das hab ich hinzugefügt   --------------------------
%------------------------------------------------------------------------------

\usepackage{geometry}
\usepackage{caption}
\usepackage{subcaption}
\usepackage[version=4]{mhchem}
\usepackage{tabularx}
\usepackage{multirow}
\usepackage{ulem}
\usepackage{color, colortbl}
\definecolor{Lightgray}{rgb}{0.8, 0.8, 0.8}
\definecolor{tu}{rgb}{0.196, 0.44, 0.00}

\addto\captionsngerman{%
  \renewcommand{\figurename}{Abb.}%
  \renewcommand{\tablename}{Tab.}%
}
\usepackage{lscape}
\newcommand*\cleartoleftpage{%
  \clearpage
  \ifodd\value{page}\hbox{}\newpage\fi
}

% Erweiterung zum Erstellen von Feynman-Diagrammen
\usepackage[compat=1.1.0]{tikz-feynman}

% Micro in \SI{}{} reparieren
\sisetup{math-micro=\text{µ},text-micro=µ}

\author{%
  Katharina Brägelmann\\%
  \href{mailto:katharina.braegelmann@tu-dortmund.de}{katharina.braegelmann@tu-dortmund.de}%
  \texorpdfstring{\and}{,}%
  Lars Kolk\\%
  \href{mailto:lars.kolk@tu-dortmund.de}{lars.kolk@tu-dortmund.de}%
}
\publishers{TU Dortmund – Fakultät Physik}

%------------------------------------------------------------------------------
%------------------------------ Bibliographie ---------------------------------
%------------------------------------------------------------------------------

% Last packages, do not change order or insert new packages after these ones
%\usepackage[pdfusetitle, unicode, linkbordercolor=tugreen]{hyperref}
\usepackage[pdfusetitle, unicode]{hyperref}
\usepackage{bookmark}
\usepackage[shortcuts]{extdash}


%------------------------------------------------------------------------------
%-------------------------    Angaben zur Arbeit   ----------------------------
%------------------------------------------------------------------------------

%\author{Katharina Brägelmann}
%\title{Gasadsorption an geladenen Lipidmonolagen}
%\date{2019}
%\birthplace{Maastricht}
%\chair{Lehrstuhl für Experimentelle Physik I}
%\division{Fakultät Physik}
%\thesisclass{Bachelor of Science}
%\submissiondate{01. Juli 2019}
%\firstcorrector{Prof.~Dr.~Metin Tolan}
%\secondcorrector{Prof.~Dr.~Heinz Hövel}
