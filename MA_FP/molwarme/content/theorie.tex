\subsection{Theoretische Grundlagen}
\subsection{Allgemeines zur Wärmekapazität}
Die spezifische Wärmekapazität, oder auch Molwärme, $c_{\text{m}}$ ist nach der Thermodynamik wie folgt über die Wärmekapazität $C_{\text{V}}$ definiert:
\begin{equation*}
	c_{\text{m}} = \frac{C_{\text{V}}}{m} = \frac{1}{m} \frac{\partial U}{\partial T} \bigg{\vert}_{V}.
\end{equation*}
\begin{center}
	\tiny{$m \, \widehat{=}$ Molmasse des Materials, $V \, \widehat{=}$ konstantes Volumen }
\end{center}
Im Festkörper sind die Atome in gitterförmigen Kristallstrukturen angeordnet.
Die Temperatur eines Materials bedeutet mikroskopisch, dass die Atome in der Gitterstruktur schwingen und so Energie speichern.
Diese Gitterschwingungen werden \textit{Phononen} genannt.
Ein Phonon beschreibt also eine wellenförmige Anregung des Gitters mit einer zugeordeten Energie (eine Mode).
%Energie, also auch Frequenz $\omega$, hängen von der Wellenzahl $k$ ab.
%Dies ist die sognenannte Dispersionsrelation $\omega(k)$.
Phononen verschiedener Energien (verschiedene Moden) können sich überlagern.
Die Energie eines Teilchens beträgt
\begin{equation*}
	U = \sum_{i=x,y,z} \left( \underbrace{ \,\, \frac{1}{2}mv_{\text{i}}^2 \,\, }_{\substack{kin. Energie}} + \underbrace{ \,\, \frac{1}{2} k i^2 \,\, }_{\substack{pot. Energie}} \right).
\end{equation*}
\begin{center}
	\tiny{$k_{\text{B}} \widehat{=}$ Boltzmann-Konstante }
\end{center}
Im einfachsten Modell wird die Gesamtenergie der Atome durch das Äquipartitionstheorem statistisch zu gleichen Teilen zu kinetischer und potenzieller Energie aufgeteilt mit je $E_{\text{kin}} = E_{\text{pot}} = 1/2 k_{\text{B}} T $
Die Gesamtenergie beträgt bei je drei Freiheitsgraden pro Teilchen und einer Teilchenanzahl $N$ entsprechend
\begin{equation*}
	U = 2 \cdot \frac{3}{2} N k_{\text{B}} T.
\end{equation*}
\begin{center}
	\tiny{$k_{\text{B}} \widehat{=}$ Boltzmann-Konstante, $T \widehat{=}$ Temperatur}
\end{center}
Somit liegt die Wärmekapazität $C_{\text{V}}$ bei
\begin{equation*}
	C_{\text{V}} = \frac{\partial U}{\partial T} \bigg{\vert}_{V} = 3 N k_{\text{B}} = 3 R.
\end{equation*}
\begin{center}
	\tiny{$R \widehat{=}$ Universelle Gaskonstante}
\end{center}
Dieser Zusammenhang wird \textit{Dulong-Petit-}Gesetz genannt.
Es beschreibt eine klassische Näherung der Schwingungen der Atome mit vielen verschiedenen Moden, also Phononen verschiedener Energien.
Bei höheren Temperaturen (je nach Material z.B. $T \gtrapprox \SI{200}{K}$) passt dieses Modell zu den experimentellen Daten, bei niedrigen Temperaturen (je nach Material z.B. $T \lessapprox \SI{200}{K}$) weichen Modell und Messdaten stark voneinander ab.

\FloatBarrier
\subsection{Einstein-Modell}
Im Einstein-Modell wird angenommen, dass alle Phononen eine Energie (eine Mode) haben, also alle mit einer Frequenz $\omega$ schwingen.
Alle Teilchen liegen also im gleichen Zustand und die Zustandsdichte $D(\omega)$ ist entsprechend eine $\delta$-Funktion.
Die Gesamtenergie der Teilchen im Gitter lässt sich mit einer Planck-Verteilung der Besetzungszahl der Phononen berechnen:
\begin{equation*}
	\langle U \rangle = \frac{1}{\exp{\left( \frac{ \hbar \omega }{ k_{\text{B}} T } \right)} - 1}.
\end{equation*}
\begin{center}
	\tiny{$\hbar \widehat{=}$ reduziertes Planck'sches Wirkungsquantum}
\end{center}

Die Wärmekapazität entspricht so
\begin{equation*}
	C_{\text{V}} = \frac{\partial U}{\partial T} \bigg{\vert}_{V} = 3 R \left( \frac{\hbar \omega}{k_{\text{B}} T} \right)^2
	\frac{ \exp{\left( \frac{ \hbar \omega }{ k_{\text{B}} T } \right)} }{ \left[ \exp{\left( \frac{ \hbar \omega }{ k_{\text{B}} T } \right)} - 1 \right]^2 } =
	3 R \left( \frac{\hbar \omega}{k_{\text{B}} T} \right)^2
	\frac{ \exp{\left( \frac{ \Theta_{\text{E}} }{ T } \right)} }{ \left[ \exp{\left( \frac{ \Theta_{\text{E}} }{ T } \right)} - 1 \right]^2 }
\end{equation*}
mit der sogenannten Einstein-Temperatur
\begin{equation*}
	\Theta_{\text{E}} = \frac{\hbar \omega}{k_{\text{B}}}.
\end{equation*}
Die Einstein-Temperatur ist somit durch die Schwingungsfrequenz $\omega$ eine materialabhängige Konstante.
Durch die Annahme, dass nur eine Mode vorhanden ist, entspricht das Modell bei sehr tiefen Temperaturen und geringen Frequenzen nicht der Realität.
Das Einstein-Modell passt jedoch sehr gut zu Anregungen mit optischen Phononen.

\subsection{Debye-Modell}
Das Debye-Modell geht von einer linearen Zustandsdichte $D(\omega) \propto \omega$ aus.
Die Phononen haben also eine kontinuierliche Energieverteilung.
Die Gesamtenergie lässt sich über Integration der Zustandsdichte $D(\omega)$ und der gemittelten Besetzungszahl über die erste Brillouin-Zone zu folgendem berechnen:
\begin{equation*}
	U = \frac{3 V \hbar}{2 \pi^2 v_{\text{s}}^3} \int_{0}^{\omega_{\text{D}}} \! \frac{ \omega^3 }{ \exp{ \left(  \frac{\hbar \omega}{k_{\text{B}} T} - 1  \right)} } \, d\omega
	  = \frac{3 V \hbar}{2 \pi^2 v_{\text{s}}^3} \int_{0}^{\omega_{\text{D}}} \! \frac{ \omega^3 }{ \exp{ \left(  \frac{\Theta_{\text{D}}}{ T} - 1  \right)} } \, d\omega
\end{equation*}

\FloatBarrier
