\section{Diskussion}
In diesem Versuch wurde die Debeye-Temperatur von Kupfer experimentell mit \\$\theta_{D_{exp}} = 346 \pm 11 \si{\kelvin}$ bestimmt.
Der Literaturwert für Kupfer beträgt  $\theta_{D} = 345 \si{\kelvin}$ und liegt bei einer Abweichung von $0{,}29 \%$ innerhalb der Messungenauigkeiten.
Auch der zuvor berechnete Wert von $\theta_{D} = 332 \si{\kelvin}$ liegt mit einer Abweichung von $3{,}77\%$ innerhalb der Messungenauigkeiten.
Eine mögliche Fehlerquelle besteht darin, dass im Aufbau kein perfektes Vakuum erzeugt werden kann und somit trotztdem Wärmetransport stattfinden kann.
Eine weitere mögliche Fehlerquelle ist, dass die Temperaturdifferenz zwischen Zylinder und Probe gerade am Anfang deutlich von Null unterscheidbar war, wodurch von Probe und Zylinder unterschiedliche Wärmestrahlungen ausgingen, die die Ergebnisse beeinflusst haben könnte.
