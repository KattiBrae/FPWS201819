%
%- messdaten nicht sehr gut
%- unsicherheit von $3\%$ nicht immer eingehalten
%- differenzieller WQ: Messdaten weisen auf einen systematischen Fehler hin, möglicherweise ist der Träger für die Folie nicht tief genug im Strahlengang und deckt den Detektor ab?
%Träger ist vorne und hinten asymmetrisch und das Problem ist bei der Messung nicht aufgefallen.
%
Die Ergebnisse dieses Versuchs weichen in einigen Stellen von den erwarteten Werten ab.
Die Aktivität der Probe ergibt sich mit dem Zerfallsgesetz und dem angegebenen Wert zu
\begin{equation*}
	A_{\text{theo}} = \SI{15.64(48)}{\kilo\becquerel};
\end{equation*}
die Messungen ergeben eine Aktivität von
\begin{equation*}
	A_{\text{exp}}= \SI{501(15)}{\kilo\becquerel}.
\end{equation*}
Die relative Abweichung der beiden Werte beträgt $\SI{97}{\%}$.
Im Falle der Messung der Foliendicke ergeben die ausgewerten Messwerte eine Dicke der Goldfolie von
\begin{equation*}
	d_{\text{exp}}= \SI{6,5(5)}{\micro \meter}.
\end{equation*}
Die Probe gibt an, dass die Folie $d_{\text{theo}} =\SI{2}{\micro m}$ dick ist.
Der gemessene Wert weicht um $\SI{225}{\%}$ vom angegebenen Wert ab.
Diese Abweichung kann durch das ungenaue Ablesen der stark fluktuierenden Pulshöhen auf dem Oszilloskop zustande kommen.
Bei der Messung des Differenziellen Wirkungsquerschnitts zeigt sich eine starke systematische Abweichung.
Als Erklärung für die fehlende Winkelabhängigkeit der Messdaten lässt sich eine möglicherweise fehlerhafte Position der Folie rekonstruieren.
Es ist möglich, dass ein Teil des Träger der Folie im Strahlengang war, statt der Folie.
Jedoch ist zu erwarten, dass der Differenzielle Wirkungsquerschnitt erst bei großen Ablenkwinkeln relevant wird, da die Teilchen einen längeren Weg innerhalb der Folie zurücklegen müssen.
Im Versuchsteil zur Mehrfachstreuung lässt sich feststellen, dass eine Verdopplung der Foliedicke den Differenziellen Wirkungsquerschnitt um drei Größenordnungen reduziert.
Der erwartete lineare Zusammenhang der Aktivität am Detektor und der Ordnungszahl des Folienmaterials kann nicht bestätigt werden.
Der Umfang der Stichprobe ist sehr gering, so ist ein nicht bestätigter Zusammenhang bei drei Messpunkten (drei Materialien) durchaus zu erklären. 
%Abschließend ergibt sich für den Zusammenhang zwischen der Ordnungszahl des Folienmaterials und der Aktivität am Detektor.
Ein weiteres Problem im Rahmen dieses Versuchs ist, dass die Mindestzählrate von $N=1000$ für eine maximale Unsicherheit von $\SI{3}{\%}$ nicht konsequent eingehalten wurde.
So sind auch deutlich größere Unsicherheiten in diesem Versuch aufgenommen worden.
