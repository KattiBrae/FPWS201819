%
%- messdaten nicht sehr gut
%- unsicherheit von $3\%$ nicht immer eingehalten
%- differenzieller WQ: Messdaten weisen auf einen systematischen Fehler hin, möglicherweise ist der Träger für die Folie nicht tief genug im Strahlengang und deckt den Detektor ab?
%Träger ist vorne und hinten asymmetrisch und das Problem ist bei der Messung nicht aufgefallen.
%
Die Ergebnisse dieses Versuchs weichen in einigen Stellen von den erwarteten Werten ab.
Im Falle der Messung der Foliendicke ergeben die ausgewerten Messwerte eine Dicke der Goldfolie von
\begin{equation*}
	d= DICKE DER FOLIE %\SI{}{\micro m}.
\end{equation*}
Die Probe gibt an, dass die Folie $d=\SI{2}{\micro m}$ dick ist.
Die beiden Werte weichen um $\huge{\SI{000000000000000000000000}{\%}}$ voneinander ab.
Bei der Messung des Differenziellen Wirkungsquerschnitts zeigt sich eine starke systematische Abweichung.
Als Erklärung für die fehlende Winkelabhängigkeit der Messdaten lässt sich eine möglicherweise fehlerhafte Position der Folie rekonstruieren.
Es ist möglich, dass ein Teil des Träger der Folie im Strahlengang war, statt der Folie.
Im Versuchsteil zur Mehrfachstreuung lässt sich der Abhängigkeit
\begin{equation*}
	ZUSAMMENHANG
\end{equation*}
aufstellen.
Abschließend ergibt sich für den Zusammenhang zwischen der Ordnungszahl des Folienmaterials und der Aktivität am Detektor:
\begin{equation*}
	ZUSAMMENHANG.
\end{equation*}
Ein weiteres Problem im Rahmen dieses Versuchs ist, dass die Mindestzählrate von $N=1000$ für eine maximale Unsicherheit von $\SI{3}{\%}$ nicht konsequent eingehalten wurde.
So sind auch deutlich größere Unsicherheiten in diesem Versuch aufgenommen worden.
