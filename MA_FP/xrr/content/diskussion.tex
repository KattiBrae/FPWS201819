\section{Diskussion}
Initial lässt sich sagen, dass der Versuch nicht von den Erwartungen abweicht.
Die gemessenen Größen liegen in den Größenordnungen der Literaturwerte und der eigenständig reproduzierten Werte.
So weicht die abgeschätzte Schichtdicke $z_{\text{Minima}}$ nur gering von der manuell angepassten Schichtdicke $z_{\text{Theoriekurve}}$ ab:
\begin{align*}
		\text{Minima}					 && \text{Theoriekurve}  	& \text{Rel. Abw.}		\\
		z = \SI{882.41 \pm 0.03 e-10}{m} && z = \SI{870e-10}{m}		& \SI{1.43}{\%}. 		\\
\end{align*}
Die Abweichung lässt sich einerseits durch eine nicht optimale Anpassung der Parameter der Theoriekurve erklären, andererseits sind nur begrenzt viele Minima in den anderen Wert der Schichtdicke eingegangen.
Eine weitere Messung mit längerer Messdauer pro Winkel könnte die Intensität bei großen Winkeln $\alpha_i$ bzw. Wellenvektorüberträgen $q$ auf repräsentative Größen anheben.
In der Messung in diesem Versuch werden die Messdaten bei größeren $q$ vom Rauschen überlagert.
Zu den Dispersionen $\delta$ (in $n = 1 - \delta$) liegen die Literaturwerte \cite{TOL} bei
\begin{align*}
				& \text{Literatur}					& \text{Angepasste Parameter} 		& \text{Rel. Abw.}	\\
	\text{PS}	& \delta = \SI{3.5e-6}{\nothing}	& \delta = \SI{0.7e-6}{\nothing}	& \SI{80}{\%} 		\\
	\text{Si}	& \delta = \SI{7.56e-6}{\nothing}	& \delta = \SI{6.6e-6}{\nothing}	& \SI{12.70}{\%}.	\\
\end{align*}
Dabei fällt gerade die starke Abweichung des Korrekturterms des Polystyrols ins Auge.
Eine mögliche Erklärung hierfür ist eine möglicherweise ungenaue Wahl des Parameters.\\
Weiterhin sind aber die kritischen Winkel in der passenden Größenordnung \cite{TOL}:
\begin{align*}
			& \text{Literatur}		& \text{aus der Angepassung} 	& \text{Rel. Abw.}	\\
\text{PS}	& \delta = \SI{0.15}{°}	& \delta = \SI{0.068}{°}		& \SI{54.67}{\%} 	\\
\text{Si}	& \delta = \SI{0.22}{°}	& \delta = \SI{0.208}{°}		& \SI{5.45}{\%}.	\\
\end{align*}
Insgesamt trotz einiger Startschwierigkeitnen lässt sich der Versuch jedoch als Erfolg betrachten.
