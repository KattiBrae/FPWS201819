\subsection{Einregeln der optimalen Verzögerungszeit}
\begin{table}[h!]
  \centering
  \caption{Messdaten zur Optimierung der Verzögerungszeit der Kabel}
  \label{tab:kalibrierung}
  \begin{tabular}{c c c}
    \toprule
%    \multicolumn{3}{c}{$f_{\text{1, theo}}=\SI{0.1}{m}$} & \multicolumn{3}{c}{$f_{\text{2, theo}}=\SI{0.05}{m}$}\\
      $T_{\text{VZ}}$ & Impulshöhe \\
      \midrule
      -32   &   2     \\
      -30   &   8     \\
      -28   &   15    \\
      -26   &   55    \\
      -24   &   75    \\
      -22   &   141   \\
      -20   &   168   \\
      -18   &   185   \\
      -16   &   198   \\
      -14   &   196   \\
      -12   &   180   \\
      -10   &   214   \\
      -8    &   189   \\
      -6    &   189   \\
      -4    &   197   \\
      -2    &   227   \\
      0     &   208   \\
      2     &   216   \\
      4     &   217   \\
      6     &   214   \\
      8     &   212   \\
      10    &   200   \\
      12    &   194   \\
      14    &   189   \\
      16    &   161   \\
      18    &   97    \\
      20    &   84    \\
      22    &   38    \\
      24    &   4     \\

    \bottomrule
  \end{tabular}
\end{table}

%\end{landscape}
%\end{document}

\begin{figure}[h!]
  \centering
  \includegraphics[width=\textwidth]{figverzögerung.pdf}
  \caption{Optimierung der Verzögerungszeit: Verzögerungszeit $T_{\text{VZ}}$ gegen Spannungsimpuls $U$}
  \label{fig:verzögerung}
\end{figure}
Fit
\begin{equation*}
  N = -a \left( T_{\text{VZ}} +b \right)^4+c
\end{equation*}
Fitparameter
\begin{align*}
  a &=& \SI{4.82149138 \pm 0.236142743e-04}{}\\
  b &=& \SI{2.08543593 \pm 0.229205656}{}\\
  c &=& \SI{208.001211 \pm 3.50366199}{}\\
\end{align*}

\FloatBarrier
\subsection{Kalibrierung des Multi-Channel-Analysers}
\begin{table}[h!]
  \centering
  \caption{Messdaten zu Kalibrierung des Multi-Channel-Analysers}
  \label{tab:kalibrierung}
  \begin{tabular}{c c c}
    \toprule
%    \multicolumn{3}{c}{$f_{\text{1, theo}}=\SI{0.1}{m}$} & \multicolumn{3}{c}{$f_{\text{2, theo}}=\SI{0.05}{m}$}\\
      Channel & $\Delta$ t \\
      \midrule
         24   &   1407   \\
         46   &   1561   \\
         69   &   1400   \\
         91   &   1294   \\
        113   &   1298   \\
        136   &   1034   \\
        158   &   1502   \\
        180   &   1336   \\
        203   &   1700   \\
        225   &   1644   \\
        247   &   1680   \\
        270   &   1555   \\
        292   &   1608   \\
        315   &   1384   \\
        337   &   1952   \\
        359   &   1880   \\
        382   &   2008   \\
        404   &   2088   \\
        427   &   2024   \\
        445   &   3384   \\

    \bottomrule
  \end{tabular}
\end{table}

%\end{landscape}
%\end{document}

\begin{figure}[h!]
  \centering
  \includegraphics[width=\textwidth]{figkalibrierung.pdf}
  \caption{Kalibrierung des Multi-Channel-Analysers: Zeitlicher Abstand des Doppelimpulses $\Delta$ t gegen den zugehörigen Channel}
  \label{fig:kalibrierung}
\end{figure}
Fit: C $\widehat{=}$ Channel
\begin{equation*}
\Delta t = a \cdot C + b
\end{equation*}
Fitparameter:
\begin{align*}
a  &=&  \SI{ 0.02234091 \pm 1.28401231692e-05}{\frac{1}{s}}  \\
b  &=&  \SI{-0.03080493 \pm 0.00345318109864 }{}  \\
\end{align*}

\FloatBarrier

\subsection{Messung der Lebensdauer}
\begin{figure}[h!]
  \centering
  \includegraphics[width=\textwidth]{figmyonen.pdf}
  \caption{Häufigkeit der Myonenzerfälle in Abhängigkeit ihrer Lebensdauer}
  \label{fig:myonen}
\end{figure}
Fit
\begin{equation*}
y = a \exp{(-x b)}+c
\end{equation*}
Fitparameter
\begin{align*}
a  &=&  \SI{51.81005345 \pm 0.9677403}{}\\
b  &=&  \SI{0.52824244 \pm 0.01864355}{}\\
c  &=&  \SI{0.74055349 \pm 0.31446226}{}\\
\end{align*}
\FloatBarrier
