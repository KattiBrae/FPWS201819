Initial lässt sich sagen, dass der Versuch gut verlaufen ist.
Alle gemessenen Werte sind in der passenden Größenordnung.
%Die Kalibrierung des Multichannelanalysers führt zu einkanaligen Messergebnissen und muss daher nicht gemittelt werden.
Zum Vergleich der Werte wird die relative Abweichung berechet durch:
\begin{equation*}
  f=\frac{x_{\text{exp}}-x_{\text{theo}}}{x_{\text{theo}}}.
\end{equation*}
Die Lebensdauer der Myonen wird als $\tau_{\text{exp}}$ gemessen und mit dem Literaturwert $\tau_{\text{theo}}$ \cite{Demtröder} verglichen.
\begin{align*}
  \tau_{\text{exp}}   &=&   \SI{1.8932e-06}{s}  \\
  \tau_{\text{theo}}  &=&   \SI{2.199e-06}{s}   \\
  f_{\tau}            &=&   \SI{13.91}{\%}      \\
\end{align*}
Die Abweichung der Lebensdauern lässt sich unter anderem durch folgendes erklären.
In die Messaparatur werden mehrere Bauteile gebaut um die Messung genauer zu gestalten.
So wird die Messung über zwei Photokathoden und zwei SEV an dem Szintillatortank gemacht.
Das Auftrennen und wieder Zusammenführen der beiden Signale über die AND-Gatter selektiert die Signale zugunsten der Signale der einzelnen, im Tank zerfallenden, Myonen.
Dennoch kann durch den Aufbau der Messung das Auftreten zweier fast hochenergetischer Myonen mit dem passenden zeitlichen Abstand des Eintritts in den Szintillatortanks nicht vollständig ausgeschlossen werden.\\
Die Untergrundrate ergibt sich als $U_{\text{0,exp}}$ aus den Fitparametern und wird mit dem theoretischen, berechneten Wert $U_{\text{0,theo}}$ verglichen:
\begin{align*}
  U_{\text{0,exp}}    &=&  \SI{0.7406 \pm 0.3145}{} \\
  U_{\text{0,theo}}   &=&  \SI{0.5194416685028387 \pm 0.720723017881}{}   \\
  f_{\text{U}}        &=&  \SI{42.58}{\%}. \\
\end{align*}
Die Abweichung der Untergrundrate ist unerwartet groß.
Auffällig ist außerdem die große Differenz der Zählwerke:
\begin{align*}
  \text{Zählrate Start}    && N_{\text{Start}}  &=\SI{1397196}{} \\
  \text{Zählrate Stop}     && N_{\text{Stop}}   &=\SI{4719}{} \\
  \text{Differenz}         && \Delta N          &=\SI{1392477}{}.\\
\end{align*}
Zwischen dem TAC und dem MCA gehen $\SI{99.66}{\%}$ der Startsignale verloren.
Eine mögliche Erklärung ist, dass der Multichannelanalyser eine zu geringe Auflösung hat und sehr gerine Zeitabstände zwischen dem Start- und dem Stopsignal nicht darstellen kann.
Außerdem wird die theoretische Untergrundrate nur durch die Zählrate der Startsignale berechnet, während der experimentelle Wert durch die Signale berechnet wird, bei denen sowohl Start- und Stopsignal vorhanden sind.
