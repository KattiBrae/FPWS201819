Myonen lassen sich mit Hilfe eines Szintilationsdetektors nachweisen.
Die energiereichen Myonen geben einen Teil ihrer Energie an Szinitilationsmoleküle ab.
Diese kommen in einen angeregten Zustand.
Beim Rücksprung in den Grundzustand wird ein Photon emitiert.
Es besteht die Warscheinlichkeit, dass ein Myon im Szinitlatormateriel ebenfalls nach der Gleichung \ref{eqn.mu-}
oder \ref{eqn.mu+} zerfällt.
Das so entstandene Elektron besitzt genug kinetische Energie um Seinerseits Atome anzuregen und so ebenfals Photonen zu erzeugen.
Mit dem Abstand dieser beiden Lichtimpulse lässt sich die Lebensdauer einen Myons bestimmen.

\subsektion{Aufbau}
Der Szintilationsdetektor besteht aus einem Edelstahlzylinder welcher ca \SI{50}{l} fassen kann.
Um eine gute Zeitauflösung zu bekommen wird als Szintilatormaterial ein Organisches material benutzt,
da dieses schnell in seinen Gundzustand zurück springt.
Die Abklingdauer des Szintilatormaterials beträgt ca. \SI{10}{s} und liegt damit unterhalb der Lebensdauer von Myonen.
Die einzelnen emitierten Photonen werden mit einem Sekundärelektronenvervielfacher(SEV) Messbar gemacht.
Diese sind an beiden Seiten des Szintilators befestigt und werden mit Hochspannung betrieben.
Der elektrische Impuls, welcher durch das eindringen eines Myons ins Szintilatormaterial
und durch das zerfallen dieses Myons im Szintilatormaterial entsteht,
wird an ein Zeit-Amplituden-Konverter (TAC) weitergegeben.
Dieser misst die Zeit zwischen den Signalen und gibt diese als Impulshöhe aus.
Mit Hilfe eines Vielkanalanalysator werden die Impulshöhen der Häufigkeit nach sortiert.
