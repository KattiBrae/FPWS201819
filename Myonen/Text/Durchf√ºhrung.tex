Myonen lassen sich mit Hilfe eines Szintilationsdetektors nachweisen.
Die energiereichen Myonen geben einen Teil ihrer Energie an Szinitilationsmoleküle ab.
Diese kommen in einen angeregten Zustand.
Beim Rücksprung in den Grundzustand wird ein Photon emitiert.
Es besteht die Warscheinlichkeit, dass ein Myon im Szinitlatormateriel ebenfalls nach der Gleichung \ref{eqn:mu-}
oder \ref{eqn:mu+} zerfällt.
Das so entstandene Elektron besitzt genug kinetische Energie um Seinerseits Atome anzuregen und so ebenfals Photonen zu erzeugen.
Mit dem Abstand dieser beiden Lichtimpulse lässt sich die Lebensdauer einen Myons bestimmen.

\subsection{Aufbau}
Der Szintilationsdetektor besteht aus einem Edelstahlzylinder welcher ca \SI{50}{l} fassen kann.
Um eine gute Zeitauflösung zu bekommen wird als Szintilatormaterial ein Organisches material benutzt,
da dieses schnell in seinen Gundzustand zurück springt.
Die Abklingdauer des Szintilatormaterials beträgt ca. \SI{10}{s} und liegt damit unterhalb der Lebensdauer von Myonen.
Die einzelnen emitierten Photonen werden mit einem Sekundärelektronenvervielfacher(SEV) Messbar gemacht.
Diese sind an beiden Seiten des Szintilators befestigt und werden mit Hochspannung betrieben.
Der elektrische Impuls, welcher durch das eindringen eines Myons ins Szintilatormaterial
und durch das zerfallen dieses Myons im Szintilatormaterial entsteht,
wird an ein Zeit-Amplituden-Konverter (TAC) weitergegeben.
Dieser misst die Zeit zwischen den Signalen und gibt diese als Impulshöhe aus.
Mit Hilfe eines Vielkanalanalysator werden die Impulshöhen der Häufigkeit nach sortiert.

Da einige Störfaktoren hinzukommen muss die Apperatur erweitert werden.
Im SEV können sich einzelnen Elektronen lösen und ebenfals ein Signal implizieren.
Die Amplitude diser Rauschsignale ist hingegen meist geringer und kann so mit einer Diskriminatorschwelle herausgefiltert werden.
Der Diskriminator normiert die eingehenden Signale zusätzlich auf eine einheitliche Höhe und Länge.
Eine weitere möglichkeit der Rauschunterdrückung wird mit einer Koinzidenzschaltung verwirklicht.
Ein emitiertes Photon wird von beiden SEV gemessen.
Die resultierende Impulse werden über seperate Kabel und Diskriminatoren in eine Koinzidenzschaltung geführt.
Da die Leitfähigkeit elektrische beuteile Variieren kann,
können mit einer Verzögerungsleitung die Impulse aufeinander abgepasst werden.
Der Koinzidenzschalter gibt nur dann ein Signal weiter wenn an beiden Eingängen ein Impuls ankommt.
Die Warscheinlichkeit, dass beide SEV gleichzeitig ein stakes Rauschsignal empfangen ist sehr gering.

Ein weiteres Problem ist, dass nur ein kleiner Teil der Myonen welche in der Szintilator eindringen dort auch zerfallen.
Das bedeutet das der TAC ein Startsignal bekommt aber ein Stopsignal.
Da bei unendlich langer Suchzeit das ein neu eintreffendes Myon ein Stopsignal geben würed anstand ein neues Startsignal,
muss die Suchzeit $T_S$ nach einer gewissen Zeit abgebrochen werden.
Der Suchzeitgeber wird mit einer monostabilen Kippstufe oder auch univibrator verwirklicht.
Nach Aktivierung bleibt die Kippstufe über die Zeit $T_S$ in einem angeregen Zustand und spring dann in seinen Grundzustand zurück.
