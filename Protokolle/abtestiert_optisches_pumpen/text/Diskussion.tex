Das in Abbildung \ref{fig:bild} dargestellte Signalbild entspricht im groben den Erwartungen.
Das ermittelte Isotopenverhältnis weicht leicht von den Literaturwerten \cite{Ver} ab.
\begin{align*}
  \text{Experimentell}&&\text{Literatur}\\
  N_{Rb^{87}}&=\SI{0,32}{}&N_{Rb^{87}}&=\SI{0,278}{}\\
  N_{Rb^{85}}&=\SI{0,68}{}&N_{Rb^{85}}&=\SI{0,722}{}\\
\end{align*}
Die ermittelten Landé-Faktoren liegen sehr nah an den Literaturwerten\cite{Ver}.
\begin{align*}
  \text{Experimentell}&&\text{Literatur}\\
  g_{Rb^{87}} &= \SI{0,497\pm0,008}{}&g_{Rb^{87}} &= \frac{1}{2}\\
  g_{Rb^{85}} &= \SI{0,334\pm0,004}{}&g_{Rb^{85}} &= \frac{1}{3}\\
\end{align*}
Die Abweichung des Kernspins von den Literaturwerten \cite{Spin} ist ebenfalls sehr gering.
\begin{align*}
  \text{Experimentell}&&\text{Literatur}\\
  I_{Rb^{87}}&=\SI{1,5153\pm0,03}{}&I_{Rb^{87}}&=\frac{3}{2}\\
  I_{Rb^{85}}&=\SI{2,4999\pm0,04}{}&I_{Rb^{85}}&=\frac{5}{2}\\
\end{align*}
