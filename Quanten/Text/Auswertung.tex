In der Messung werden für den Frequenzbereich von 6000 bis 9000 $\si{Hz}$ Rohrlängen von $\SI{75}{mm}$ bis $\SI{600}{mm}$ vermessen.
Bei bestimmten Frequenzen bildet sich eine stehende Welle aus. Das hat Resonanz zur Folge, so dass die Schallwelle eine höhere Intensität hat.
In den folgenden 2 Abbildungen wird die Intensität gegen die Wellenlänge aufgetragen.
In der Abbildung \ref{fig.1} sind die Messwerte für die Rohrlängen : $\SI{75}{mm}, \SI{150}{mm}, \SI{225}{mm}, \SI{300}{mm}$ abbgebildet.
\begin{figure}[h!]
  \centering
  \includegraphics[width=\textwidth]{1234.pdf}
  \caption{Intensität der Schallwelle im Bereich von 6000 bis 9000 $\si{Hz}$ für die Rohrlängen $\SI{75}{mm}, \SI{150}{mm}, \SI{225}{mm}, \SI{300}{mm}$}
  \label{fig.1}
\end{figure}
In der Abbildung \ref{fig.2} sind die Messwerte für die Rohrlängen : $\SI{375}{mm}, \SI{450}{mm}, \SI{525}{mm}, \SI{600}{mm}$ abbgebildet.
\begin{figure}[h!]
  \centering
  \includegraphics[width=\textwidth]{5678.pdf}
  \caption{Intensität der Schallwelle im Bereich von 6000 bis 9000 $\si{Hz}$ für die Rohrlängen $\SI{375}{mm}, \SI{450}{mm}, \SI{525}{mm}, \SI{600}{mm}$}
  \label{fig.2}
\end{figure}
Die Intensitätsmaxima sind mit einem blauen Punkt gekennzeichnet.
Auffällig ist, dass sich für alle Rohrlängen Intensitätsmaxima an der Stelle von $\SI{6880}{Hz}$ bilden. Das heißt, dass bei dieser Frequenz eine stehende Welle entsteht die eine Wellenlänge von $\lambda=\frac{1}{2n}d$ besitzt.
$d$ ist dabei die Länge des Rohres und n die Anzahl der Knotenpunkte.
Desweiteren ist auffällig, dass für eine Rohrlänge die Abstand zwichen zwei Maxima immer konstannt ist.

\subsection{Bestimmung der Lichtgeschwindigkeit}
In den Graphen \ref{fig.2} und \ref{fig.2} werden die Resonanzfrequenzen einer jeden Messreihe von 1 bis n nummeriert und anschließend gegen die Frequenz aufgetragen.
Die Resonazfrequenzabstände sind konstant für eine Messreihe, daher ergeben sich Geraden der Form: $y=ax+b$ welche an die Messwerte gefittet werden.
In der Abb. \ref{fig.linearfit} sind die Messwerte mit den Fitfunktionen abgebildet.
%Für die Fitparameter ergeben sich folgende Werte.
%\begin{align*}
%  a &= 286.54545455 \pm 0.37848474\\
%  b &= 5737.09090907 \pm 2.56700835
%\end{align*}
Mit Hilfe der Steigung $a$ kann die Schallgeschwindigkeit $c$ berechnet werden.
\begin{align*}
  c = a\cdot2d
\end{align*}
Für die Schallgeschwindigkeit ergeben sich so Werte von:
\begin{align*}
a &= 1140.0 && d = 2\cdot\SI{0.075}{m} && c=\SI{342.0}{\frac{m}{s}}\\
a &= 763.0  && d = 3\cdot\SI{0.075}{m} && c=\SI{343.3}{\frac{m}{s}}\\
a &= 571.0  && d = 4\cdot\SI{0.075}{m} && c=\SI{342.6}{\frac{m}{s}}\\
a &= 458.6  && d = 5\cdot\SI{0.075}{m} && c=\SI{343.9}{\frac{m}{s}}\\
a &= 381.9  && d = 6\cdot\SI{0.075}{m} && c=\SI{343.7}{\frac{m}{s}}\\
a &= 327.2  && d = 7\cdot\SI{0.075}{m} && c=\SI{343.5}{\frac{m}{s}}\\
a &= 286.5  && d = 8\cdot\SI{0.075}{m} && c=\SI{343.8}{\frac{m}{s}}\\
\end{align*}
Der Mittelwert ergibt:
\begin{align*}
  c=\SI{343.28\pm1.63}{\frac{m}{s}}.
\end{align*}
%###########Bild##########
%\begin{figure}[h!]
%  \centering
%  \includegraphics[width=\textwidth]{A1L8x75mmF6000-9000S10.pdf}
%  \caption{Messung eines $\SI{0.6}{m}$ Rohres für den Frequenzbereich von 6000 bis 9000 Hz}
%  \label{fig.frequenz/rohrlänge}
%\end{figure}
\begin{figure}[h!]
  \centering
  \includegraphics[width=\textwidth]{linearfit.pdf}
  \caption{Die Intensitätsmaxima werden nummeriert und die Nummer gegen die Frequenz aufgetragen.}
  \label{fig.linearfit}
\end{figure}
%#########
\FloatBarrier
Die Schallgeschwindigkeit kann auch mit Hilfe einer anderen Methode bestimmt werden.
dazu betrachten wir ein Rohr der Länge $\SI{75}{mm}$ für die Frequenzen von 6000 bis 9000 $\si{Hz}$.
Da die Rohrlänge $d$ ein vielfaches $n$ der halben Wellenlänge $\lambda$ ist gilt:
\begin{align*}
  \lambda = \frac{2d}{n}.
\end{align*}
Die Schallgeschwindigkeit ist gegeben durch:$c=f \cdot \lambda$.
Daraus ergibt sich :
\begin{align*}
  n&=1  c&=\SI{1032}{\frac{m}{s}}\\
  n&=2  c&=\SI{516}{\frac{m}{s}}
\end{align*}
\begin{align}
  n&=3  c&=\SI{344}{\frac{m}{s}}
  \label{eqn.schall}
\end{align}
\begin{align*}
  n&=4  c&=\SI{258}{\frac{m}{s}}.
\end{align*}
Die Freqenz ist dabei gegeben als $\SI{6880}{\frac{m}{s}}$.
Diese Freqenz ist die einzige Frequenz im Frequenzbereich welche bei allen Vielfachen der Rohrlänge von $\SI{75}{mm}$ vertreten ist,
zusehen ist dies in den Abb. \ref{fig.1} und \ref{fig.2}.
\textbf{\huge{Den Ergebnissen aus \ref{eqn.schall} ist zu entnehmen, dass die Wellenlänge bei $\frac{3}{2}\lambda$ liegt, da die dazugehörige Schallgeschwindigkeit von $\SI{344}{\frac{m}{s}}$ gut zum Literaturwert??? von $\SI{343.2}{\frac{m}{s}}$ passt.}}

\FloatBarrier
Desweiteren soll die Schallgeschwindigkeit mithilfe des Verhältnisses von Frequenzübergang zu Rohrlänge bestimmt werden.
Dieser Zusammenhang ist in Abb. \ref{fig.1/x} dargestellt.
An die Messwerte wird eine Funktion der Form $a\cdot\frac{1}{x}+b$ gefittet.
Für a und b ergegben sich die Werte:
\begin{align*}
  a&=168692.913\pm253.977\\
  b&=8.566\pm1.995\\
\end{align*}
%################Bild#####################
\begin{figure}[h!]
  \centering
  \includegraphics[width=\textwidth]{geschi.pdf}
  \caption{Der Abstand der Frequenzen wird gegen die Rohrlänge Aufgretragen.}
  \label{fig.1/x}
\end{figure}
\FloatBarrier

\subsection{Aufgabe 2: Dispersion}
In dieser Aufgabe wird $f(k)$ geplottet.
Aus der Messung mit einer Rohrlänge von $12\cdot\SI{50}{mm}$ und einem Frequenzbereich von $400-12000\si{Hz}$ werden die Maximalstellen mit Hilfe einer Peak-Piking ermittelt.
Die Maximalstellen werden nummeriert und die darzugehörige Freqenz in $f(k)$ umgerechnen.
\begin{align}
  k= \frac{n \pi}{L}
\end{align}
$L$ ist dabei die gesammte Rohrlänge.

In der Abbildung \ref{fig.Aufgabe2} ist $k$ gegen $f(k)$ aufgetragen. Dargestellt ist zum einen die Dispersionsrelation von Schallwellen und zum anderen die Dispersionsrelation eines quantenmechanischen Teilchens im Potenzialtopf nach der Funktion \ref{eqn:dispersionqm}.
Es ist zu erkennen, dass die Dispersion von Schalwellen linear verläuft wohingegen die Dispersion eines quantenmechanischen Teilchens im Potenzialtopf quadratisch ist.
Das Analogon zwischen den beiden Dispersionsrelationen ist daher nicht sehr zutreffend.
\begin{figure}[h!]
  \centering
  \includegraphics[width=\textwidth]{f(k).pdf}
  \caption{Dispersionsrelation für Schalwellen X und die Theorikurve eines quantenmechanischen Teilchens im Potenzialtopf}
  \label{fig.Aufgabe2}
\end{figure}
\FloatBarrier

\subsection{Aufgabe 3: Bandstruckturen}
In dieser Aufgabe werden Atome in einer Kette simmuliert.
Dazu wird ein Rohr der Länge 12 mal 50 $\si{mm}$ mit Blenden unterteilt.
Die Blendengrößen werden zwischen 10, 13 und $\SI{16}{mm}$ variiert.
In der Abbildung \ref{fig.Aufgabe3} werden die Bandlücken in Reation zu der Blendengröße gesetzt.
Es wird deutlich, dass die Bandlücken mit zunehmender Blendengröße kleiner werden.
Die Bandlücken werden bestimmt indem man die Differenz vom Anfangs- und End- punkt der Bandlücke betrachtet.
\begin{figure}[h!]
  \centering
  \includegraphics[width=\textwidth]{Bandlücken3.pdf}
  \caption{Größe der Bandlücken aufgetragen gegen die Blendengröße}
  \label{fig.Aufgabe3}
\end{figure}
\FloatBarrier

\subsection{Aufgabe 4: Bandstruckturen}
Diese Aufgabe ist sehr ähnlich zur Aufgabe 3. Hier wird die Bandlücke jedoch im Verhältniss zur Rohrlänge betrachtet.
Dementsprechend ist in der Abbildung \ref{fig.Aufgabe4} die Badlücken gegen die Rohrlängen aufgetragen.
Die Rohrlänge wird zwichen 8, 10 und 12 mal $\SI{50}{mm}$ variiert.
Die Messung ist nicht sehr aussagekräftig.
Lediglich ist zu erkennen, dass die Bandlücken relativ unabhängig von der Rohrlänge sind.
\begin{figure}[h!]
  \centering
  \includegraphics[width=\textwidth]{Bandlücken.pdf}
  \caption{Größe der Bandlücken aufgetragen gegen die Rohrlänge}
  \label{fig.Aufgabe4}
\end{figure}
\FloatBarrier

\subsection{Aufgabe 5}
In dieser Aufgabe wird die Bandlücke ebenfals ins Verhältniss zu der Rohrlänge gesetzte. Dabei wird zwischen 8 mal $\SI{75}{mm}$ und 8 mal $\SI{50}{mm}$ gewechselt.
Es werden dehmentsprechend unterschiedlich große Einheitszellen betrachtet.
In den Abbildungen \ref{fig.Aufgabe5} und \ref{fig.Aufgabe575} sind die Maxima gegen k geplottet. Anhand der hinterlegten Bandlücken wird deutlich, dass die Bandlücken für größere Rohrabschnitte kleiner werden.
In der Abbildung \ref{fig.Aufgabe5a} wird ebenfals deutlich, dass die Größe der Bandlücke fällt mit größeren Einheitszellen.
%\FloatBarrier

\begin{figure}
 \centering
 \begin{subfigure}{0.48\textwidth}
  \centering
  \includegraphics[width=1\textwidth]{bla.pdf}
  \caption{8 mal \SI{50}{mm}}
  \label{fig.Aufgabe5}
 \end{subfigure}
 \begin{subfigure}{0.48\textwidth}
  \centering
  \includegraphics[width=1\textwidth]{bla75.pdf}
  \caption{8 mal \SI{75}{mm}}
  \label{fig.Aufgabe575}
 \end{subfigure}
 \caption{Bandlücken für unterschiedliche Rohrlängen}
 %\label{fig:500-6-7}
\end{figure}

\begin{figure}[h!]
  \centering
  \includegraphics[width=\textwidth]{newtest.pdf}
  \caption{Größe der Bandlücken aufgetragen gegen die größe der Einheitszelle}
  \label{fig.Aufgabe5a}
\end{figure}
\FloatBarrier

\subsection{Aufgabe 6 und 7}
In dieser Aufgabe werden wieder die Maxima gegen k geplottet. Es entsteht wie erwartet eine Gerade welche in Abbildung \ref{fig.Aufgabe6} und \ref{fig.Aufgabe7} zusehen ist.
Anhand der Abbildungen ist zu sagen, dass sich die beiden Messungen nicht viel unterscheiden.

\begin{figure}
 \centering
 \begin{subfigure}{0.48\textwidth}
  \centering
  \includegraphics[width=1\textwidth]{A6.pdf}
  \caption{1 mal \SI{50}{mm}}
  \label{fig.Aufgabe6}
 \end{subfigure}
 \begin{subfigure}{0.48\textwidth}
  \centering
  \includegraphics[width=1\textwidth]{A7.pdf}
  \caption{1 mal \SI{75}{mm}}
  \label{fig.Aufgabe7}
 \end{subfigure}
 \caption{Maxima gegen k}
 %\label{fig:500-6-7}
\end{figure}

\subsection{Aufgabe 8}
Es werden für 2 mal $\SI{50}{mm}$ Rohrlänge unterschiedliche Blendendurchmesser untersucht. Der Blendendurchmesser soll die Stärke von Bindungen zwischen Atomen symbolisieren.
Für die unterschiedlichen Blendendurchmesser entstehen unterschiedlich breite Bandlücken.
In der Abbidung $\ref{fig.Aufgabe8}$ ist der zusammenhang zwischen Blendendurchmesser und Bandlücke dargestellt.
Es wird deutlich, dass bei größeren Blendendurchmessern die Bandlücken kleiner werden.
\begin{figure}[h!]
  \centering
  \includegraphics[width=\textwidth]{A8.pdf}
  \caption{Bandlücke in Abhängigkeit von dem Blendendurchmesser}
  \label{fig.Aufgabe8}
\end{figure}
\FloatBarrier

\subsection{Aufgabe 9}
In dem Aufgabenteil werden die Bandlücken in Abhängigleit des Blendendurchmessers dargestellt.
Für die Blendegrößen von 10, 13 und 16 $\si{mm}$ werden jeweils 3, 4 und 6 Einheitszellen vermessen.
Die gemessenen Abhängigkeiten sind in den Abbildungen \ref{fig.Aufgabe9}, \ref{fig.Aufgabe91} und \ref{fig.Aufgabe92} abgebildet.
Aus den Abbildungen wird deutlich, dass die Bandlücken mit steigender Anzahl von Einheitszellen kleiner werden.
  \begin{figure}
   \centering
   \begin{subfigure}{0.48\textwidth}
    \centering
    \includegraphics[width=1\textwidth]{BandlückenD10.pdf}
    \caption{Blendendurchmesser $D=\SI{10}{mm}$}
    \label{fig.Aufgabe9}
   \end{subfigure}
   \begin{subfigure}{0.48\textwidth}
    \centering
    \includegraphics[width=1\textwidth]{BandlückenD13.pdf}
    \caption{Blendendurchmesser $D=\SI{13}{mm}$}
    \label{fig.Aufgabe91}
   \end{subfigure}
   \caption{Die Bandbreit in abhängigkeit von der Anzahl der Einheitszellen}
   %\label{fig:500-6-7}
  \end{figure}
  \begin{figure}[h!]
    \centering
    \includegraphics[width=\textwidth]{BandlückenD16.pdf}
    \caption{Blendendurchmesser $D=\SI{16}{mm}$ in abhängikeit von der Anzahl der Einheitszellen}
    \label{fig.Aufgabe92}
  \end{figure}
  \FloatBarrier

\subsection{Aufgabe 10}
In diesem Aufgabenteil weden die Bänder unter periodischen Defekten betrachtet.
Vermessen wird ein Rohr der Länge $L=12 \cdot \SI{50}{mm}$ mit abwechselnden Blendengrößen von 16 und 13 $\si{mm}$.
In der Abbildung \ref{fig.Aufgabe10} ist die Bandlücke in Abhängigkeit von der Bandlückennummer dargestellt.
Vergleicht man die Werte mit denen aus Abbildung \ref{fig.Aufgabe92} und \ref{fig.Aufgabe91} für 6 Einheitszellen so wird deutlich, dass
die periodischen Defekte dazu führen, dass die Bandlücken zwischen denen von den Blendengrößen 16 und 13 $\si{mm}$, ohne defekte, liegen.
Die Frequenzen bei denen die Bandlücken entstehen bleiben konstant.
  \begin{figure}[h!]
    \centering
    \includegraphics[width=\textwidth]{A10.pdf}
    \caption{Bandlücken bei alternierenden Blendengrößen}
    \label{fig.Aufgabe10}
  \end{figure}
  \FloatBarrier

\subsection{Aufgabe 11}
In dieser Aufgabe geht es ebenfalls darum das Verhalten von Bandlücken unter Verwendung von Defekten zu beobachten.
Die Defekte treten hier jedoch nicht wie in der Aufgabe zuvor in Form von der Blendengröße auf, sonder die Rohrabschnitte werden verändert.
Für die Messung wird eine alternierende Folge von 50 und $75\si{mm}$ langen Röhren benutzt, welche jeweils mit Blenden vom Durchmesser 16 $\si{mm}$ abgetrennt sind.
Insgesammt werden 10 Rohrabschnitte verwendet.
Die Bandlücken werden nummeriert und wie in Abbildung \ref{fig.Aufgabe11} geplottet.
An der Abbildung ist deutlich zu erkennen das die Größe der Bandlücken nicht mehr wie üblich mit steigender Nummer linear ansteigt.
\begin{figure}[h!]
  \centering
  \includegraphics[width=\textwidth]{A11.pdf}
  \caption{Bandlücken bei alternierenden Rohrlängen}
  \label{fig.Aufgabe11}
\end{figure}
\FloatBarrier

\subsection{Aufgabe 12}
Nun sollen vereinzelnt Defekte in die Kette eingebaut werden.
Es werden 4 Messungen mit unterschiedlichen Defekten durchgeführt.
Es werden wieder die Bandlücken berechnet und nummeriert.
In der Abbildung \ref{fig.Aufgabe12} ist der Zusammenhang abgebildet.
Aus der Abbildung folgt die Annahme das kleinere Defekte keinen großen Einfluss auf das Verhalten der Kette haben.
  \begin{figure}[h!]
    \centering
    \includegraphics[width=\textwidth]{A12.pdf}
    \caption{Bandlücken bei alternierenden Rohrlängen}
    \label{fig.Aufgabe12}
  \end{figure}
  \FloatBarrier
