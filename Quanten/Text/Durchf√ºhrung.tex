%Aufbau
%- Schiene mit Lautsprecher und Mikrofon, zwischen denen die die Röhren eingeklemmt werden
%- Röhren verschiedener Längen (75mm, 50mm, 12.5mm)
%- Scheiben verschiedener Durchmesser (10mm, 13mm, 16mm)
%- Computer

Der Aufbau setzt sich zusammen aus einer Schiene mit einem Lautsprecher und einem Mikrofon, zwischen denen die Röhren eingeklemmt werden.
Lautsprecher und Mikrofon schließen dicht mit den Röhren ab.
Die Röhren bestehen aus Aluminium und sind in drei verschiedenen Ausführungen vorhanden.
Während der Durchmesser jeweils gleich ist, gibt es drei verschiedene Längen mit $L_{75} = \SI{75}{mm}$, $L_{50} = \SI{50}{mm}$ und für die letzten Versuchsteile $L_{12.5} = \SI{12.5}{mm}$.
Außerdem gibt es drei verschiedene Arten Blenden, Scheiben mit einem Loch in der Mitte, nämlich die Lochdurchmesser $d_{10} = \SI{10}{mm}$, $d_{13} = \SI{13}{mm}$ und $d_{16} = \SI{16}{mm}$.
Die initiale Datenaufnahme und -verarbeitung wird mit einem Rechner gemacht, an den sowohl Mikrofon als auch Lautsprecher angeschlossen sind.\\
%
%Durchführung
In der Durchführung des Versuchs werden immer Frequenzspektren aufgenommen.
Dies wird vom Rechner gesteuert.
Dabei können verschiedene Parameter vorgegeben werden, darunter die Größe des aufgenommenen Frequenzbereiches, die Messzeit pro Schritt und die Schrittgröße.\\
%- 1: Schallgeschwindigkeit:
%    - QA1: Seite 6 unten (Experiment): 8x Frequenzspektrum für verschiedene Rohrlängen (75mm-Röhren), kleine Frequenzbereiche
%    - QA1: Seite 14 oben (Setup): 1x Frequenzspektrum mit kleinen Schritten (1x Steps 2Hz)
In der ersten Messung wird die Schallgeschwindigkeit in Luft gemessen.
Dazu werden acht Frequenzspektren mit den $L_{75}$-Röhren mit der Schrittlänge steps $= \SI{10}{Hz}$ aufgenommen.
Nach jeder Messung wird die Anzahl der Röhren um eine erhöht, bis acht Röhren erreicht sind.
Aus den verschiedenen Frequenzspektren werden die jeweiligen Resonanzfrequenzen notiert.
Außerdem wird ein Frequenzspektrum mit kleiner Schrittlänge (steps $= \SI{2}{Hz}$) aufgenommen.
%
- 2: Eindimensionaler Festkörper / Dispersion Frequenz (k)
    - QA4: Seite 2 unten (Experiment): 1x Frequenzspektrum 12x 50mm Röhren, großer Frequenzbereich
- 3: Eindimensionaler Festkörper / Bandlücken
    - QA4: Seite 3 mittig (Experiment), unten (Experiment): 3x Frequenzspektren mit 15x 50mm Röhren mit Blenden a 10, 13, 16mm Durchmesser
- 4: Eindimensionaler Festkörper / Anzahl und Breite der Einheitszellen, Gitterkonstante
    - QA4: Seite 4 mittig (Experiment): 1x Frequenzspektrum mit Blenden 16mm aber weniger Röhren (10x 50mm)
- 5: Eindimensionaler Festkörper / das selbe wie in Aufgabe 4, reziproker Raum, Bandlücken
    - QA4: Seite 4 mittig (Experiment): 2x Frequenzspektrum mit Blenden 16mm aber weniger Röhren, unterschiedliche Röhren (10x 50mm, 8 x 50mm)
- 6: Molekülkette Eigenzustände / ???
    - QA4: Seite 9 oben (Experiment): 1x Einzelne Röhre (50mm)
- 7: Molekülkette Eigenzustände / ???
    - QA4: Seite 9 mittig (Experiment): 1x Einzelne Röhre (75mm)
- 8: Molekülkette Eigenzustände / Simulation einzelnes Molekül
    - QA4: Seite 9 mittig (Experiment): 1x 50mm+10er Blende+50mm
    - QA4: Seite 9 mittig (Experiment): 1x 50mm+13er Blende+50mm
    - QA4: Seite 9 mittig (Experiment): 1x 50mm+16er Blende+50mm
- 9: Molekülkette Eigenzustände / Simulation Molekülkette
    - QA4: Seite 9 unten (Experiment): "Molekülanzahl" erhöht, 9 verschiedene Anzahlen von Einheitszellen
- 10: Superstrukturen / Molekülkette mit Defekten
    - QA4: Seite 10 mittig (Experiment): 12x 50mm Röhren mit 13mm und 16mm Blenden (abwechselnd)
- 11: Superstrukturen / Molekülkette mit Defekten
    - QA4: Seite 10 mittig (Experiment): 5x 50mm und 5x 75mm (abwechselnd) mit 16mm Blenden
- 12: Defekte und Bandstruktur (Halbleiterdotierung)
    - QA4: Seite 11 oben (Experiment): 4 Frequenzspektren, 12x 50mm mit 16mm Blenden, Defekte (= 1x 12.5mm bzw. 1x 75mm ) ersetzen einzelne Röhren (Stelle 3 bzw Stelle 8)
