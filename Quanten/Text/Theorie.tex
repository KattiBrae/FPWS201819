%- grundsätzlicher Aufbau
Im Versuch werden in verschiedenen Aufbauten Röhren benutzt, in denen sich unter Verwendung von Lautsprechern akustische stehende Wellen ausbilden.
Die Röhre wird mit dem Mikrofon verschlossen.\\
%
%- Theorie Resonanz
In der Röhre bildet sich bei bestimmten Frequenzen das Phänomen der Resonanz.
Die Resonanzfrequenz ist diejenige Frequenz, bei der materialgegeben der Energieübertrag die geringsten Verluste hat.
Ein Objekt kann mehrere Resonanzfrequenzen haben.
Die Bedingung für Resonanz in einer Röhre mit geschlossenen Enden lautet
\begin{equation}
  n \lambda = 2 L = \frac{n c}{f} \Leftrightarrow f(n) = \frac{c}{2L} n.
  \label{eqn:resonanz}
\end{equation}
Hierbei ist $\lambda$ die Wellenlänge, $L$ die Länge der Röhre, $c$ die Schallgeschwindigkeit, $f$ die Schallfrequenz und $n$ die Angabe, um welches ganzzahlige Vielfache der Resonanz es sich handelt.\\
%
%- Schallwellen Herleitung?
Die Wellengleichung für Schallwellen lässt sich über die linearisierte Eulergleichung (Gleichung \eqref{eqn:euler}) und die Kontinuitätsgleichung (Gleichung \eqref{eqn:konti}) herleiten:
\begin{equation}
  \frac{ \partial \vec{u}}{ \partial t} = - \frac{1}{\rho} \nabla p
  \label{eqn:euler}
\end{equation}
\begin{equation}
  \frac{\partial \rho}{ \partial t} = - \rho \nabla \vec{u}
  \label{eqn:konti}
\end{equation}
\begin{equation}
  \frac{ \partial^2 p}{\partial t^2} = \frac{1}{\rho \kappa} \Delta p
  \label{eqn:schallwelle}
\end{equation}


%    - Dispersion Schallwellen
%- Potenzialtopf
%    - Dispersion QM
%- Analogien/Differenzen
%
%- Eindimensionaler Festkörper
%    - Bändermodell
%    - Teilchenkette (Lattice)
%    - Dispersionsrelation
%    - Bragg-Bedingung
%
%- Superstrukturen
%    - Defekte und Bandstrukturen
