%- grundsätzlicher Aufbau
Im Versuch werden in verschiedenen Aufbauten Röhren benutzt, in denen sich unter Verwendung von Lautsprechern akustische stehende Wellen ausbilden.
Die Röhre wird mit dem Mikrofon verschlossen.\\
%
\subsection{Stehende Schallwellen}
%- Theorie Resonanz
In der Röhre bildet sich bei bestimmten Frequenzen das Phänomen der Resonanz.
Die Resonanzfrequenz ist diejenige Frequenz, bei der materialgegeben der Energieübertrag die geringsten Verluste hat.
Ein Objekt kann mehrere Resonanzfrequenzen haben.
Die Bedingung für Resonanz in einer Röhre mit geschlossenen Enden lautet
\begin{equation}
  n \lambda = 2 L = \frac{n c}{f} \Leftrightarrow f(n) = \frac{c}{2L} n.
  \label{eqn:resonanz}
\end{equation}
Hierbei ist $\lambda$ die Wellenlänge, $L$ die Länge der Röhre, $c$ die Schallgeschwindigkeit, $f$ die Schallfrequenz und $n$ die Angabe, um welches ganzzahlige Vielfache der Resonanz es sich handelt.\\
%
%- Schallwellen Herleitung?
\begin{equation}
  \frac{ \partial \vec{u}}{ \partial t} = - \frac{1}{\rho} \nabla p
  \label{eqn:euler}
\end{equation}
\begin{equation}
  \frac{\partial \rho}{ \partial t} = - \rho \nabla \vec{u}
  \label{eqn:konti}
\end{equation}
%
Die Wellengleichung für Schallwellen lässt sich über die linearisierte Eulergleichung (Gleichung \eqref{eqn:euler}) und die Kontinuitätsgleichung (Gleichung \eqref{eqn:konti}) herleiten:
\begin{equation}
  \frac{ \partial^2 p}{\partial t^2} = \frac{1}{\rho \kappa} \Delta p
  \label{eqn:schallwelle}
\end{equation}
Für den Vergleich zum quantenmechanischen Teilchen sind die Randbedingungen wichtig.
Auch wenn es aus der Wellengleichung nicht direkt ersichtlich ist, ist die Schallgeschwindigkeit am Rand bzw. an der Wand am Ende der Röhre null.
Damit ist auch die Ableitung des Schalldrucks senkrecht zur Wand verschwindet:
\begin{align*}
  \frac{ \partial u }{\partial t} = 0 && \frac{ \partial p }{ \partial t} = 0.
\end{align*}
Die Wellengleichung lässt sich hier auf ein eindimensionales Problem reduzieren.
So lässt sich der Ansatz
\begin{equation*}
  p(x)=p_{0} \cos{(kx + \alpha)} \cos{(\omega t)}
\end{equation*}
verwenden.
Mit den Randbedingungen ergibt sich, dass $\alpha=0$ und $k= \frac{n \pi}{L}$.
%
%    - Dispersion Schallwellen
%
\subsection{Das quantenmechanische Teilchen}
%- Potenzialtopf
Die zeitabhängige Schrödingergleichung ist bekannt als
\begin{equation*}
  - \frac{ \hbar^2}{2 m} \psi(r, t) + V(r, t) \psi(r, t) = i \hbar \psi(r, t).
\end{equation*}
Die eindimensionale, zeitunabhängige Schrödingergleichung reicht aus, um den eindimensionalen, unendlich tiefen Potenzialtopf zu berechnen.
Der Potenzialtopf wird hier mit einer Wand an der Stelle $x=0$ und der zweiten Wand bei $x=L$ gewählt.
Da im Topf $V(x)=0$ gilt, ergibt sich:
\begin{equation}
  - \frac{ \hbar^2}{2 m} \psi(x) = E \psi(x).
  \label{eqn:schrödinger}
\end{equation}
Der passende Ansatz für diese Gleichung ist
\begin{equation}
  \psi(x)= A \sin{(kx+\alpha)}.
  \label{eqn:ansatz}
\end{equation}
Die Randbedingungen des Potenzialtopfs belaufen sich auf
\begin{align*}
    \psi(0)=0       && \psi(L)=0.\\
\end{align*}
Damit ergibt sich nach Einsetzen des Randbedingungen in den Ansatz in Gleichung \eqref{eqn:ansatz} und Einsetzen des Ansatzes in die Schrödingergleichung \eqref{eqn:schrödinger} folgendes:
\begin{equation}
  E(k) = \frac{\hbar^2 k^2}{2 m} = \frac{\hbar^2 \pi^2 n^2}{2 m L^2}.
  \label{eqn:dispersionqm}
\end{equation}
Dies ist die Dispersionsrelation des quantenmechanischen Teilchens im Potenzialtopf.
%    - Dispersion QM
%- Analogien/Differenzen
\subsection{Vergleich von stehenden Schallwellen und Teilchen im Potenzialtopf}
Während die Schallwellen mechanische Phänomene sind, ist das Teilchen im Potenzialtopf ein quantenmechanisches Problem.
Die Schallwellen sind damit genau lokalisierbar.
Beim quantenmechanischen Teilchen im Potenzialtopf hingegen lässt sich nur eine Aufenthaltswahrscheinlichkeit $|\psi^2|$ formulieren.\\
Auch die Randbedingungen sind unterschiedlich.
Während bei dem quantenmechanischen Teilchen die Wellenfunktion an den Rändern null sein muss, ist dies bei den stehenden Schallwellen nicht der Fall.
Zwar ist bei den Schallwellen die Geschwindigkeit $\vec{u}$ an den Rändern null, aber die vektorielle Geschwindigkeit entspricht nicht der skalaren quantenmechanischen Wellenfunktion.
Obwohl der Schalldruck $p$ an den Rändern maximal ist, trifft er das quantenmechanische Teilchen besser, da es eine skalare Größe ist.\\
Die Dispersionsrelation der Schallwellen $\omega(k)$ entspricht nicht der Dispersionsrelation des quantenmechanischen Teilchens $E(k)$, schon da $\omega ≠ k$.
Dennoch lassen sich die beiden Dispersionen vergleichen, da sich in beiden Phänomenen diskrete Niveaus ergeben.\\
Auch die Eigenzustände unterscheiden sich.
%Zeitunabhängige quantenmechanische Eigenzustände sind dauerhaft, während mechanische Eigenzustände instabil sind.
%Zeitabhängige Eigenzustände in der Quantenmechanik sind wiederum nicht beständig.
%Es ergibt sich ein Spektrum der Eigenfrequenzen, beschrieben durch folgende Gleichung:
%\begin{equation*}
%  |A(\omega)| = \frac{1}{2 \pi ( \lambda + i ())}
%\end{equation*}
%
%
%- Eindimensionaler Festkörper
%    - Bändermodell
%    - Teilchenkette (Lattice)
%    - Dispersionsrelation
%    - Bragg-Bedingung
%
%- Superstrukturen
%    - Defekte und Bandstrukturen
