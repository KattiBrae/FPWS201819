1. Hysterese fitten (linear, da nicht-lineare bereiche erst bei +16A und wir haben nur bis 15.5A hochgedreht)
2. Bilder auswerten




\subsection{Rot Normaler Zeeman-Effekt}

\subsection{Blau Anormaler Zeeman-Effekt}

\begin{table}[h!]
  \centering
  \caption{Messdaten zum anormalen Zeeman-Effekt}
  \label{tab:blau}
  \begin{tabular}{c c c c c c c c c c c}
    \toprule
%    \multicolumn{3}{c}{$f_{\text{1, theo}}=\SI{0.1}{m}$} & \multicolumn{3}{c}{$f_{\text{2, theo}}=\SI{0.05}{m}$}\\
      blau 1 & blau 2 & blau 3 & Verh 2/1 & Verh 3/1  & $\partial \lambda 1$ & $\partial \lambda 2$\\
      \midrule
      140   &   49    &   50    &   0.35                  &     0.35714285714285715     &   4.716603662557499e-12   & 4.812860880160714e-12    \\
      154   &   52    &   65    &   0.33766233766233766   &     0.42207792207792205     &   4.550341195788312e-12   & 5.687926494735389e-12    \\
      173   &   54    &   63    &   0.31213872832369943   &     0.36416184971098264     &   4.206384769250289e-12   & 4.90744889745867e-12     \\
      177   &   62    &   57    &   0.3502824858757062    &     0.3220338983050847      &   4.720410445174576e-12   & 4.339732183466949e-12    \\
      187   &   64    &   63    &   0.3422459893048128    &     0.33689839572192515     &   4.6121105332877e-12     & 4.5400463062050805e-12   \\
      197   &   68    &   69    &   0.34517766497461927   &     0.350253807106599       &   4.651617825292386e-12   & 4.720023969781979e-12    \\
      209   &   72    &   66    &   0.3444976076555024    &     0.3157894736842105      &   4.642453365743541e-12   & 4.255582251931578e-12    \\
      233   &   74    &   74    &   0.31759656652360513   &     0.31759656652360513     &   4.279934653945493e-12   & 4.279934653945493e-12    \\
      250   &   82    &   80    &   0.328                 &     0.32                    &   4.4201314323396e-12     & 4.312323348624e-12       \\
      282   &   89    &   90    &   0.31560283687943264   &     0.3191489361702128      &   4.25306713239734e-12    & 4.3008544035478725e-12   \\
      321   &   103   &   97    &   0.32087227414330216   &     0.30218068535825543     &   4.324078124107009e-12   & 4.072190078042523e-12    \\
      399   &   120   &   135   &   0.3007518796992481    &     0.3383458646616541      &   4.0529354780300746e-12  & 4.559552412783834e-12    \\
    \bottomrule
  \end{tabular}
\end{table}

blau1
\begin{equation*}
  \mu = \SI{4.45250571816 \pm 0.215951778182 e-12}{m}
\end{equation*}

blau2 mu= 4.56570632339e-12 +- 4.15188327987e-13

%\end{landscape}
%\end{document}
