Im Versuch werden die Übergänge des Cadmiums unter Aufspaltung durch den Zeeman-Effekt betrachtet.
Dem roten Licht liegt dabei der normale Zeeman-Effekt zugrunde, dem blauen Licht der anormale Zeeman-Effekt.\\
Der Landé-Faktor des Übergangs $g_{12}$ ergibt sich für den normalen Zeeman-Effekt (rot) zu
\begin{equation*}
  g_{ij}  = \SI{0.974 \pm 0.011}{}.
\end{equation*}
was mit einer relativen Abweichung von $f=\SI{2.65}{\%}$ zum theoretischen Landé-Faktor
\begin{equation*}
  g_{12, \text{theo}} = \SI{1}{}
\end{equation*}
des zirkularen Übergangs passt.
\begin{figure}[h!]
  \centering
  \includegraphics[width=\textwidth]{normal.png}
  \caption{Aufspaltung durch den normalen Zeeman-Effekt \cite{1}}
  \label{fig:normal}
\end{figure}
\FloatBarrier
Für die Übergänge des anormalen Zeeman-Effekts werden die Landé-Faktoren des Übergangs $g_{ij}=m_{1}g_{1}-m_{2}g_{2}$ berechnet und mit der Theorie verglichen.
Der theoretische Wert für $\sigma$ mittelt sich hier aus den Werten $g_{12, \text{theo}, 1}=2$ und $g_{12, \text{theo}, 2}=\frac{3}{2}$, da sich die Linien durch den optischen Doppler-Effekt stark überlagern und verbreitern, die Messung aber über die Mitte der Linie ausgeführt wird.
Wiederum der zweite theoretische Wert für den $\pi$-Übergang $g_{12, \text{theo}} = 0 $ ist stark unterdrückt und fällt im Mittel nicht ins Gewicht.
So ergibt sich:
\begin{align*}
  \sigma  &&& g_{12, \text{exp}} &= \SI{1.30 \pm 0.02}{}    &&&   g_{12, \text{theo}} &= \SI{1.75}{}     &&&  f=\SI{25.71}{\%}   \\
  \pi     &&& g_{12, \text{exp}} &= \SI{0.442 \pm 0.004}{}  &&&   g_{12, \text{theo}} &= \SI{0.5}{}   &&&  f=\SI{11.60}{\%}   .
\end{align*}
\begin{figure}[h!]
  \centering
  \includegraphics[width=\textwidth]{anormal.png}
  \caption{Aufspaltung durch den anormalen Zeeman-Effekt \cite{1}}
  \label{fig:normal}
\end{figure}
\FloatBarrier
Mögliche Erklärungen für die Abweichungen sind der bereits genannte optische Doppler-Effekt, verschmutzte Linsen, ein ungenauer Aufbau der Linsenapparatur, eine nicht optimale Auswertung der Bilder und eine Ungenauigkeit durch Ablesefehler an der Ampereanzeige des analogen Feldstromgenerators von ca. $\pm \SI{0.25}{A}$.
Tatsächlich ist auf den Bildern in der Nachbearbeitung ein Fingerabdruck auf der Lummer-Gehrke-Platte zu erkennen.
Eine gewisse Beeinträchtigung ist durch den Generator des Stroms für den Elektromagneten gegeben, da bei diesem häufig die Sicherung herausspringt.
So müssen einige Versuchsteile mehrfach durchgeführt werden.
Dies hat jedoch keine Auswirkung auf den Messwert.
Nicht auzuschließen ist jedoch, dass durch Unachtsamkeit und Stoßen an die Apparatur der Linsenaufbau geringfügig geändert wird, dieses wird jedoch so weit möglich unterbunden.
