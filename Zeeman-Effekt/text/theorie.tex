\subsection{Einleitung}
Der Zeeman-Effekt beschreibt die Aufsplatung und Polarisation von Spekrtallinien eines Atoms unter Einfluss eines äußeren Magnetfeldes.
Durch das Aufspalten der diskreten Energieniveaus kommt es bei der Lichtemmision zu kleinen unterschieden in der Wellenlänge.

\subparagraph{Magnetische Moment}
Hüllenelektronen können mit Bahndrehimpuls $\vec{l}$ und mit dem Eigendrehimpuls $\vec{s}$ beschrieben werden.
Dabei gilt:
\begin{align*}
  |\vec{l}|=\sqrt{l(l+1)}\hbar&& \text{mit } l &= 0,1,2,...,n-1\\
  |\vec{s}|=\sqrt{s(s+1)}\hbar&& \text{mit } s &= \frac{1}{2}
\end{align*}
Die magnetischen Momente welche durch die Drehimpulse und die Ladung der Elektronen entstehen können beschriben werden mit:
\begin{align*}
  \vec{\mu}_l &= -\mu_B \frac{\vec{l}}{\hbar} = -\mu_B \sqrt{l(l+1)}\vec{l}_e\\
  \vec{\mu}_s &= -g_S \frac{\mu_B}{\hbar}\vec{s} = -g_S \mu_B \sqrt{s(s+1)}\vec{s}_e\\
\end{align*}
$\vec{l}_e$ und $\vec{s}_e$ sind die Einheitsvektoren in die jeweilige Richung.
Die Größe $g_S$ ist der Lande´-Faktor. $\mu_B$ beschreibt das Borsche Magneton und ist dabei gegebne als:
\begin{align*}
  -\frac{1}{2} e_0 \frac{\hbar}{m_0}
\end{align*}
Weiter gilt, dass $e_0$ die Elementarladung und $m_0$ die Elektronenmasse beschreibt.

\subsection{Wechselwirkung der Drehimpulse und magnetischer Momente untereinader}
Für Atome mit mehreren Elektronen gibt es viele unterschiedliche Arten wie Bahndrehimpuls und Spin miteinander wechselwirken können.
Im wesnetlichen sind können zwei einfache Grenzfälle betrachtet werden, welche häufig in der Natur vorkommen.

Für Atome mit niedriger Kernladungszahl kann der Gesammtdrehimpuls $\vec{L}$ der Hülle aus den Bahndrehimpulsen $\vec{l}$ vektoriell zusammengesetzt werden.
Das liegt an der großen Wechselwirkung zwischen den Bahndrehimpulsen.
\begin{align*}
  \vec{L} = \sum_i{\vec{l}_i} \text{ mit } |\vec{L}|= \sqrt{L(L+1)}\hbar
\end{align*}
Für den Gesamtbahndrehimpulls müssen nur unabgeschlossene Schalen betrachtet werden, da abgschlossene Schalen immer einen Bahndrehimpulls von 0 besitzen.
$\vec{l}$ kann dabei nur ganzzahlige Quantenzahlen von 0,1,2 oder 3 annehmen.
Je nach Quantenzahl kann zwischen S,P,D und F-Term unterschieden werden.
Das magnetische Moment $\vec{\mu}_L$ vom Gesamtbahndrehimpuls $\vec{L}$ läst sich errechnen mit:
\begin{align*}
  |\vec{\mu}_L| = \mu_B\sqrt{L(L+1)}
\end{align*}

Für den Gesamtspin der Elektronenhülle $\vec{S}$ gilt für Atome mit nut zu hoher Ordnungszahl ebenfalls die verktorielle Summation der einzelnen Komponenten.
Die Einzelkomponenten sind hier die Einzelspins $\vec{s}_i$.
\begin{align*}
  \vec{S} = \sum_{i}{\vec{s}i}
\end{align*}
Die Gesamtspinquantenzahl $S$ kann der Werte $\frac{N}{2}, \frac{N}{2}-1, ..., \frac{1}{2},0$ annehmen.
$N$ beschribt dabei die Anzahl der Elektronen aus den unabgeschlossenen Schalen.
Der Betrag des Gesamtspins läst sich aufstellen zu:
\begin{align*}
  |\vec{S}| = \sqrt{S(S+1)}\hbar.
\end{align*}
Der dazugehörige Betrag des magnetischen Momentes ist gegeben als:
\begin{align*}
  |\vec{mu}_S| = g_S\mu_B\sqrt{S(S+1)}.
\end{align*}
Im Falle, dass das Atom keinem zu großen Magnetfeld ausgesetzt ist kann der Gesammtdrehimpuls $\vec{J}$ geschreiben werden als:
\begin{align*}
  \vec{J} = \vec{L}+ \vec{S}
\end{align*}
Die beschriebene LS-Kopplung ist für die Betrachtung des Zeeman-Effekts zugrunde gelegt.
$\vec{J}$ kann abhängig von $S$ ganz- oder habzahlig sein.
Der Betrag vom Gesamtdrehimpuls ist gegeben als:
\begin{align*}
  |\vec{J}| = \sqrt{J(J+1)}\hbar
\end{align*}
Beschreibt man ein Energie Nivau kann das mit
\begin{align*}
  {}^M\mathcal{L}_J
\end{align*}
erfolgen.
$M$ ist dabei von $S$ abhängig in der Form $M=2S+1$.
Für das Drehimpulssymbol $\mathcal{L}$ gilt:
$\mathcal{L}\in\{S(L=0), P(L=1), D(L=2), F(L=3)\}$.
Wobei $L$ wieder der Gesamtdrehimpuls ist.\\

Der zweite Grenzfall betrachtet die j-j-Kopplung bei Atomen mit höheren Kernladungszahlen.
Durch die starke Kopplung zwischen den Spin und den Bahndrehimpuls eines Einzelelektrons setzt sich der Gesamtdrehimpuls des Elektrons nun zusammen aus:
\begin{align*}
  \vec{j}_i = \vec{l}_i + \vec{s}_i.
\end{align*}
Der gesammt Drehimpuls der Elektronenhülle läst sich schreiben als:
\begin{align*}
  \vec{J}=\sum_i\vec{j}_i.
\end{align*}
Es bei dieser Betrachtung kann kein Gesammtdrehimpuls $\vec{L}$ oder ein Gesammtspin $\vec{S}$ deviniert werden.\\
Für Atome mit mittlerer Kernadungszahl besteht ein fließender Übergang zwischen den beiden Grenzfällen.

\subsection{Aufspaltung der Energienivaus eines Atoms im homogenen Magnetfeld}
Das magnetische Moment welches zum Gesammtdrehimpuls $\vec{J}$ gehört läst sich brerchnen mit
\begin{align*}
  \vec{\mu}_J&=\mu_Bg_J\sqrt{J(J+1)},
\end{align*}
wobei für den Landé-Faktor $g_J$ des entsprechneden Atoms gilt:
\begin{align}
		g_J:=&\frac{3J(J+1)+S(S+1)-L(L+1)}{2J(J+1)}.
\end{align}

\subsection{Energieaufspaltung und Übergänge}
Durch die Richtungsquantelung sind nur genau $2J+1$ Einstellungen des atomaren magnetischen Momentes zu der äußeren Feldrichtung möglich.
Die zusätzliche Energie die das Moment $\vec{\mu}$ im äußeren Magntfeld bekommt ist gegeben als:
\begin{align*}
  E_{\text{mag}} = -\vec{\mu}_J \cdot \vec{B} = mg_J\mu_B.
\end{align*}
Für die Orientierungsquantenzahl $m$ gilt $-J < m < J$.
Für den Fall, dass $B \ne 0$ Salltet sich also das Enaginiveau $E_0$ eines Atoms aus in $2J+1$ äqidistante Niveaus.
In der Abbildung %\ref{fig.J2}
ist diese aufspaltung für $J = 2$ Abgebildet.
%\begin{figure}[h!]
%	\centering
%	\includegraphics[width=0.6\textwidth]{Aufspaltung.pdf}
%	\caption{Aufspaltung eines Energieniveaus von einem Atom mit $J=2$. \cite{V27}}\label{fig:J2}
%\end{figure}
Die Aufspaltung führt bei Lichtemmision zur aufspaltung des Spektrums die wird als Zeeman-Effekt bezeichnet.

Da nur bestimmte Energieübergänge möglich sind gibt es die Auswahlregeln.

Für die festlegung der Auswahlregeln wird die zeitabhängige Schrödingergleichung benötigt.
\begin{align}
	-\frac{\hbar^2}{2m}\Delta \psi(\vec{r},t)+U\psi(\vec{r},t)-i\hbar\frac{\partial \psi(\vec{r},t)}{\partial t}=0
\end{align}
Die Lösungen $\psi$ beschreiben den Überganz zwischen den Energieniveaus $\alpha$ und $\beta$.
Aus den Lösungen ergibt sich eine Schwingung des Elektrons zwischen den beiden Energieniveaus mit der Frequenz:
\begin{align*}
  \nu_{\alpha\beta}:=\frac{E_\alpha-E_\beta}{h}
\end{align*}
Das Elektron läst sich dementsprechent als Dipol beschreiben welches in die $x$-Richtung mit:
\begin{align*}
  	D_x=-e_0\text{const } 2 \Re \real\left( \underbrace{\int x\psi^*_\beta\psi_\alpha dV}_{x_{\alpha\beta}}\exp(2\pi i\nu_{\alpha,\beta}t) \right)
\end{align*}
abstrahl.
Für die $y$ und $z$ Richtung kann die Formel analog aufgestellt werden.
Das Integal $x_{\alpha\beta}$ und seine alalogen $y$ und $z$ Komponenten werden Matxixelemente bezeichnet und sind wichtig für dei berechnung des Poynting-Vektors $\vec{S}_{\alpha\beta}$
Der Poyning-Vektor berechnet sich nach:
\begin{align*}
  |\vec{S}_{\alpha\beta}| \sim \left(|x_{\alpha\beta}|^2+|y_{\alpha\beta}|^2+|z_{\alpha\beta}|^2\right) \text{sin}^2(\gamma)
\end{align*}
$\gamma$ beschreibt dabei den Winkel zwischen Dipolmoment und Ausbreitungsrichtung der Strahlung.
Es kann gezeigt werden, dass die Intensität der vom Dipol emitierten Strahlung mit den Matixelementen zusammenhängt.
Für den Fall, dass das B-Feld in die Z-Richtung zeigt verschwindet $z_{\alpha\beta}$ außer wenn gilt, dass $m_{\alpha} = m_{\beta}$.
$x_{\alpha\beta}\pm i y_{\alpha\beta}$ verschwindet ebenfalls außer wenn gilt, dass $m{\beta} = m_{\alpha} \pm 1$.
Zum Zeeman-Effekt kommt es also nur wenn sich die Orienteirungsquantenzahlen $m_{\alpha}$ und $m_{\beta}$ garnicht oder nur um $\pm 1$ unterscheiden.\\

Für den Fall, dass $\Delta m = 0$ ($z_{\alpha\beta} \neq 0,\ x_{\alpha\beta} = i y_{\alpha\beta} = 0$) kommt es schwingung des Dipols parallel zu Magntfeldachse,
dies führt bei der Emmision zu linear-polarisiertem Licht parallel zum $\vec{B}$.
Durch die Polarisation kann das emmitierte Licht am besten senkrecht (Transversal) zur Feldrichunug beobachtet werden.
Die Strahlungsart wird als $\pi$ bezeichnet.

Für den Fall, das $\Delta m = \pm 1$ ($z_{\alpha\beta} = 0,\ x_{\alpha\beta} = \pm i y_{\alpha\beta} \neq 0$) kommt es zu links oder rechts zirkular-polarisierter Stahlung um dei Magnetfeldachse.
Bei betrachtung aus der Transversaler Achse zur Feldachse erscheit emmitiertes Licht linear polarisiert.
Die Strahlungsareten werden als $\sigma$ bezeichnet.

Die oben getroffenen Aussagen gelten nur für den Fall, dass $S=0$.
Diesen Spezialfall bezeichnet man als normalen Zeeman-Effekt.
Für Übergenge mit $S=0$ gilt $g_J = 1$. Die Verschiebung der Energieniveaus ist dementsprächend unabhängig von den Quantenzahlen.
Der Energieunterschied $\Delta E$ zwischen den niveaus ist unabhängig von $L$ und $J$ gleich groß.
\begin{align*}
  \Delta E = m \mu_B B \text{ für } -J \leq m \leq J
\end{align*}

Der annormale Zeeman-Effekt kommt deulich häufiger vor und tritt auf wenn $S \neq 0$.
Es gelten für die Übergänge die selben Auswahlregeln $\Delta m = 0, \pm 1$.
Da $g_J = 1$ nicht mehr gegeben ist, ergeben sich für die Übergänge die Energien von:
\begin{align*}
	E=(m_ig_{J_i}-m_jg_{J_j})\mu_BB+E_0.
\end{align*}
