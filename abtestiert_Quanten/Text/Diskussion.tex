Die berechnete Schallgeschwindigkeit von $\SI{343.28 \pm 1.63}{\frac{m}{s}}$ weicht nicht signifikant vom Literaturwert $\SI{343.2}{\frac{m}{s}}$ \cite{Physik} ab, da die Abweichung innerhalb des $2 \sigma$ Bereiches liegt.
%\textbf{\huge{Wie sieht der Literaturwert nochmal aus?, Verweis?, Ergebnis hier ausschreiben, und vergleichen (relativen Fehler z.B.?)}}

Da in den übrigen Aufgaben nichts berechnet wird, ist schwer zu sagen, ob die gemessenen Werte zufriedenstellend sind.
Die Dispersionsrelationen der stehenden Schallwelle und des quantenmechanischen Teilchen passen wie erwartet nicht zueinander (Abb. \ref{fig.Aufgabe2}).
Die quantenmechanische Dispersionsrelation steigt mit $k^2$, während die akustische mit $k$ ansteigt.\\
%\textbf{\huge{Welche Ergebnisse?}}
\\In Aufgabe 3 wird deutlich, dass sich die Breite der Bandlücken $\Delta f$ mit steigender Blendengröße $d$ wie folgt verhält (Abb. \ref{fig.Aufgabe3}):
\begin{equation*}
  \Delta f \propto - d.
\end{equation*}
Die Blendengröße $d$ stellt hier ein Analogon zur Barrierenhöhe zwischen zwei Atomen dar.
\\Bei der Untersuchung des Einflusses der Gesamtrohrlänge $L$ auf die Bandlückenbreite $\Delta f$ lässt sich sagen, dass ein möglicher Einfluss entweder äußerst gering ist, oder die gesamte Rohrlänge hat keinen Einfluss auf die Breite der Bandlücken (Abb. \ref{fig.Aufgabe4}).
\\Bei Aufgabe 5 werden die Wellenzahlen $k$ gegen die Frequenz $f(k)$ geplottet (Abb. \ref{fig.Aufgabe5}, \ref{fig.Aufgabe575}).
Hier ist insbesondere die Ähnlichkeit zu den Bildern der Bändern und Bandlücken aus der Theorie (Abb. \ref{fig:bändermodell}, \ref{fig:bändermodell2}) hervorzuheben.
In den Darstellungen sind die auftretenden Bandlücken farbig hinterlegt.
\\In Aufgabe 8 wird der Verlauf der Bandlückenbreite in Abhängigkeit der Blendenbreite in Verbindung gebracht mit der Bindung zweier Atome in einem Festkörper.
Auch hier lässt sich der Zusammenhang
\begin{equation*}
  \Delta f \propto - w
\end{equation*}
aufstellen, wobei hier $w$ die Stärke der Bindung beschreibt.
\\Nun wird in Aufgabe 9 die Bandlückenbreite in Abhängigkeit der Anzahl der Einheitszellen betrachtet (Abb. \ref{fig.Aufgabe9}, \ref{fig.Aufgabe91}, \ref{fig.Aufgabe92}).
Hier wird die Rohrlänge übersetzt in eine Anzahl von Einheitszellen.
Es wird deutlich, dass die Bandlückenbreite $\Delta f$ mit der Anzahl $j$ der betrachteten Einheitszellen ansteigt:
\begin{equation*}
  \Delta f \propto j.
\end{equation*}
\\Anschließend wird in Aufgabe 10 der Effekt von periodischen Defekten, hier abwechselnde Blendenbreiten, untersucht (Abb. \ref{fig.Aufgabe101}).
In dem Graphen ist $f(k)$ geplottet, und es ist deutlich zu erkennen, dass mit einem erkennbaren Muster zwischen den 'regulären' Bandlücken neue, schmale Bandlücken auftauchen.
\\Danach wird der Effekt von abwechselnden Rohrlängen untersucht (Abb. \ref{fig.Aufgabe111}).
Hier ist kein Muster erkennbar, nach dem sich die zusätzlichen Bandlücken bilden, aber die Defekte haben einen großen Einfluss auf die Bandstruktur.
Die entstehenden Bänder und Lücken entsprechen einer Zusammensentzung der beiden Konfigurationen.
\\Hingegen bei Aufgabe 12 soll der Einfluss von einzelnen Defekten beleuchtet werden (Abb. \ref{fig.Aufgabe12}).
Durch das Einbauen eines Defektes entsteht ein Peak in der ersten Bandlücke.
%Auffällig ist, dass es keine nennenswerten Unterschiede in den Graphen gibt, obwohl alle vier Messungen mit unterschiedlichen Defekten an unterschiedlichen Stellen gemacht werden.
%Dies führt zu der Interpretation, dass ein einzelner Defekt nicht zwingend einen großen Einfluss auf die Eigenschaften eines Festkörpers hat.
\\Bei der Messung kommt es jedoch zu einigen Ungenauigkeiten.
Zum einen hat das Mikrofon auch Geräusche aus der Umgebung aufgenommen, so kann es zu Fehlmessungen gekommen sein.
Zum anderen ist die Peak-Picking Funktion nicht fortgeschritten, sodass auch Maximalwerte markiert werden, welche global keine Relevanz haben.
Auffällig war auch, dass das Messprogramm veraltet war und nach jeder Messung abgestürzt ist.
