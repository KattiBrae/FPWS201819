%\section{Hysterese des Magneten}
%Vor der Aufnahme der roten und blauen Spekrallinie der Cd-Spekrallampe,
%muss zunächst die Hysterese des verwendeten Elektromagneten aufgenommen
%werden. Die aufgenommenen Messwerte für die, mittels Hall-Sonde gemessene,
%magnetische Flussdichte sind in Abhängigkeit der eingestellten Stromstärke
%in den Tabellen \ref{tab:hysterese_zunehmend} und \ref{tab:hysterese_abnehmend}
%aufgeführt. Dabei wurde die Stromstärke während der erste Messung sukzessive
%erhöht und während der zweiten verringert.
%Desweiteren wurden die Messwerte in \cref{fig:hysterese_messung_i_zunehmend}
%respektive \cref{fig:hysterese_messung_i_abnehmend} grafisch dargestellt.
%Für den Vergleich der beiden Messungen wurden jeweils beide Messreihen
%dargestellt, jedoch wurden die zusätzlich dargestellten Regressionsgraden
%nur auf Grundlage jeweils einer Messung bestimmt.
%
%\input{../Tabellen/Hysterese_zunehmend}
%\input{../Grafiken/fig_Hysterese_Messung_I_zunehmend}
%\input{../Tabellen/Hysterese_abnehmend}
%\input{../Grafiken/fig_Hysterese_Messung_I_abnehmend}
%
%Für die Bestimmung der Regressionsgeraden, mit der
%\emph{Methode der kleinsten Quadrate} wurde der Ansatz
%\begin{empheq}{equation}
%    B(I) = a \cdot I + b
%\end{empheq}
%verwendet. Die Parameter für die beiden Messreihen ergeben sich zu
%\addtocounter{equation}{-1}
%\begin{subequations}
%  \begin{empheq}{align}
%    a_{\mathrm{zu}} &= \SI{56(1)}{\milli\tesla\per\ampere} & b_{\mathrm{zu}} &= \SI{30(10)}{\milli\tesla}\\
%    a_{\mathrm{ab}} &= \SI{54.6(8)}{\milli\tesla\per\ampere} & b_{\mathrm{ab}} &= \SI{37(8)}{\milli\tesla}.
%  \end{empheq}
%\end{subequations}
%%a_ab =  54.60268320943114 +/- 0.8066786330934029
%%b_ab =  37.43274821608795 +/- 8.03310381999233
%%a_in =  56.16821466417253 +/- 1.030043435313515
%%b_in =  29.07017534402186 +/- 10.257426200589356
%
\section{Bestimmung der Landé-Faktoren}

Die Aufnahmen der roten und blaues Spektrallinien wurden
unter dem Einfluss unterschiedlicher magnetischer Flussdichten
aufgenommen. Dabei wurden diese unter Verwendung der zuvor bestimmten
Regressionsgerade (für zunehmenden Strom) berechnet. Die entsprechenden Werte
sind mit Angabe der jeweiligen Messung in %\ref{tab:strom_magnetfeld} eingetragen.
%\input{../Tabellen/Magnetfeld_Aufnahmen}
Die Abbildungen \ref{fig:1-i0_t15_sigma_2-i10_t15_sigma},
\ref{fig:3-i0_t8_sigma_4-i5_t8_sigma}
und \ref{fig:5-i0_t8_pi_6-i17_t8_pi} zeigen jeweils eine Aufnahme einer
Spekrallinie bei ausgeschaltetem Magnetfeld im oberen Teil und eine Aufnahme
mit eingeschaltetem Magnetfeld im unteren. Dabei entsprechen die Abbildungen den
Spekrallinien rot-$\sigma$, blau-$\sigma$ und blau-$\pi$.
Aus diesen Abbildungen wurden wie in \ref{fig:messmethode} veranschaulicht, die Abstände
$\Delta s$ und $\delta s$ der unaufgespaltenen respektive der aufgespaltenen
Linien, durch Ausmessen, bestimmt. Die entspechenden Werte sind in den Tabellen
\ref{tab:linienverschiebung_rot_sigma}, \ref{tab:linienverschiebung_blau_sigma}
und \ref{tab:linienverschiebung_blau_pi} eingetragen. Neben den Abständen
$\Delta s$ und $\delta s$ ist auch das Verhältnis dieser beiden Größen angegeben.
Unten Verwendung dieser Verhältnisse und der berechneten Dispersionsgebiete
$\Delta \lambda_{\mathrm{D}}$ in \ref{tab:Lummer-Gehrcke} können mit
%\begin{equation}
%   \delta \lambda = \frac{1}{2}\frac{\delta s}{\Delta s} \Delta \lambda_{\mathrm{D}}
%   \label{eq:wellenlaengenaenderung}
%\end{equation}
die ebenfalls angegebenen Wellenlängenverschiebungen $\delta \lambda$ berechnet
werden.
%
%\input{../Grafiken/fig_Messmethode}
%\input{../Grafiken/fig_1-I0_t15_sigma_2-I10_t15_sigma}
%\input{../Tabellen/Linienverschiebung_rot_sigma}
%\input{../Grafiken/fig_3-I0_t8_sigma_4-I5_t8_sigma}
%\input{../Tabellen/Linienverschiebung_blau_sigma}
%\input{../Grafiken/fig_5-I0_t8_pi_6-I17_t8_pi}
%\input{../Tabellen/Linienverschiebung_blau_pi}
%
Aus den berechneten Wellenlängenänderungen $\delta \lambda$
wird für jede der drei Spekrallinien eine mittlere Wellenlängenänderung $\overline{\delta \lambda}$
berechnet. Diese ergeben sich zu:
%
%\begin{empheq}{gather*}
%  \overline{\delta \lambda}_{\mathrm{rot}-\sigma} = \SI{12.3(1)}{\pico\meter} \\
%  \overline{\delta \lambda}_{\mathrm{blau}-\sigma} = \SI{6.81(8)}{\pico\meter} \\
%  \overline{\delta \lambda}_{\mathrm{blau}-\pi} = \SI{5.83(8)}{\pico\meter}
%\end{empheq}
%
%(1.23+/-0.01)e-11 m
%(6.81+/-0.08)e-12 m
%(5.83+/-0.08)e-12 m
%
Aus diesen mittleren Wellenlängenänderungen $\overline{\delta \lambda}$ kann nun
eine Energieänderung $\Delta E$ bestimmt werden. Die Energie ist dabei eine Funktion
der Wellenlänge der Form $E(\lambda) = \mathrm{hc}\lambda^{-1}$.
Die Energieänderung $\Delta E$ ergibt sich nach
%
%\begin{align}
%  \notag
%  \Delta E &= |E(\lambda_{0} + \overline{\delta \lambda}) - E(\lambda_{0})|\\
%  \notag
%  &= \left|E(\lambda_{0}) + \frac{\partial E}{\partial \lambda}\overline{\delta \lambda} - E(\lambda_{0})\right|\\
%  \notag
%  &= \left|\frac{\partial E}{\partial \lambda}\overline{\delta \lambda}\right|\\
%  \label{eq:energieaenderung}
%  &= \frac{\mathrm{hc}}{\lambda^{2}}\overline{\delta \lambda}
%\end{align}
mit den ensprechenden Wellenlängen $\lambda$ der Spekrallinien aus \ref{tab:Lummer-Gehrcke}.
Nach \ref{eq:anormaler_zeemann_energie} ergibt sich der Landé-Faktor des
Übergangs $g_{ij}$ aus der Energieänderung $\Delta E$ durch die Umformung
%
%\begin{align}
%  \notag
%  \Delta E &= g_{ij} \cdot \mu_{\mathrm{B}}\cdot B\\
%  \label{eq:lande_energieaenderung}
%  \Leftrightarrow g_{ij} &= \frac{\Delta E}{\mu_{\mathrm{B}}\cdot B}.
%\end{align}
%
%
\begin{table}[h!]
	\centering
	\begin{tabular}{l c c c c }
		\toprule
		Messung & Energieänderung & Lande-Faktor & Lande-Faktor (theo.) & rel. Abweichung\\
		 & $\Delta E$/\si{10\tothe{-24}\joule} & $g_{ij}$ & $g_{ij,\text{theo}}$ & $\frac{|g_{ij} - g_{ij,\text{theo}}|}{g_{ij,\text{theo}}}$/\si{\percent}\\
\midrule
		rot $\sigma$ & \num{5.91(7)} & \num{1.08(5)} & \num{1} & \num{7(5)}\\
		blau $\sigma$ & \num{5.87(7)} & \num{2.0(2)} & \num{2} & \num{2(10)}\\
		blau $\pi$ & \num{5.02(7)} & \num{0.55(2)} & \num{0.5} & \num{10(4)}\\
		\bottomrule
	\end{tabular}
%	\caption{Unter Verwendung von \cref{eq:energieaenderung} aus der mittleren Wellenlängenveränderung $\mean{\delta \lambda}$
%                          berechnete Energieänderung $\Delta E$ und die entsprechenden Lande-Faktoren nach \cref{eq:lande_energieaenderung}.
%                          Zusätzlich sind die theoretischen Lande-Faktoren und die relativen Abweichungen
%                          der berechneten Lande-Faktoren angegeben. \label{tab:lande_ergebnis}}
\end{table}

