Die gemessenen Umdrehungen lassen sich mit der Gleichung
\begin{align*}
  1 \text{Umdrehung} = \SI{0,1}{A}
\end{align*}
umrechnen. Aus der resultierenden Stromstärke kann das B-Feld berrechnet werden.
\begin{align*}
  B=\mu_0\frac{8}{\sqrt{125}}\frac{N\cdot I}{R}
\end{align*}
Die Errechnete Werte sind zusammen mit den dazugehörigen Frequenzen in der Tabelle \ref{tab.Tab1} eizusehen.
\begin{table}
  \centering
  \caption{In Abhängigkeit der eingestellten Frequenz aufgenommene Stromstärken durch die beiden horizontalspulen. Die Messwerte der Sweep-Spule sind dabei mit (S) und die der Horizontal-Feld-Spule mit H gekenzeichnet. Aufgenommen wurden die Stromstärke an den Resonanzstellen für die beiden Isotope (1) und (2), mit dem gegebenen Maßen der Spulen wurde das horizontale Gesamtfeld aus den Stromstärken bestimmt.}
  \label{tab,Tab1}
    \begin{tabular}{c c c c c c c}
      \toprule
      Frequenz  & Stromstärke & Stromstärke & B-Feld 1 & Stromstärke & Stromstärke &  B-Feld 2\\
        & Horizontalspule & Sweep-Spule & & Horizontalspule & Sweep-Spule&\\
      kHz & A& A& $\mu$T & A& A& $\mu$T\\
      \midrule
      \midrule
      100     & 0,000  &  0,501 &  30,234  &  0,000 &  0,621 &  37,476 \\
      200     & 0,024  &  0,432 &  47,117  &  0,024 &  0,677 &  61,902 \\
      300     & 0,045  &  0,451 &  66,680  &  0,045 &  0,833 &  89,733 \\
      400     & 0,060  &  0,400 &  76,757  &  0,060 &  0,880 & 105,724 \\
      500     & 0,081  &  0,324 &  90,587  &  0,081 &  0,907 & 125,769 \\
      600     & 0,093  &  0,390 & 105,093  &  0,111 &  0,824 & 147,069 \\
      700     & 0,111  &  0,342 & 117,982  &  0,138 &  0,769 & 167,429 \\
      800     & 0,114  &  0,549 & 133,105  &  0,180 &  0,510 & 188,631 \\
      900     & 0,138  &  0,440 & 147,574  &  0,204 &  0,544 & 211,730 \\
      1000    & 0,144  &  0,617 & 163,518  &  0,234 &  0,503 & 235,565 \\
      \bottomrule
    \end{tabular}
\end{table}

\FloatBarrier
In der Graphik \ref{fig:Bf} sind die werde Graphisch dagestellt.
Es wird eine Ausgleichsrechnung der Form:
\begin{align*}
  B(f)=af+b
\end{align*}
durchgeführt.
\begin{figure}[h!]
  \centering
  \includegraphics[width=0.8\textwidth]{B-Feld.pdf}
  \caption{Das für die Resonanzstelle benötigte Magnetfeld aufgetragen auf die Frequenz}
  \label{fig:Bf}
\end{figure}
\begin{align*}
  B1&\\
  a_1&=\SI{1.438\pm 0.023e-10}{\frac{T}{Hz}}\\
  b_1&=\SI{1.876\pm 0.144e-5}{T}
\end{align*}
\begin{align*}
  B2&\\
  a_2&=\SI{2.141\pm0.031e-10}{\frac{T}{Hz}}\\
  b_2&=\SI{1.935\pm0.190e-5}{T}
\end{align*}

Mit der Steigung kann wie folgt der Landesche $g_F$-Faktor bestimmt werden.
\begin{align*}
  g_F = \frac{h}{\mu_0\cdot a_i}
\end{align*}
Somit ergeben sich die Landeschen Faktoren zu:
\begin{align*}
  g_{F1} = 0.4967800215411603\\
  g_{F2} = 0.3337255596982711.
\end{align*}

\begin{align*}
  g_J = \frac{(g_s+1)\cdot J\cdot(J+1)+(g_s-1)\cdot[S\cdot (S+1)-L\cdot (L+1)]}{2\cdot J \cdot (J+1)}
\end{align*}
$g_s$ ist gegeben als
\begin{align*}
  g_s = 2,0023
\end{align*}
Die Quantenzalen in dem Versuch sind gegebnen als
\begin{align*}
  S = \frac{1}{2}, L=0, J=\frac{1}{2}, F = I+J
\end{align*}
Durch einsezten ergiegt sich, dass
\begin{align*}
  g_J=g_s
\end{align*}
ist.
Mit Hilfe der Formel
\begin{align*}
  I = \frac{1}{2}\left(\frac{g_J}{g_F}-1\right)
\end{align*}
ergeben sich Kernspinzahlen von
\begin{align*}
  I_1=1.515278305464324\\
  I_2=2.499920056783072
\end{align*}
