\subsection{Aufbau der Messapparatur}
- Spektrallampe\\
- Sammelline/Kollimator\\
- $D_{1}$-Interferenzfilter\\
- Polarisationsfilter + $\lambda/4$-Platte\\
- Dampfzelle\\
  --- Heizer\\
- Helmholtzspulenpaare\\
  --- Vertikalfeld\\
  --- Horizontalfeld\\
  --- Sweepfeld\\
  --- RF-Feld mit Frequenzgenerator (Sinusspannung)\\
- Kollimator\\
- Photodiode\\
- Verstärker\\
- Oszilloskop\\


\subsection{Vorbereitung}
- Intensitätsmaximum der optischen Elemente auf die Photodiode bringen\\
- Ausrichten des Tisches mit der Messapparatur\\
- Vertikalfeld erhöhen bis der Peak auf dem Oszilloskop möglichst schmal ist\\

\subsection{Messung der Resonanzstellen}
- RF-Frequenz setzen\\
- B-Feld der Sweep-Spule erhöhen, um Resonanzstelle des B-Felds zu finden\\
- B-Feld propotional zu den Umdrehungen des verwendeten Potentiometers, Strom durch Potentiometerumdrehungen ablesen\\
-
