\begin{table}
  \centering
  \caption{Der Quadratische Zeemaneffekt zu den beiden Isotopen}
  \label{tab.QZ}
    \begin{tabular}{c c c c c}
      \toprule
      Frequenz  &  B-Feld 1 & Quad. Zeeman  & B-Feld 2  &Quad. Zeeman\\
      kHz & $\mu$T & 1$\cdot10^{-20}$ & $\mu$T  & 1$\cdot10^{-20}$ \\
      \midrule
      \midrule
      100     &   30,234 &  -1,271\pm0,021   &  37,476  &   -1,998\pm0,029     \\
      200     &   47,117 &  -3,099\pm0,050   &  61,902  &   -5,464\pm0,078     \\
      300     &   66,680 &  -6,219\pm0,100   &  89,733  &  -11,491\pm0,164     \\
      400     &   76,757 &  -8,246\pm0,133   & 105,724  &  -15,957\pm0,228     \\
      500     &   90,587 & -11,493\pm0,185   & 125,769  &  -22,587\pm0,323     \\
      600     &  105,093 & -15,477\pm0,250   & 147,069  &  -30,891\pm0,442     \\
      700     &  117,982 & -19,512\pm0,315   & 167,429  &  -40,041\pm0,573     \\
      800     &  133,105 & -24,843\pm0,401   & 188,631  &  -50,829\pm0,727     \\
      900     &  147,574 & -30,545\pm0,493   & 211,730  &  -64,047\pm0,916     \\
      1000    &  163,518 & -37,510\pm0,605   & 235,565  &  -79,284\pm0,113     \\
      \bottomrule
    \end{tabular}
\end{table}
