\begin{table}
  \centering
  \caption{In Abhängigkeit der eingestellten Frequenz aufgenommene Stromstärken durch die beiden Horizontalspulen. Aufgenommen wurden die Stromstärke an den Resonanzstellen für die beiden Isotope (1) und (2), mit dem gegebenen Maßen der Spulen wurde das horizontale Gesamtfeld aus den Stromstärken bestimmt.}
  \label{tab.Tab1}
    \begin{tabular}{c c c c c c c}
      \toprule
      Frequenz  & Stromstärke & Stromstärke & B-Feld 1 & Stromstärke & Stromstärke &  B-Feld 2\\
        & Horizontalspule & Sweep-Spule & & Horizontalspule & Sweep-Spule&\\
      kHz & A& A& $\mu$T & A& A& $\mu$T\\
      \midrule
      \midrule
      100     & 0,000  &  0,501 &  30,234  &  0,000 &  0,621 &  37,476 \\
      200     & 0,024  &  0,432 &  47,117  &  0,024 &  0,677 &  61,902 \\
      300     & 0,045  &  0,451 &  66,680  &  0,045 &  0,833 &  89,733 \\
      400     & 0,060  &  0,400 &  76,757  &  0,060 &  0,880 & 105,724 \\
      500     & 0,081  &  0,324 &  90,587  &  0,081 &  0,907 & 125,769 \\
      600     & 0,093  &  0,390 & 105,093  &  0,111 &  0,824 & 147,069 \\
      700     & 0,111  &  0,342 & 117,982  &  0,138 &  0,769 & 167,429 \\
      800     & 0,114  &  0,549 & 133,105  &  0,180 &  0,510 & 188,631 \\
      900     & 0,138  &  0,440 & 147,574  &  0,204 &  0,544 & 211,730 \\
      1000    & 0,144  &  0,617 & 163,518  &  0,234 &  0,503 & 235,565 \\
      \bottomrule
    \end{tabular}
\end{table}
